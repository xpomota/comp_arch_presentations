\documentclass[aspectratio=169,14pt]{beamer}

\usepackage[utf8]{inputenc}
\usepackage[main=russian,english]{babel}
\usepackage[T1]{fontenc}
\usepackage[labelformat=empty]{caption}
\usepackage{multimedia}
\usepackage{xcolor}
\usepackage{hyperref}
\usepackage{setspace}

\usetheme{Singapore}

\definecolor{urlcolor}{HTML}{799B03} % цвет гиперссылок
\hypersetup{urlcolor=urlcolor, colorlinks=true}

\graphicspath{{../Images/}}

\title{История развития ЭВМ}

\begin{document}

% Слайд ДаВинчи, Шиккард
\begin{frame}{История развития вычислительной техники}
    \begin{itemize}
        \item Леонардо~Да~Винчи~--- эскизы суммирующей машины (тринадцать разрядов, десятичная, на зубчатых колесах)
        \pause
        \item 1623~год: Вильгельм Шиккард (профессор математики)~--- десятичный механический калькулятор, шесть разрядов (<<часы для счета>>)
    \end{itemize}
    \begin{columns}[T,onlytextwidth]
        \begin{column}{0.5\textwidth}
            \begin{figure}[htp]
                \centering
                \includegraphics[width=4cm,keepaspectratio]{SchickardCalculatingClock.png}
                \caption{\tiny{Изображение из письма Иоганну Кеплеру}}
                \label{fig:SchickardCalculatingClock}
            \end{figure}
        \end{column}
        \begin{column}{0.5\textwidth}
            \begin{figure}[htp]
                \centering
                \includegraphics[height=2.5cm, keepaspectratio]{SchickardMachine.png}
                \caption{\tiny{<<Часы для счёта>>~--- воссозданная модель}}
                \label{fig:SchickardMachine}
            \end{figure}
        \end{column}
    \end{columns}
\end{frame}

% Слайд Паскалина
\begin{frame}{История развития вычислительной техники}
    \begin{itemize}
        \item 1642-1645 Блез Паскаль~--- <<Паскалина>>, шесть и восемь разрядов, для сложения и вычитания десятичных чисел. Создано около 50 экземпляров
    \end{itemize}
    \begin{columns}[T,onlytextwidth]
        \begin{column}{0.5\textwidth}
            \begin{figure}[htp]
                \centering
                \includegraphics[width=6cm,keepaspectratio]{Pascalina.png}
                \caption{<<Паскалина>>~--- один из сохранившихся экземпляров}
                \label{fig:Pascalina}
            \end{figure}
        \end{column}
        \begin{column}{0.5\textwidth}
            % \includemedia[
            %     activate=pageopen,width=3cm,height=3cm,keepaspectratio,
            %     addresource=../Videos/Pascalina_adder_scale.mp4,
            %     flashvars={%
            %           source=../Videos/Pascalina_adder_scale.mp4
            %           &loop=true}
            %   ]{}{VPlayer.swf}
            % % https://www.youtube.com/watch?v=3h71HAJWnVU
            \movie[externalviewer]{Сложение чисел в <<Паскалине>>}{../Videos/Pascalina_adder_scale.mp4}
        \end{column}
    \end{columns}
\end{frame}

% Слайд Лейбниц
\begin{frame}{История развития вычислительной техники}
    \begin{itemize}
        \item 1673 Готфрид Лейбниц~--- <<Арифметический прибор>>, первый арифмометр, позволяющий
        выполнять все арифметические действия
    \end{itemize}
    \begin{figure}[htp]
        \centering
        \includegraphics[width=6cm,keepaspectratio]{LeibnizArif.png}
        \caption{\href{https://www.youtube.com/watch?v=v6ruUDIeK6I}
         {\glqq Die Vierspezies-Rechenmaschine\grqq}}
        \label{fig:LeibnizArif}
    \end{figure}
\end{frame}

% Слайд Гаспар Де Прони
\begin{frame}{История развития вычислительной техники}
    \begin{itemize}
        \item 1755-1839 Гаспар Де Прони, французский математик и инженер. Вычисления в три этапа:
        \begin{itemize}
            \item вывод математических выражений, пригодных для численных вычислений;
            \item вычисление с помощью этих выражений значений для аргументов, отстоящих
            на 5-10 интервалов и занесение их в таблицу;
            \item вычисление остальных значений.
        \end{itemize}
        Прони <<...впервые в истории математики великолепным образом провел научную организацию труда,
        сведя аналитическую работу к <<иерархии>> элементарных операций>>.
    \end{itemize}
\end{frame}

% Слайд Беббидж
\begin{frame}{История развития вычислительной техники}
    \begin{itemize}
        \item Чарльз Беббидж~--- <<Аналитическая машина>>
        \begin{itemize}
            \item \textbf{Арифметическое устройство} (мельница, mill)~--- для выполнения операций над
            переменными, а также хранения в регистрах значений переменных, с которыми в данный
            момент осуществляется операция. Регистры представляли собой набор десятичных колес,
            по одному на разряд.
            \item \textbf{Запоминающее устройство} (склад, store)~--- для хранения как значений переменных,
            так и результатов операций;
            \item \textbf{Устройство управления}~--- для управления последовательностью операций,
            помещением переменных на склад и извлечение их из склада.
            \item \textbf{Устройство ввода и вывода данных}
        \end{itemize}
    \end{itemize}
\end{frame}

% Слайд Беббидж - Ада Байрон
\begin{frame}{История развития вычислительной техники}
    \begin{itemize}
        \item ~<<Аналитическая машина>> Беббиджа
        \begin{itemize}
            \item Управление по аналогии с управлением ткацким станком
            Жаккарда (прообраз перфокарт)
            \item Идея условной передачи управления
            \item Вывод: печать или пробивка перфокарт
        \end{itemize}
        \item Ада Байрон Лавлейс
        \begin{itemize}
            \item \href{https://habr.com/ru/company/wolfram/blog/303552/}{Описала} <<Аналитическую машину>> Беббиджа
            \item Привела пример использования (первая программа)
            \item Идеи <<рабочая ячейка>>, <<цикл>> и ряд других
        \end{itemize}
    \end{itemize}
\end{frame}

% Слайд Герман Холлерит
\begin{frame}{История развития вычислительной техники}
    \begin{itemize}
        \item Конец 19 века Герман Холлерит <<Машина для переписи населения>>~--- <<Табулятор>>
        \begin{itemize}
            \item Исходные данные на перфокартах
            \item Сортировка перфокарт
            \item 1902 год: автоматическая подача перфокарт
            \item 1908 год: конструкция сумматора, использовавшаяся во многих
            счетно-аналитических машин
            \item Основатель компании, которая впоследствии образовала IBM
        \end{itemize}
    \end{itemize}
\end{frame}

% Слайд Конрад Цузе - Z1
\begin{frame}{История развития вычислительной техники}
    \begin{itemize}
        \item 1936 Конрад Цузе~--- Z1
        \begin{itemize}
            \item Бинарные числа, отдельные от памяти вычислитель и блок управления
            \item Механический привод
        \end{itemize}
    \end{itemize}
    \begin{figure}[htp]
        \centering
        \includegraphics[height=4.5cm,keepaspectratio]{Z1Emulator.png}
        \caption{\href{http://zuse-z1.zib.de/simulations/z1/adders/wgl/}{\scriptsize{Эмуляция сумматора машины <<Z1>>}}}
        \label{fig:Z1Emulator}
    \end{figure}
\end{frame}

% Слайд Конрад Цузе - Z3
\begin{frame}{История развития вычислительной техники}
    \begin{itemize}
        \item 1941 Конрад Цузе~--- Z3 (архитектурно повторяет Z1)
        \begin{itemize}
            \item Электромеханические реле
            \item Числа с плавающей запятой
            \item 9 инструкций
            \item механические микросеквенсеры для выполнения инструкций
        \end{itemize}
    \end{itemize}
    \begin{columns}[T,onlytextwidth]
        \begin{column}{0.5\textwidth}
            \begin{figure}[htp]
                \centering
                \includegraphics[height=3cm,keepaspectratio]{Z3ControlWheels.png}
                \caption{\tiny{Микросеквенсер}}
                \label{fig:Z3ControlWheels}
            \end{figure}
        \end{column}
        \begin{column}{0.5\textwidth}
            \begin{figure}[htp]
                \centering
                \includegraphics[height=3cm, keepaspectratio]{Z3Architecture.png}
                \caption{\tiny{Основные блоки Z3}}
                \label{fig:Z3Architecture}
            \end{figure}
        \end{column}
    \end{columns}
\end{frame}

% Слайд Конрад Цузе - архитектура Z3
\begin{frame}{История развития вычислительной техники}
    \begin{columns}[onlytextwidth]
        \begin{column}{0.3\textwidth}
            Z3~--- архитектура
        \end{column}
        \begin{column}{0.7\textwidth}
            \begin{figure}[htp]
                \centering
                \includegraphics[height=6cm, keepaspectratio]{Z3FullArhitecture.png}
                \caption{\tiny{Архитектура Z3}}
                \label{fig:Z3FullArhitecture}
            \end{figure}
        \end{column}
    \end{columns}
\end{frame}

% Слайд Алан Тьюринг
\begin{frame}{История развития вычислительной техники}
    \begin{itemize}
        \item 1912-1954 Алан Тьюринг~--- теоретическая <<Машина Тьюринга>>, абстрактное понятие
        \begin{itemize}
            \item Формализация понятия <<алгоритм>>
            \item Расширение конечного автомата, способна имитировать все исполнители
            \item Исполнители, для которых возможно имитировать
            Машину Тьюринга, называются полными по Тьюрингу (Turing complete)
        \end{itemize}
    \end{itemize}
\end{frame}

% Слайд ЭНИАК
\begin{frame}{История развития вычислительной техники}
    \begin{itemize}
        \item 1945 Джон Моучли, Преспер Эккерт <<ЭНИАК>>, Электронный цифровой
        интегратор и вычислитель (Electronics Numerical Integrator and Computer, ENIAC)
        \begin{itemize}
            \item Газонаполненные лампы
            \item Десятичная система, 20 разрядов
            \item Около 18000 ламп, 1500 реле, > 150 кВт потребление
            \item Частота 100 kHz, вычислительный такт 20 импульсов,
            один такт на сложение, 14 тактов на умножение
        \end{itemize}
    \end{itemize}
\end{frame}

% Слайд ЭНИАК детали
\begin{frame}{История развития вычислительной техники}
    \begin{itemize}
        \item ~<<ЭНИАК>>
        \begin{itemize}
            \item Ввод чисел в машину производился с помощью перфокарт
            \item Программное управление последовательностью выполнения
            операций осуществлялось с помощью штекеров и наборных полей
        \end{itemize}
    \end{itemize}
    \begin{columns}[T,onlytextwidth]
        \begin{column}{0.6\textwidth}
            \begin{figure}[htp]
                \centering
                \includegraphics[width=7cm,keepaspectratio]{ENIAC_FunctionalUnits.png}
                \caption{\tiny{Общая схема}}
                \label{fig:ENIAC_FunctionalUnits}
            \end{figure}
        \end{column}
        \begin{column}{0.4\textwidth}
            \begin{figure}[htp]
                \centering
                \includegraphics[height=3.2cm, keepaspectratio]{ENIAC_ACCFrontPanel.png}
                \caption{\tiny{Наборное поле}}
                \label{fig:ENIAC_ACCFrontPanel}
            \end{figure}
        \end{column}
    \end{columns}
\end{frame}

% Слайд принципы фон Неймана
\begin{frame}{История развития вычислительной техники}
    \begin{itemize}
        \item Принципы фон Неймана (1903-1957)
        \begin{itemize}
            \item Принцип однородности памяти
            \item Принцип адресности
            \item Принцип программного управления
            \item Принцип двоичного кодирования
        \end{itemize}
    \end{itemize}
    \begin{figure}[htp]
        \centering
        \includegraphics[height=4cm,keepaspectratio]{NeumannArch.png}
        \caption{\scriptsize{Архитектура фон Неймана}}
        \label{fig:NeumannArch}
    \end{figure}
\end{frame}

% Слайд основные исторические события
\begin{frame}{История развития вычислительной техники}
    \begin{figure}[htp]
        \centering
        \includegraphics[width=9.5cm,keepaspectratio]{MajorEvents.png}
        \caption{\scriptsize{Основные исторические события}}
        \label{fig:MajorEvents}
    \end{figure}
\end{frame}

% Слайд характеристики первых ЭВМ
\begin{frame}{История развития вычислительной техники}
    \begin{figure}[htp]
        \centering
        \includegraphics[height=6.5cm,keepaspectratio]{CompareEarlyComputers.png}
        \caption{\scriptsize{Характеристики первых ЭВМ}}
        \label{fig:CompareEarlyComputers}
    \end{figure}
\end{frame}

% Слайд предпосылки к созданию ЭВМ
\begin{frame}{История развития вычислительной техники}
    \begin{itemize}
        \item В конце 40-х годов двадцатого века только США, Англия и СССР
        были способны к созданию ЭВМ. Необходимые условия:
        \begin{itemize}
            \item наличие актуальных, масштабных задач, которые нельзя решить без ЭВМ;
            \item наличие технологической инфраструктуры и передовых технологий;
            \item наличие ВУЗов и преподавателей для подготовки кадров.
        \end{itemize}
    \end{itemize}
\end{frame}

% Слайд Брук, Рамеев
\begin{frame}{История развития вычислительной техники}
    \begin{itemize}
        \item 1948 Исаак Семенович Брук, Башир Искандарович Рамеев~--- заявка на
        изобретение <<Автоматическая цифровая вычислительная машина>>
        (Энергетический институт АН СССР):
        \begin{itemize}
            \item главный программный датчик машины \ldots включает отдельные элементы машины
            в соответствии с программой решения задачи;
            \item применяется релейно-кодовый принцип работы;
            \item применяется вспомогательная вычислительная машина
            с несколькими фиксированными программами, записанными на непрерывно
            вращающихся барабанах, для интерполирования и выполнения часто повторяющихся
            вычислительных операций;
            \item применяются отдельный сумматор, умножитель и делитель;
            \item применяются дешифраторы двоичного кода для управления работой машины.
        \end{itemize}
    \end{itemize}
\end{frame}

% Слайд Брук, Рамеев
\begin{frame}{История развития вычислительной техники}
    \begin{columns}[onlytextwidth]
        \begin{column}{0.6\textwidth}
            \begin{itemize}
                \item ~<<М1>>: блок---схема
                \begin{itemize}
                    \item АУ~--- арифметический узел
                    \item ГПД~--- главный программный датчик
                    \item ЭП~--- электронная память
                    \item МП~--- магнитная память
                    \item УВВ~--- узел ввода/вывода
                \end{itemize}
            \end{itemize}
            \begin{scriptsize}
            \begin{spacing}{1.5}
            В состав ГПД входило 12 типов блоков: генератор тактирующих импульсов,
            блок пуска и синхронизации, распределитель импульсов, блок формирования
            импульсов, регистр адреса, пусковой регистр, селекционный регистр,
            регистр сравнения, блок операций и шифра, клапанный блок, блок выбора памяти,
            блок операции сравнения.
            \end{spacing}
            \end{scriptsize}
        \end{column}
        \begin{column}{0.4\textwidth}
            \begin{figure}[htp]
                \centering
                \includegraphics[height=5.5cm, keepaspectratio]{M1BlockSchema.png}
                \caption{\tiny{Архитектура <<М1>>}}
                \label{fig:M1BlockSchema}
            \end{figure}
        \end{column}
    \end{columns}
\end{frame}

% Слайд МЭСМ
\begin{frame}{История развития вычислительной техники}
    \begin{itemize}
        \item 1951 <<МЭСМ>> (Малая электронная счетная машина), Сергей Алексеевич Лебедев,
        Институт электротехники АН Украины
        \begin{itemize}
            \item электронные реле
            \item двоичная система, 16 разрядов
            \item память на триггерных ячейках с возможностью использования барабанов
            \item универсальное арифметическое устройство
            \item 50 операций в секунду
            \item ввод с перфорационных карт или посредством набора кодов
            на штекерном коммутаторе
            \item вывод~--- фотографирование или посредством электромеханического
            печатающего устройства
        \end{itemize}
    \end{itemize}
\end{frame}

% Слайд СТРЕЛА
\begin{frame}{История развития вычислительной техники}
    \begin{itemize}
        \item 1953 <<СТРЕЛА>>, Базилевский Юрий Яковлевич, СКБ-245, НИИ Счетмаш,
        завод САМ (Москва)
        \begin{itemize}
            \item лампы
            \item 2000 команд в секунду
            \item операции с плавающей точкой, 43 разряда
            \item память на электронно-лучевых трубках
            \item ПЗУ на полупроводниковых диодах емкостью 15 стандартных
            подпрограмм по 16 команд и 256 операндов
            \item ввод с устройства ввода перфокарт или с магнитной ленты
            \item вывод на магнитную ленту, перфоратор карт или на широкоформатное
            печатающее устройство
        \end{itemize}
        \item Первая серийная машина, для неё создана
        программа расчета термоядерного взрыва (М.Р.~Шура-Бура)
    \end{itemize}
\end{frame}

% Слайд БЭСМ-1
\begin{frame}{История развития вычислительной техники}
    \begin{itemize}
        \item 1953 <<БЭСМ-1>> (Большая электрическая счетная машина), С.А.~Лебедев
        \begin{itemize}
            \item лампы (около 5000)
            \item двоичная с плавающей запятой, 32 разряда
            \item ОЗУ на ферритовых сердечниках емкостью 1024 числа
            \item ввод с фотосчитывающего устройства на перфоленте
            \item вывод на электромеханическое печатающее устройство
            \item 8000-10000 операций в секунду
            \item \begin{spacing}{0.8}наличие ДЗУ:  не изменяется во время работы машины.
            В первых двадцати ячейках ДЗУ можно набирать вручную любые числа и
            команды при помощи штекеров либо вставлять заранее пробитые перфокарты.
            Две ячейки ДЗУ выведены на пульт управления, и их содержимое задаётся
            включением соответствующих тумблеров. В остальных ячейках ДЗУ постоянно
            хранятся некоторые наиболее часто встречающиеся константы и подпрограммы.
            \end{spacing}
        \end{itemize}
    \end{itemize}
\end{frame}

% Слайд БЭСМ-2, СЕТУНЬ
\begin{frame}{История развития вычислительной техники}
    \begin{itemize}
        \item 1956 С.А.~Лебедев, идея многопроцессорной машины
        \item 1955 первый советский транзистор
        \item 1956 <<БЭСМ-2>>~--- серийный аналог <<БЭСМ-1>>
        \item 1958 <<Сетунь>>~--- троичная машина (Николай Петрович Брусенцов, МГУ)
        \begin{itemize}
            \item \begin{scriptsize}\begin{spacing}{1.2}В отличие от двоичного кода с цифрами <<0>>, <<1>> у которого нет
            возможности непосредственного представления отрицательных чисел, троичный
            код с цифрами <<-1>>, <<0>>, <<1>> обеспечивает оптимальное построение
            арифметики чисел со знаком. При этом, не только нет нужды в искусственных
            и несовершенных дополнительном, прямом либо обратном кодах чисел,
            но арифметика обретает ряд значительных преимуществ: единообразие кода чисел,
            варьируемая длина операндов, единственность операции сдвига, трехзначность
            функции знака числа, оптимальное округление чисел простым отсечением младших
            разрядов, взаимокомпенсируемость погрешностей округления в процессе вычисления.
            \end{spacing}
            \end{scriptsize}
        \end{itemize}
    \end{itemize}
\end{frame}

% Слайд КИЕВ
\begin{frame}{История развития вычислительной техники}
    \begin{itemize}
        \item 1958-1959 <<КИЕВ>>, Виктор Михайлович Глушков, Институт кибернетики АН УССР
        \begin{itemize}
            \item лампы
            \item двоичная, 40 разрядов
            \item ОЗУ на ферритовых сердечниках, 1023 слова
            \item ПЗУ на трансформаторах, 512 слов
            \item ввод с перфоленты, линий связи, АЦП
            \item вывод на перфоратор или печатающее устройство
            \item предназначена для решения широкого круга математических задач и
            для научно-экспериментальных работ, связанных с исследованиями алгоритмов
            управления производственными процессами
        \end{itemize}
    \end{itemize}
\end{frame}

% Слайд M-20
\begin{frame}{История развития вычислительной техники}
    \begin{itemize}
        \item 1955-1959 <<М-20>> С.А.~Лебедев, М.Р.~Шура-Бура и П.П.~Головистиков
        \begin{itemize}
            \item усовершенствованная элементная база~--- максимальный отказ от ламп
            в пользу диодов
            \item двоичная, 45 разрядов
            \item ОЗУ на ферритовых сердечниках, 4096 слова
            \item внешняя память на магнитных барабанах и лентах
            \item 20000 операций в секунду
            \item индексная арифметики, позволяющая во многих случаях избавиться от
            переменных команд
            \item новые логические операции в процессоре
            \item системы команд с автоматической модификацией адреса
            \item совмещение работы АУ с выборкой команд из памяти
            \item совмещение вывода информации на печать с работой процессора.
        \end{itemize}
    \end{itemize}
\end{frame}

% Слайд ДНЕПР, МИР-2
\begin{frame}{История развития вычислительной техники}
    \begin{itemize}
        \item 1960 <<ДНЕПР>> (В.М.~Глушков, Б.Н.~Малиновский)
        \begin{itemize}
            \item первая машина на полупроводниках
            \item среднее быстродействие~--- 10000 операций в секунду
            \item система прерываний
            \item ОЗУ 512 26 разрядных слов + 3x512 дополнительных блока
            \item ~<<Днепр>>~--- управляющая машина широкого назначения
        \end{itemize}

        \item 1969 <<МИР-2>> (В.М.~Глушков) машина для инженерных расчетов
        \begin{itemize}
            \item ОЗУ 4096 12 разрядных слов
            \item два стола: пишущая машинка и отладочный пульт
            \item в дальнейшем дисплей со световым пером
            \item специальный язык высокого уровня <<АНАЛИТИК>>
        \end{itemize}
    \end{itemize}
\end{frame}

% Слайд АИСТ
\begin{frame}{История развития вычислительной техники}
    \begin{itemize}
        \item 1966-1974 проект <<АИСТ-0>>, <<АИСТ-1>> (Автоматическая информационная
        станция) Андрей Петрович Ершов
        \begin{itemize}
            \item режим разделения времени
            \item совмещение режима пакетной обработки с
            диалоговым режимом множества удаленных пользователей
            \item параллельное функционирование всех входящих в систему аппаратных устройств
            \item рабочие процессоры: <<М-220>>, управляющий процессор <<Минск-22>>
            \item ~<<АИСТ-1>>~--- доступ к <<БЭСМ-6>> с различных терминалов
        \end{itemize}
    \end{itemize}
\end{frame}

% Слайд БЭСМ-6
\begin{frame}{История развития вычислительной техники}
    \begin{itemize}
        \item 1967 <<БЭСМ-6>> С.А.~Лебедев, Институт точной механики и вычислительной
        техники (ИТМ и ВТ) АН СССР
        \begin{itemize}
            \item транзисторы
            \item ОЗУ на ферритовых сердечниках 32Кб-128Кб 50 разрядных слов
            \item до 1~млн. операций в секунду
            \item впервые (в СССР и независимо от IBM) применен принцип
            совмещения выполнения команд (водопровод, конвейер)
            \item ассоциативная память на быстрых регистрах (типа cache) позволяла
            автоматически сохранять в ней наиболее часто используемые операнды и тем самым сократить число обращений к оперативной памяти
            \item \begin{spacing}{0.9}механизмы прерывания, защиты памяти, преобразования виртуальных
            адресов в физические и привилегированный режим работы для ОС
            позволили использовать <<БЭСМ-6>> в мультипрограммном режиме и режиме разделения времени
            \end{spacing}
        \end{itemize}
    \end{itemize}
\end{frame}

\begin{frame}{История развития вычислительной техники}
    \begin{itemize}
        \item 1979 МВК <<ЭЛЬБРУС-1>> Институт точной механики и вычислительной техники
        \begin{itemize}
            \item микросхемы
            \item до 15~млн. операций в секунду, ОЗУ 1~млн. слов
            \item 1, 2, 4 и 10 процессорные
            \item первая коммерческая суперскалярная ЭВМ
        \end{itemize}
        \item 1985 <<ЭЛЬБРУС-2>>
        \begin{itemize}
            \item новая элементная база
            \item до 125~млн операций в секунду, 144 Мб ОЗУ
        \end{itemize}
        \item 2007 «ЭЛЬБРУС-3М»
        \begin{itemize}
            \item VLIW архитектура
        \end{itemize}
        \item 2018 «Эльбрус-8СВ»
        \begin{itemize}
            \begin{scriptsize}
            \item многоядерный процессор
            \item 64-разрядная VLIW архитектура E2K (<<ЭЛЬБРУС 2000>>) 5-го поколения
            \end{scriptsize}
        \end{itemize}
    \end{itemize}
\end{frame}


\end{document}