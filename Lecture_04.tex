\documentclass[aspectratio=169,14pt]{beamer}

\usepackage[utf8]{inputenc}
\usepackage[main=russian,english]{babel}
\usepackage[T1]{fontenc}
\usepackage[labelformat=empty]{caption}
\usepackage{multimedia}
\usepackage{xcolor}
% \usepackage{hyperref}
\usepackage{setspace}
% \usepackage{verbatim}
% \usepackage{multirow}

\usetheme{Singapore}

\definecolor{urlcolor}{HTML}{799B03} % цвет гиперссылок
\hypersetup{urlcolor=urlcolor, colorlinks=true}

\graphicspath{{../Images/}}

\title{Центральные процессоры - 80168 - Pentium III}

\begin{document}

% Слайд Intel 80186
\begin{frame}{Intel 80186}
    \begin{itemize}
        \item 1982 год~--- процессор 80186
        \pause
        \item 6-25 МГц, 3мкм, 55000 транзисторов
        \pause
    \end{itemize}
    \begin{columns}[T,onlytextwidth]
        \begin{column}{0.5\textwidth}
            \begin{figure}[htp]
                \centering
                \includegraphics[height=5cm, keepaspectratio]{8086ComponentFamily.png}
                \captionsetup{skip=-5pt}
                \caption{\tiny{Микросхемы поддержки Intel 8088}}
            \end{figure}
        \end{column}
        \begin{column}{0.5\textwidth}
            \begin{figure}[htp]
                \centering
                \includegraphics[width=4.9cm, keepaspectratio]{80188BlockArchitecture.png}
                \includegraphics[height=2cm, keepaspectratio]{80188Foto.jpg}
                % \captionsetup{skip=-5pt}
                % \caption{\tiny{Микросхемы поддержки Intel 8088}}
            \end{figure}
        \end{column}
    \end{columns}
\end{frame}

% Слайд Intel 80286
\begin{frame}{Intel 80286}
    \begin{itemize}
        \begin{spacing}{0.8}
            \item 1982 год~--- второе поколение 16-разрядных процессоров
            \item 16-разрядная шина данных, 24-разрядная шина адреса, демультиплексированы
            \item Частота 6-20 МГц, технология 1.5 мкм
            \item Увеличено количество регистров (+11)
            \item 16 новых команд
        \end{spacing}
        \pause
    \end{itemize}
    \begin{columns}[T,onlytextwidth]
        \begin{column}[c]{0.5\textwidth}
        %     \begin{figure}[htp]
        %         \centering
                \includegraphics[height=3cm, keepaspectratio]{80286Face.jpg}
            % \end{figure}
        \end{column}
        \begin{column}[c]{0.5\textwidth}
            % \begin{figure}[htp]
            %     \centering
                \includegraphics[width=3cm, keepaspectratio]{80286UnFace.jpg}
            % \end{figure}
        \end{column}
    \end{columns}
\end{frame}

% Слайд Intel 80286 - внутренняя архитектура
\begin{frame}{Intel 80286~--- внутренняя архитектура}
    \begin{itemize}
        \begin{footnotesize}
        \begin{spacing}{0.8}
            \item \textbf{Блок шины}, BU (Bus Unit)~--- операции с шиной, генерация адресов
            и данных на выводах, а также управляющих сигналов для доступа к внешней
            памяти и устройствам ввода/вывода. Выполняет предварительное извлечение
            инструкций из памяти в очередь
            \item \textbf{Блок инструкций}, IU (Instruction Unit)~--- извлекает инструкции
            из очереди, декодирует их и помещает в другую очередь (глубиной 3)
            декодированных инструкций для EU
        \end{spacing}
        \end{footnotesize}
        \pause
    \end{itemize}
    \begin{columns}[T,onlytextwidth]
        \begin{column}[c]{0.4\textwidth}
            \begin{itemize}
                \begin{footnotesize}
                \begin{spacing}{0.8}
                    \item \textbf{Блок исполнения}, EU (Execution Unit)~--- исполняет инструкции
                    \item \textbf{Блок адресов}, AU (Address Unit)~--- предоставляет управление памятью
                    и защиту для CPU, транслируя логические адреса в физические для BU
                \end{spacing}
                \end{footnotesize}
            \end{itemize}
        \end{column}
        \begin{column}[c]{0.6\textwidth}
            \includegraphics[height=4cm, keepaspectratio]{80286InternalBlockDiagramm.png}
        \end{column}
    \end{columns}
\end{frame}

% Слайд Intel 80286 - конвейер
\begin{frame}{Intel 80286~--- конвейер}
    Наличие отдельных блоков позволяет организовать конвейер, повышающий производительность процессора

    \includegraphics[width=10cm, keepaspectratio]{80286Pipeline.png}
\end{frame}

% Слайд Intel 80286 - режимы работы
\begin{frame}{Intel 80286~--- режимы работы}
    \begin{itemize}
        \item 8086 Real-Address Mode
        \begin{itemize}
            \item Адресация до 1М с использованием 20 разрядов шины адреса
        \end{itemize}
        \item Protected Virtual-Address Mode
        \begin{itemize}
            \item Защищенный режим, позволяющий адресовать до 16Мб
             ($2^{24}$) физической памяти (до 1Гб ($2^{30}$) виртуальной памяти)
        \end{itemize}
    \end{itemize}
\end{frame}

% Слайд Intel 80286 - организация памяти
\begin{frame}{Intel 80286~--- организация памяти}
    \begin{itemize}
        \item Виртуальное адресное пространство~--- организация памяти с точки
        зрения приложения
        \item В реальном режиме приложение напрямую <<видит>> всю память,
        виртуальное адресное пространство совпадает с физическим
        \item В защищенном режиме приложения не имеют прямого доступа к памяти.
        Для них память представляется как большое виртуальное адресное
        пространство размером 1Гб
        \item Виртуальная адресное пространство представляется как набор
        до 16К линейных подпространств с определенным размером (сегмент).
        Размер сегмента может меняться от одного байта до 64 килобайт
    \end{itemize}
\end{frame}

% Слайд Intel 80286 - организация памяти 2
\begin{frame}{Intel 80286~--- организация памяти}
    \begin{itemize}
        \begin{spacing}{0.9}
        \begin{footnotesize}
            \item В реальном режиме содержимое сегментных регистров
        интерпретируется как физический адрес начала соответствующего сегмента
        \item В защищенном режиме содержимое регистра сегментов
        представляет собой <<селектор>> сегмента
        \item Селектор~--- это индекс (14 бит), который указывает на адрес
        начала сегмента (всего может быть $2^{14}$ сегментов) в таблице
        дескрипторов и 2 бита, которые описывают свойства сегмента (RPL,
        Requested Privilege Level).
        \item Один бит индекса селектора определяет глобальное адресное
        пространство или локальное адресное пространство
        \begin{itemize}
            \begin{footnotesize}
            \item Глобальное адресное пространство используется для
            общеупотребительных процедур, включая операционные системы,
            библиотеки, которые доступны всем приложениям. GDT (global descriptor
            table)~--- таблица глобальных дескрипторов для глобальных сегментов
            \item Локальное адресное пространство принадлежит только
            одному приложению. LDT (local description table)~--- таблиц локальных
            дескрипторов для локальных сегментов (может быть несколько
            таких таблиц)
            \end{footnotesize}
        \end{itemize}
        \item Тринадцать бит индекса определяют сегмент в таблице дескрипторов
        \end{footnotesize}
        \end{spacing}
    \end{itemize}
\end{frame}

% Слайд Intel 80286 - таблица дескрипторов сегментов
\begin{frame}{Intel 80286~--- таблица дескрипторов сегментов}
    \begin{itemize}
        \item Каждая запись в таблице дескрипторов состоит из 8 байт и описывает
        соответствующий сегмент
        \item Сегментные дескрипторы, описывают обычные сегменты
        (данные, код, стек) ($S=1$)
        \item Специальный или системный дескриптор ($S=0$, например,
        описывает сегмент, где хранится LDT и т.п.)
    \end{itemize}

    \includegraphics[width=7cm, keepaspectratio]{80286SegmentDescription.png}
\end{frame}

% Слайд Intel 80286 - дескриптор сегмента
\begin{frame}{Intel 80286~--- дескриптор сегмента}
    \begin{itemize}
        \item Limit~--- размер сегмента
        \item Base~--- адрес начала сегмента
        \item Type~--- тип сегмента и информация о доступе
        \item DPL~--- уровень привилегий дескриптора
    \end{itemize}

    \includegraphics[width=6cm, keepaspectratio]{80286SegmentDescriptionDetailCodeSegment.png}
    \includegraphics[width=6cm, keepaspectratio]{80286SegmentDescriptionDetailDataSegment.png}
\end{frame}

% Слайд Intel 80286 - трансляция адресов
\begin{frame}{Intel 80286~--- трансляция адресов}
    \begin{columns}[T,onlytextwidth]
        \begin{column}[c]{0.5\textwidth}
            \begin{itemize}
                \begin{footnotesize}
                    \item Для хранения базы сегментов GDT и LDT используются
                    специальные регистры GDTR (40-бит, Base (24) и Limit (16)) и LDTR
                    \item Значение регистра LDTR (56 бит: Base, Limit, Visible Selector,
                    кэшируется) перезаписывается при переключении задачи
                    \item Таблица векторов прерываний также хранится в отдельном
                    сегменте, адрес которой содержится в регистре IDTR
                    (40-бит, Limit и Base)
                    \end{footnotesize}
            \end{itemize}
        \end{column}
        \begin{column}[c]{0.5\textwidth}
            \includegraphics[width=7cm, keepaspectratio]{80286AddressTranslation.png}
        \end{column}
    \end{columns}
\end{frame}

% Слайд Intel 80286 - механизмы защиты
\begin{frame}{Intel 80286~--- механизмы защиты памяти}
    \begin{itemize}
        \begin{footnotesize}
        \item Аспекты защиты
        \begin{itemize}
            \begin{footnotesize}
            \item Изоляция системного кода (например, выполнение
            операций I/O) от пользовательского
            \item Изоляция задач пользователей друг от друга
            \item Проверка данных
            \end{footnotesize}
        \end{itemize}
        \item Механизмы защиты основываются на понятии <<иерархия доверия>>
        \item Введены четыре уровня привилегий: уровень 0~--- самый
        доверенный, уровень 4~--- наименее доверенный
        \item Все сегменты данных и сегменты кода отнесены к одному
        из четырех уровней (DPL)
        \item Приложение (задача, task) выполняется в одном из
        четырех уровней, при этом не может обращаться к данным из
        более высокого уровня доверия или вызывать процедуры
        из более низкого уровня привилегий
        \item У каждой задачи есть текущий уровень
        привилегий (СPL, 2 бита), определяется по селектору сегмента кода
        \item При загрузке сегмента (дескриптора) сравнивается его DPL и CPL задачи
        \end{footnotesize}
    \end{itemize}
\end{frame}

% Слайд Intel 80386
\begin{frame}{Intel 80386}
    \begin{itemize}
        \begin{spacing}{0.8}
            \item 1985 год~--- 32-разрядный процессор (DX)
            \item 32-разрядные шины адреса и данных, до 4Гб ($2^{32}$) физической
            памяти, до 64Тб ($2^{46}$) виртуальной памяти
            \item Частота 12-40 МГц, технология 1.5-1.0 мкм, 275000 транзисторов
            \item Почти все регистры 32-битные (за исключением сегментных).
            Увеличено количество регистров (+11)
            \item Набор инструкций расширен в основном за счет появления
            32-битных вариантов существующих команд
        \end{spacing}
        \pause
    \end{itemize}
    \begin{columns}[T,onlytextwidth]
        \begin{column}[c]{0.5\textwidth}
        %     \begin{figure}[htp]
        %         \centering
                \includegraphics[height=2cm, keepaspectratio]{80386Face.jpg}
            % \end{figure}
        \end{column}
        \begin{column}[c]{0.5\textwidth}
            % \begin{figure}[htp]
            %     \centering
                \includegraphics[width=2cm, keepaspectratio]{80386UnFace.jpg}
            % \end{figure}
        \end{column}
    \end{columns}
\end{frame}

% Слайд Intel 80386 - внутренняя архитектура
\begin{frame}{Intel 80386: внутренняя архитектура}
    \includegraphics[width=13cm, keepaspectratio]{80386FunctionalUnits.png}
\end{frame}

% Слайд Intel 80386 - внутренняя архитектура
\begin{frame}{Intel 80386: внутренняя архитектура}
    \begin{itemize}
        \begin{footnotesize}
        \begin{spacing}{0.8}
        \item \textbf{Bus Interface Unit}~--- интерфейс между ЦПУ и <<внешним>>
        миром, извлекает инструкции, обеспечивает пересылку данных,
        управляет взаимодействием с внешними контроллерами системной шины
        \item \textbf{Code Prefetch Unit}~--- посылает команды BIU для извлечения
        возможных следующих инструкций программы. 16 байтовая очередь
        (Code Queue)
        \item \textbf{Instruction Decode Unit}~--- извлекает инструкции из очереди
        и транслирует их в микрокод. Декодированные инструкции хранятся
        в очереди (Instruction Queue)
        \item \textbf{Execution Unit}~--- исполняет инструкции из очереди
        Instruction Queue, взаимодействуя с остальными блоками
            \begin{itemize}
                \begin{tiny}
                \begin{spacing}{0.8}
                \item \textbf{Control Unit}~--- содержит микрокод и быстродействующие
                параллельные схемы для выполнения операций умножения, деления,
                вычисления адресов
                \item \textbf{Data Unit}~--- содержит АЛУ, восемь 32-битных
                регистров общего назначения,  64-битное устройство сдвига;
                выполняет команды по запросу от Control Unit
                \item \textbf{Protection Test Unit}~--- выполняет проверку
                доступа к сегментам с помощью микрокода
                \end{spacing}
                \end{tiny}
            \end{itemize}
        \item \textbf{Segmentation Unit}~--- преобразует логические адреса в
        линейное адресное пространство по запросу Execution Unit.
        Полученные адреса передаются в блок Paging Unit. Содержит кэш
        таблицы дескрипторов сегментов. Также выполняет проверку доступа
        к сегментам
        \item \textbf{Paging Unit}~--- транслирует адреса из линейного адресного
        пространства в физические адреса, после этого передает их в
        блок Bus Inferface Unit
        \end{spacing}
        \end{footnotesize}
    \end{itemize}
\end{frame}

% Слайд Intel 80386 - режимы работы
\begin{frame}{Intel 80386: режимы работы}
    \begin{itemize}
        \item \textbf{Real-Address Mode (Real Mode)}~--- реальный режим,
        соответствует 8086 с несколькими новыми инструкциями
        \item \textbf{Protected Mode}~--- защищенный режим, доступны
        все инструкции и возможности процессора
        \item \textbf{Virtual 8086 Mode (V88 mode)}~--- динамический
        режим, в котором происходит переключение между реальным режимом
        и защищенным режимом
    \end{itemize}
\end{frame}

% Слайд Intel 80386 - организация памяти
\begin{frame}{Intel 80386: организация памяти}
    \begin{columns}[T,onlytextwidth]
        \begin{column}[c]{0.7\textwidth}
            \begin{itemize}
            \begin{footnotesize}
            \begin{spacing}{0.8}
            \item Адреса в программах не зависят от физического адресного пространства
            \item ~«Плоская» (flat) модель – память для приложения представлена как единый массив размером 4Гб
            \item Сегментная модель – память рассматривается как набор 16383 сегментов, каждый их которых имеет длину до 4Гб; адрес состоит из двух частей:
            \begin{itemize}
                \begin{footnotesize}
                \item селектор сегмента (16 бит), который определяет соответствующий сегмент
                \item смещение (32 бита)
                 \end{footnotesize}
            \end{itemize}
            \item  Преобразование адресов из логического адресного пространства в физическое происходит в два этапа
            \begin{itemize}
                \begin{footnotesize}
                \item Преобразование сегмента: логический адрес преобразуется в линейный
                \item Преобразование страницы: линейный адрес преобразуется в физический
                \end{footnotesize}
            \end{itemize}
            \end{spacing}
            \end{footnotesize}
            \end{itemize}
        \end{column}
        \begin{column}[c]{0.3\textwidth}
            \includegraphics[width=5cm, keepaspectratio]{80386AddressTranslationOverview.png}
        \end{column}
    \end{columns}
\end{frame}

% Слайд Intel 80386 - преобразование сегментов
\begin{frame}{Intel 80386: дескрипторы сегментов}
    \begin{itemize}
        \item BASE~--- определяет расположение начала сегмента в 4Гб линейном
        пространстве
        \item LIMIT~--- размер сегмента, измеряется в единицах по
        1 байту или по 4Кб (бит гранулярности)
    \end{itemize}

    \includegraphics[height=4cm, keepaspectratio]{80386SegmentTranslation.png}
    \includegraphics[height=4.3cm, keepaspectratio]{80386SegmentDescriptor.png}
\end{frame}

% Слайд Intel 80386 - преобразование страниц
\begin{frame}{Intel 80386: преобразование страниц}
    \begin{columns}[T,onlytextwidth]
        \begin{column}[c]{0.7\textwidth}
            \begin{itemize}
            \begin{small}
            \begin{spacing}{0.8}
            \item Таблица страниц~--- массив 32-битных идентификаторов
            (спецификаторов) страниц, занимает 4Кб, определяет 1К страниц
            \item Каталог страниц~--- массив указателей на таблицу страниц,
            1К указателей
            \item Всего 1М страниц по 4К~--- полное физическое адресное
            пространство ($2^{20}*2^{12} = 2^{32}$)
            \item Физический адрес текущего каталога страниц хранится в
            специальном регистре CR3 (PDBR, Page Directory Base Register)
            \item Может использоваться один каталог страниц для всех задач
            или отдельный каталог для каждой задачи
            \end{spacing}
            \end{small}
            \end{itemize}
        \end{column}
        \begin{column}[c]{0.3\textwidth}
            \includegraphics[height=3.4cm, keepaspectratio]{80386PageTranslation.png}
        \end{column}
    \end{columns}
\end{frame}

% Слайд Intel 80386 - преобразование страниц
\begin{frame}{Intel 80386: дескриптор страницы}
    \begin{itemize}
        \begin{spacing}{0.8}
        \item Frame~--- страница в оперативной памяти
        \item ACCESSED и DIRTY могут использоваться операционной системой
        для алгоритма замещения страниц
        \item READ/WRITE и USER/SUPERVISOR используются для защиты
        на уровне страниц
        \item Только бит P (Present) проверяется аппаратно
        \item Часто используемые части таблицы страниц хранятся
        в кэше процессора
        \end{spacing}
    \end{itemize}
    \includegraphics[height=3cm, keepaspectratio]{80386PageTableEntry.png}
\end{frame}

% Слайд Intel 80386 - мультизадачность
\begin{frame}{Intel 80386: мультизадачность}
    \begin{columns}[T,onlytextwidth]
        \begin{column}[c]{0.7\textwidth}
            \begin{itemize}
            \begin{small}
            \begin{spacing}{1}
            \item Для повышения производительности системы в целом
            \item В процессоре есть специальные структуры для поддержки многозадачности
            \begin{itemize}
                \item Task State Segment: содержимое регистров, селектор TSS предыдущей задачи, селектор LTB, карта I/O
                \item TSS Descriptor: спецификатор TSS
                \item Task Register: в нем хранится селектор на TSSD
                \item Task Gate Descriptor: предоставляет непрямой, защищенный доступ к TSS
            \end{itemize}
            \end{spacing}
            \end{small}
            \end{itemize}
        \end{column}
        \begin{column}[c]{0.3\textwidth}
            \includegraphics[height=5.4cm, keepaspectratio]{80386TaskRegister.png}
        \end{column}
    \end{columns}
\end{frame}

% Слайд Intel 80486
\begin{frame}{Intel 80486}
    \begin{itemize}
        \begin{spacing}{0.8}
            \item 1989 год~--- 32-разрядный процессор (усовершенствованная версия 80386)
            \item Встроенный математический сопроцессор
            \item Частота 16-100 МГц, технология 600-1000 нм, около 1,2М-1,6М транзисторов
            \item Встроенный кэш первого уровня 8-16Кб
            \item Гибридное CISC-RISC ядро
        \end{spacing}
        \pause
    \end{itemize}

    \includegraphics[height=2.6cm, keepaspectratio]{80486Face.jpg}
    \includegraphics[height=2.6cm, keepaspectratio]{80486UnFace.jpg}
\end{frame}

% Слайд Intel 80486 - внутренняя архитектура
\begin{frame}{Intel 80486: внутренняя архитектура}
    \includegraphics[height=7cm, keepaspectratio]{80486FunctionalUnits.png}
\end{frame}

% Слайд Intel 80486 - конвейер
\begin{frame}{Intel 80486: конвейер}
    \begin{itemize}
        \begin{spacing}{0.8}
        \item Fetch~--- извлечение инструкции
        \item Decode~1~--- декодирование инструкции
        \item Decode~2~--- вычисление <<сложных>> режимов адресации
        \item Execution~-- исполнение, доступ к кешу, изменение регистров
        \item Register Write-back~--- изменение регистра флагов, запись других регистров
    \end{spacing}
    \end{itemize}

    \includegraphics[height=3.4cm, keepaspectratio]{80486Pipeline.png}
\end{frame}

% Слайд Intel Pentium
\begin{frame}{Intel Pentium}
    \begin{itemize}
        \begin{spacing}{0.8}
            \item 1993 год~--- 32-разрядный процессор
            \item Частота 60-233, технология 800-280 нм, 3.1-4.5 М транзисторов
            \item 64-х разрядная шина данных
            \item Суперскалярная архитектура
        \end{spacing}
        \pause
    \end{itemize}

    \includegraphics[height=3.4cm, keepaspectratio]{PentiumFace.jpg}
    \includegraphics[height=3.4cm, keepaspectratio]{PentiumUnFace.jpg}
\end{frame}

% Слайд Intel Pentium - внутренняя архитектура
\begin{frame}{Intel Pentium: внутренняя архитектура}
    \includegraphics[height=7cm, keepaspectratio]{PentiumFunctionalUnits.png}
\end{frame}

% Слайд Intel Pentium - конвейер
\begin{frame}{Intel Pentium: конвейер}
    \begin{itemize}
        \item Как и в 80486, конвейер состоит из 5 стадий
        \begin{itemize}
            \item PF (Prefetch)
            \item D1 (Instruction Decode)
            \item D2 (Address Generate)
            \item EX (Execute – ALU и доступ к кэшу)
            \item WR (WriteBack)
        \end{itemize}
        \item Два конвейера: \textbf{u} (выполнение любых инструкций)
        и \textbf{v} (выполнение <<простых>> целочисленных инструкций
        и инструкции обмена значениями регистров FXCH для чисел с
        плавающей точкой)~--- \textbf{суперскалярная архитектура}
        \item Одновременно на конвейеры \textbf{u} и \textbf{v} могут быть помещены только парные инструкции
        \item Конвейер сопроцессора
    \end{itemize}
\end{frame}

% Слайд Intel Pentium - конвейер
\begin{frame}{Intel Pentium: конвейер}
    \begin{itemize}
        \begin{spacing}{0.8}
        \item Конвейер сопроцессора: 8 стадий
        \begin{itemize}
            \begin{footnotesize}
            \item PF, D1, D2
            \item EX - чтение из памяти и регистров, преобразование FP данных для записи в память, запись в память)
            \item X1 – первая стадия выполнения инструкции
            \item X2 – вторая стадия выполнения инструкции
            \item WF – округление и запись результата в регистр
            \item ER – отчет об ошибке/изменение статусного слова
            \end{footnotesize}
        \end{itemize}
        \item Одновременно на конвейер могут помещаться
        только <<совместимые>> инструкции
        \end{spacing}
    \end{itemize}

    \includegraphics[height=3cm, keepaspectratio]{PentiumPipelineInstructions.png}
    \includegraphics[height=2.6cm, keepaspectratio]{PentiumPipelines.png}
\end{frame}


% Слайд Intel Pentium - предсказание ветвлений
\begin{frame}{Intel Pentium: конвейер}
    \begin{itemize}
        \begin{spacing}{0.8}
        \item На стадии PF реализованы два буфера, в которые помещаются инструкции при извлечении из памяти
        \begin{itemize}
            \item Линейная последовательность
            \item Возможный переход
        \end{itemize}
        \item Для определения возможного перехода используется
        Branch Target Buffer
        \item Branch Prediction Buffer (256 элементов)~--- запоминается
        статистика совершения перехода в инструкции
        \item Каждый элемент содержит адрес инструкции (source), бит
        валидности, биты истории (2) и адрес перехода (target)
        \item Переход считается предсказанным, если запись в BTB
        найдена, бит валидности установлен и биты истории $> 1$
        \item Биты истории увеличиваются на 1 при каждом успешном
        переходе, уменьшаются при каждом невыполненном переходе
        \end{spacing}
    \end{itemize}
\end{frame}

% Слайд проблемы конвейеров
\begin{frame}{Проблемы конвейера и суперскаляров}
    \begin{itemize}
        \item \textbf{Структурные проблемы}: инструкциям для
        выполнения нужны ресурсы внутри процессора, но эти ресурсы
        могут быть заняты
        \item \textbf{Проблемы данных}: инструкции зависят
        от данных, которые будут вычислены более ранними инструкциями
        \item \textbf{Проблемы управления}: будет или нет осуществлено
        выполнение инструкции зависит от решения, принятого ранней
        инструкцией (переходы)
    \end{itemize}
    Возникновение той или иной проблемы приводит в лучшем случае,
    к приостановке конвейера/конвейеров, в худшем~--- к его/их полной
    очистке
\end{frame}

\end{document}
