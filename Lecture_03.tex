\documentclass[aspectratio=169,14pt]{beamer}

\usepackage[utf8]{inputenc}
\usepackage[main=russian,english]{babel}
\usepackage[T1]{fontenc}
\usepackage[labelformat=empty]{caption}
\usepackage{multimedia}
\usepackage{xcolor}
\usepackage{hyperref}
\usepackage{setspace}
\usepackage{verbatim}
\usepackage{multirow}

\usetheme{Singapore}

\definecolor{urlcolor}{HTML}{799B03} % цвет гиперссылок
\hypersetup{urlcolor=urlcolor, colorlinks=true}

\graphicspath{{../Images/}}

\title{Центральные процессоры}

\begin{document}

% Слайд центральные процессоры
\begin{frame}{Центральные процессоры}
    \begin{itemize}
        \item Из простейших электронных элементов можно сконструировать АЛУ и
        память для хранения промежуточных результатов выполнения арифметических
        и логических операций
        \pause
        \item Микропроцессор~--- устройство, отвечающее за выполнение арифметических,
        логических операций и операций управления, записанных в машинном коде
        \pause
        \item Реализуется в виде одной или нескольких микросхем
    \end{itemize}
\end{frame}

% Слайд коды операций, регистры
\begin{frame}{Коды операций (инструкций). Регистры}
    \begin{itemize}
        \begin{spacing}{0.8}
        \item Простейшее АЛУ: операция для выполнения задается набором
        входных сигналов F0 и F1
        \pause
        \item Каждой операции можно поставить в соответствие некоторое число: получаем
        набор операций (команд или инструкций) для данного АЛУ (процессора)
        \pause
        \item Для выполнения операций АЛУ должно получить доступ к операндам
        (иметь электрические контакты с устройством хранения). Такие устройства
        называются регистрами. Набор регистров называется файлом регистров.
        \pause
        \item Для удобства написания программы регистры часто
        именуются (0,1,2\ldots; R1-R31; AX, BX\ldots)
        \pause
        \item Названия регистров можно использовать в качестве
        операндов в операциях.
        \end{spacing}
    \end{itemize}
\end{frame}

% Слайд машинный код, ассемблер
\begin{frame}[containsverbatim]{Машинный код, ассемблер}
    \begin{itemize}
        \item \footnotesize{Машинный код~--- последовательность команд в виде, готовом для исполнения на процессоре}
        \begin{table}
            \centering
            \begin{tiny}
            \begin{tabular}{ccc}
                \textbf{Адрес в памяти} & \textbf{Содержимое памяти} & \textbf{Комментарий} \\ \hline
                00000 & 11100101 & cчитываем слово в регистр AX\ldots \\
                00001 & 00000101 & \ldotsиз порта номер 5 \\
                00002 & 01000000 & увеличиваем на 1 регистр AX \\
                00003 & 11100111 & записываем слово из регистра AX\ldots \\
                00004 & 00000010 & \ldotsв порт номер 2 \\
                00005 & 11101011 & повторяем действия с помощью перехода\ldots \\
                00006 & 11111001 & \ldotsназад на 7 байт \\
            \end{tabular}
            \end{tiny}
        \end{table}
        \item \footnotesize{Ассемблер~-- язык, в котором машинные инструкции представляются символами (символическими именами)}
        \begin{columns}[T,onlytextwidth]
            \begin{column}{0.7\textwidth}
                \begin{tiny}
                \begin{verbatim}
CYCLE:
IN      AX, 5   ;считываем слова из порта номер 5 в регистр AX
INC     AX      ;увеличиваем на 1 регистр AX
OUT     2, AX   ;записываем слово из регистра AX в порт номер 2
JMP     CYCLE   ;повторяем действия
                \end{verbatim}
                \end{tiny}
            \end{column}
            \begin{column}{0.3\textwidth}
                \begin{tabular}{|c|c|}
                    \hline
                    \textbf{Инструкция} & \textbf{OpCode} \\ \hline
                    IN & \verb 1110010- \\ \hline
                    INC & \verb 01000--- \\ \hline
                    OUT & \verb 1110011- \\ \hline
                    JMP & \verb 11101011 \\ \hline
                \end{tabular}
            \end{column}
        \end{columns}
    \end{itemize}
\end{frame}

% Слайд архитектура и микроархитектура
\begin{frame}{Архитектура (Instruction Set Architecture) и Микроархитектура}
    \begin{itemize}
        \item ISA
        \begin{itemize}
            \item Состояние, видимое программисту (память, регистры, \ldots)
            \item Набор инструкций (синтаксис) и семантика их выполнения
            \item Прерывания
            \item Ввод/вывод
        \end{itemize}
        \pause
        \item Микроархитектура (реализация ISA в <<железе>>)
        \begin{itemize}
            \item Количество и глубина конвейера
            \item Микрокод
            \item Кeш
            \item \ldots
        \end{itemize}
        \pause
        \item Без абстракций сложно перейти от физических устройств к приложениям
    \end{itemize}
\end{frame}

% Слайд архитектура и микроархитектура
\begin{frame}{Характеристики ISA}
    \begin{itemize}
        \begin{spacing}{0.8}
        \item Классы инструкций
        \begin{itemize}
            \begin{footnotesize}
            \item Арифметико-логические
            \item Контроль управления
            \item Передача данных
            \item Плавающая точка, мультимедиа, строки, \ldots
            \end{footnotesize}
        \end{itemize}
        \pause
        \item Адресация памяти (регистры, смещения, индексы, абсолютная, \ldots)
        \pause
        \item Кодирование инструкций
        \begin{itemize}
            \begin{footnotesize}
            \item Фиксированный размер (RISC: MIPS, PowerPC, SPARC, \ldots)
            \item Переменный размер (CISC: IBM360, x86, VAX, \ldots)
            \item Mostly Fixed or Compressed
            \item Very Long Instruction Word
            \end{footnotesize}
        \end{itemize}
        \pause
        \item Почему отличаются ISA?
        \begin{itemize}
            \begin{footnotesize}
            \item Влияние технологий (дорогие устройства хранения -> сжатие; Multicore/Manycore, \ldots)
            \item Влияние приложений (инструкции для приложений: DSP; новые улучшенные компиляторы, \ldots)
            \end{footnotesize}
        \end{itemize}
        \end{spacing}
    \end{itemize}
\end{frame}

% Слайд пример единая ISA - разная микроархитектура
\begin{frame}{Пример: единая ISA~--- различная микроархитектура}
    \begin{columns}[T,onlytextwidth]
        \begin{column}{0.5\textwidth}
            \begin{itemize}
                \begin{footnotesize}
                \begin{spacing}{0.8}
                \item X86 Instruction Set
                \item Quad Core
                \item 125W
                \item Decode 3 Instructions/Cycle/Core
                \item 64KB L1 I Cache, 64KB L1 D Cache
                \item 512KB L2 Cache
                \item Out-of-order
                \item 2.6GHz
                \end{spacing}
                \end{footnotesize}
                \begin{figure}[htp]
                    \centering
                    \includegraphics[height=2cm, keepaspectratio]{AMDPhenomX4.jpg}
                \end{figure}
            \end{itemize}
        \end{column}
        \begin{column}{0.5\textwidth}
            \begin{itemize}
                \begin{footnotesize}
                \begin{spacing}{0.8}
                \item X86 Instruction Set
                \item Single Core
                \item 2W
                \item Decode 2 Instructions/Cycle/Core
                \item 32KB L1 I Cache, 24KB L1 D Cache
                \item 512KB L2 Cache
                \item In-order
                \item 1.6GHz
                \end{spacing}
                \end{footnotesize}
                \begin{figure}[htp]
                    \centering
                    \includegraphics[height=2cm, keepaspectratio]{IntelAtom.jpg}
                \end{figure}
            \end{itemize}
        \end{column}
    \end{columns}
\end{frame}

% Слайд пример различная ISA - разная микроархитектура
\begin{frame}{Пример: различная ISA~--- различная микроархитектура}
    \begin{columns}[T,onlytextwidth]
        \begin{column}{0.5\textwidth}
            AMD Phenom X4
            \begin{itemize}
                \begin{footnotesize}
                \begin{spacing}{0.8}
                \item X86 Instruction Set
                \item Quad Core
                \item 125W
                \item Decode 3 Instructions/Cycle/Core
                \item 64KB L1 I Cache, 64KB L1 D Cache
                \item 512KB L2 Cache
                \item Out-of-order
                \item 2.6GHz
                \end{spacing}
                \end{footnotesize}
                \begin{figure}[htp]
                    \centering
                    \includegraphics[height=2cm, keepaspectratio]{AMDPhenomX4.jpg}
                \end{figure}
            \end{itemize}
        \end{column}
        \begin{column}{0.5\textwidth}
            IBM POWER7
            \begin{itemize}
                \begin{footnotesize}
                \begin{spacing}{0.8}
                \item Power Instruction Set
                \item Eight Core
                \item 200W
                \item Decode 6 Instructions/Cycle/Core
                \item 32KB L1 I Cache, 32KB L1 D Cache
                \item 256KB L2 Cache
                \item Out-of-order
                \item 4.25GHz
                \end{spacing}
                \end{footnotesize}
                \begin{figure}[htp]
                    \centering
                    \includegraphics[height=2cm, keepaspectratio]{IBMPower7.jpg}
                \end{figure}
            \end{itemize}
        \end{column}
    \end{columns}
\end{frame}

% Слайд Intel 8086/8088
\begin{frame}{Процессор Intel 8086/8088}
    \begin{spacing}{0.8}
    \begin{itemize}
        \item 1978 г. Intel 8086 (первое поколение 16 разрядных процессоров Intel)
        \item 1979 г. Intel 8088 (восьмиразрядная шина данных для того, чтобы
        использовать уже существующие микросхемы <<поддержки>>: доступ к памяти,
        контроллер прерываний, I/O, \ldots)
        \item Двадцатиразрядная шина адреса
        \item Мультиплексированная шина адреса/данных
        \item Частота 4-10 МГц, технология 3 мкм
        \item 29 000 транзисторов
    \end{itemize}
    \begin{figure}[htp]
        \centering
        \includegraphics[height=1.8cm, keepaspectratio]{8088Foto.jpg}
    \end{figure}
    \end{spacing}
\end{frame}


% 8088PinsMinMode

% Слайд Intel 8086/8088 - описание контактов
\begin{frame}{Intel 8088 - описание контактов}
    \begin{figure}[htp]
        \centering
        \includegraphics[height=6cm, keepaspectratio]{8088PinsMinMode.png}
        \captionsetup{skip=-5pt}
        \caption{\tiny{Контакты процессора Intel 8088}}
    \end{figure}
\end{frame}

% Слайд Intel 8086/8088 - архитектура
\begin{frame}{Intel 8086/8088: архитектура}
    \begin{columns}[T,onlytextwidth]
        \begin{column}{0.5\textwidth}
            \begin{itemize}
                \begin{spacing}{0.8}
                \item BIU (Bus Interface Unit)~--- \textbf{Блок интерфейса шины}: извлекает
                инструкции, считывает операнды и записывает результат
                \item EU (Execution Unit)~--- \textbf{Блок исполнения}: выполняет
                инструкции
                \item Организована очередь инструкций для ускорения доступа
                \end{spacing}
            \end{itemize}
        \end{column}
        \begin{column}{0.5\textwidth}
            \begin{figure}[htp]
                \centering
                \includegraphics[height=6cm, keepaspectratio]{8086InternalStructureEng.png}
                \captionsetup{skip=-5pt}
                \caption{\tiny{Общая архитектура Intel 8088}}
            \end{figure}
        \end{column}
    \end{columns}
\end{frame}

% Слайд Intel 8086/8088 - архитектура, детали
\begin{frame}{Intel 8086/8088: детали архитектуры}
    \begin{figure}[htp]
        \centering
        \includegraphics[height=6.6cm, keepaspectratio]{8086InternalStructureDetail.png}
        \captionsetup{skip=-5pt}
        \caption{\tiny{Детали архитектуры Intel 8088}}
    \end{figure}
\end{frame}

% Слайд Intel 8086/8088 - конвейер
\begin{frame}{Intel 8086/8088: конвейер}
    \begin{columns}[T,onlytextwidth]
        \begin{column}{0.5\textwidth}
            \begin{itemize}
                \begin{spacing}{0.8}
                \item Для ускорения выполнения последовательности инструкций
                \item Блоки EU и BIU работают параллельно (одновременно)
                \end{spacing}
            \end{itemize}
        \end{column}
        \begin{column}{0.5\textwidth}
            \begin{figure}[htp]
                \centering
                \includegraphics[height=2.8cm, keepaspectratio]{8088InternalPipeline.png}
                \captionsetup{skip=-8pt}
                \caption{\tiny{Конвейер Intel 8088}}
            \end{figure}
        \end{column}
    \end{columns}
    \begin{figure}[htp]
        \centering
        \includegraphics[height=1.2cm, keepaspectratio]{StandartSerialExecution.png}
        \captionsetup{skip=-5pt}
    \end{figure}
    \begin{figure}[htp]
        \centering
        \includegraphics[height=1.6cm, keepaspectratio]{8088ParallelExecution.png}
        \captionsetup{skip=-5pt}
    \end{figure}
\end{frame}

% Слайд Intel 8086/8088 - регистры
\begin{frame}{Intel 8086/8088: регистры}
    \begin{columns}[T,onlytextwidth]
        \begin{column}{0.6\textwidth}
            \begin{itemize}
                \begin{spacing}{0.8}
                \item DATA~--- AX, BX, CX, DX регистры общего назначения, для
                арифметических и логических операций; некоторые специальные
                \item POINTER и INDEX~--- BP, SP указывают на стек,  SI (Source),
                DI (Destination)~--- для работ со строками, имеют автоувеличение
                и автоуменьшение
                \item CONTROL~--- IP (указатель инструкций), flags
                (флаги)~--- результат операции
                \end{spacing}
            \end{itemize}
        \end{column}
        \begin{column}{0.4\textwidth}
            \begin{figure}[htp]
                \centering
                \includegraphics[height=3.4cm, keepaspectratio]{8088Registers.png}
                \captionsetup{skip=-8pt}
                \caption{\tiny{Регистры Intel 8088}}
                \includegraphics[height=2cm, keepaspectratio]{8088FlagRegister.png}
                \captionsetup{skip=-8pt}
                \caption{\tiny{Регистр флагов Intel 8088}}
            \end{figure}
        \end{column}
    \end{columns}
\end{frame}

% Слайд Intel 8086/8088 - организация памяти
\begin{frame}{Intel 8086/8088: организация памяти}
    \begin{columns}[T,onlytextwidth]
        \begin{column}{0.5\textwidth}
            \begin{footnotesize}
            \begin{itemize}
                \begin{spacing}{0.8}
                \item 20-разрядная шина адреса, до 1Мб памяти
                \item С точки зрения хранения: память~--- линейная
                последовательность байт
                \item С точки зрения программы: память разбивается на
                сегменты по 64Кб (логическое деление)
                \end{spacing}
                \begin{itemize}
                    \begin{footnotesize}
                    \begin{spacing}{0.8}
                    \item Каждый сегмент имеет начальный (базовый) адрес,
                    выровненный по границе 16 байт (4 младших бита == 0)
                    \item Сегменты могут пересекаться
                    \item Каждое приложение определяет и использует
                    сегменты самостоятельно
                    \item Возможно динамическое перераспределение программы в памяти
                    \end{spacing}
                    \end{footnotesize}
                \end{itemize}
            \end{itemize}
            \end{footnotesize}
        \end{column}
        \begin{column}{0.5\textwidth}
            \begin{figure}[h]
                \centering
                \includegraphics[height=2.6cm, keepaspectratio]{8086MemorySegments.png}
                \captionsetup{skip=-8pt}
                \caption{\tiny{Cегменты Intel 8088}}
                \includegraphics[height=3.2cm, keepaspectratio]{8086RegisterSegment.png}
                \captionsetup{skip=-8pt}
                \caption{\tiny{Регистры сегментов Intel 8088}}
            \end{figure}
        \end{column}
    \end{columns}
\end{frame}

% Слайд Intel 8086/8088 - формирование физического адреса
\begin{frame}{Intel 8086/8088: формирование физического адреса}
    \begin{columns}[t,onlytextwidth]
        \begin{column}{0.7\textwidth}
            \begin{itemize}
                \begin{footnotesize}
                \begin{spacing}{0.8}
                \item Физический адрес = Сегмент * 16 + Смещение
                \item Осуществляется блоком BIU
                \item Смещение (в случае эффективного адреса) вычисляется блоком EU
                \end{spacing}
                \end{footnotesize}
                \begin{figure}[h]
                    \centering
                    \includegraphics[height=2cm, keepaspectratio]{8086PhysicalAdressCalculation.png}
                    \begin{tiny}
                    \begin{tabular}{|p{2.2cm}|p{1.8cm}|p{1.8cm}|p{1.2cm}|}
                        \hline
                        \textbf{Тип ссылки} & \textbf{Базовый регистр по умолчанию} & \textbf{Альтернативный базовый регистр} & \textbf{Смещение} \\ \hline
                        Извлечение инструкции & CS & None & IP \\ \hline
                        Операции со стеком & SS & None & SP \\ \hline
                        Данные (переменные) & DS & CS, ES, SS & Эфф. адрес \\ \hline
                        Источник строки & DS & CS, ES, SS & SI \\ \hline
                        Приемник строки & ES & None & DI \\ \hline
                        BP как базовый регистр & SS & CS, DS, ES & Эфф. адрес \\ \hline
                    \end{tabular}
                    \end{tiny}
                \end{figure}
            \end{itemize}
        \end{column}
        \begin{column}{0.3\textwidth}
            \begin{figure}[h]
                \centering
                \includegraphics[height=4.8cm, keepaspectratio]{8086LogicalPhysicalAdress.png}
            \end{figure}
        \end{column}
    \end{columns}
\end{frame}

% Слайд Intel 8086/8088 - зарезервированная память
\begin{frame}{Intel 8086/8088: зарезервированная память}
    \begin{columns}[t,onlytextwidth]
        \begin{column}{0.6\textwidth}
            \begin{itemize}
                \item Intel зарезервировала часть оперативной памяти для
                специальных функций процессора
                \item Используется для прерываний и обработки системного
                сброса (system reset)
            \end{itemize}
        \end{column}
        \begin{column}{0.4\textwidth}
            \begin{figure}[t]
                \centering
                \includegraphics[height=5.4cm, keepaspectratio]{IntelReservedMemory.png}
                \captionsetup{skip=-5pt}
                \caption{\tiny{зарезервированная память}}
            \end{figure}
        \end{column}
    \end{columns}
\end{frame}

% Слайд Intel 8086/8088 - ввод/вывод
\begin{frame}{Intel 8086/8088: ввод/вывод}
    \begin{itemize}
        \item Ввод/вывод осуществляется через нумерованные ячейки~---
        так называемые порты ввода/вывода
        \item Пространство портов ввода/вывода, отдельное от памяти
        \begin{itemize}
            \item ~<<+>>: быстрая работа, не расходуется память
            \item ~<<->>: отдельные инструкции, сложнее программировать
        \end{itemize}
        \item Также может использоваться память для реализации
         портов ввода/вывода
        \item Пространство портов не сегментируется
    \end{itemize}
\end{frame}

% Слайд Intel 8086/8088 - прерывания
\begin{frame}{Intel 8086/8088: прерывания}
    \begin{spacing}{0.8}
    \begin{footnotesize}
    \begin{itemize}
        \begin{footnotesize}
            \item Прерывание~--- изменение последовательности выполнения команд
            \begin{itemize}
                \begin{footnotesize}
                \item внешние, аппаратные (при возникновении сигнала на специально выделенных для этих целей входных контактах процессора)
                \item внутренние процессора (деление на ноль, трассировка программы)
                \item программные (специальная инструкция, чаще всего для операций ввода/вывода или вызова функций BIOS)
                \end{footnotesize}
            \end{itemize}
        \item У каждого прерывания есть номер, по которому процессор определяет подпрограмму для обработки этого прерывания (обработчик прерывания)
        \begin{itemize}
            \begin{footnotesize}
            \item Поддерживается 256 прерываний
            \item Существует таблица прерываний, содержащая адреса обработчиков.  Порядковый  номер элемента в этой таблице соответствует номеру прерывания. Эта таблица хранится в памяти
            \item Содержимое элемента таблицы прерываний (вектор прерывания) – двойное слово – адрес вызываемой процедуры (обработчика)
            \end{footnotesize}
        \end{itemize}
        \item Маскируемые и немаскируемые прерывания: можно наложить маску (фильтр), которая запретит поступление определенных прерываний в процессор
        \end{footnotesize}
    \end{itemize}
    \end{footnotesize}
    \end{spacing}
\end{frame}

% Слайд Intel 8086/8088 - таблица прерываний
\begin{frame}{Intel 8086/8088: таблица прерываний, обработка аппаратных прерываний}
    \begin{columns}[t,onlytextwidth]
        \begin{column}{0.5\textwidth}
            \begin{figure}[t]
                \centering
                \includegraphics[height=5.4cm, keepaspectratio]{8086InterruptTable.png}
                \captionsetup{skip=-5pt}
                \caption{\tiny{Таблица прерываний}}
            \end{figure}
        \end{column}
        \begin{column}{0.5\textwidth}
            \begin{itemize}
                \begin{spacing}{0.8}
                \begin{footnotesize}
                    \item Маскируемые прерывания поступают через контроллер прерываний
                    \item Немаскируемые~-- через специальный контакт процессора
                \end{footnotesize}
                \end{spacing}
                \begin{figure}[t]
                    \centering
                    \includegraphics[height=3cm, keepaspectratio]{Intel8088MaskableNonMaskableInterrupt.png}
                    \captionsetup{skip=-5pt}
                    \caption{\tiny{Поступление аппаратных прерываний}}
                \end{figure}
            \end{itemize}
        \end{column}
    \end{columns}
\end{frame}

% Слайд Intel 8086/8088 - обработка прерываний
\begin{frame}{Intel 8086/8088: обработка прерываний}
    \begin{columns}[t,onlytextwidth]
        \begin{column}{0.6\textwidth}
            \begin{itemize}
                \begin{footnotesize}
                \begin{spacing}{0.8}
                    \item Содержимое регистра флагов помещается на стек
                    \item Флаги IF и TF очищаются для запрещения прерываний (очищается сигнал INTR) и возможной передачи управления отладчику
                    \item Содержимое регистра CS помещается на стек
                    \item Содержимое регистра IP помещается на стек
                    \item Извлекается вектор прерывания и его содержимое помещается в регистры IP и CS соответственно. Следующей командой будет выполняться первая команда обработчика прерывания
                    \item Возврат из обработчика прерывания должен быть осуществлен специальной инструкцией IRET, которая восстанавливает флаги и возобновляет работу с инструкции, на которую до появления прерывания указывал IP
                \end{spacing}
                \end{footnotesize}
            \end{itemize}
        \end{column}
        \begin{column}{0.4\textwidth}
            \begin{figure}[t]
                \centering
                \includegraphics[height=6cm, keepaspectratio]{8086InterruptProcessing.png}
            \end{figure}
        \end{column}
    \end{columns}
\end{frame}

% Слайд Intel 8086/8088 - аппаратные прерывания
\begin{frame}{Intel 8086/8088: аппаратные прерывания}
    \begin{footnotesize}
    \begin{table}
        \centering
        \begin{tabular}{|p{4.4cm}|p{5cm}|}
            \hline
            \textbf{Линия прерывания} & \textbf{Типичное использование} \\ \hline
            IRQ3 & Последовательный порт 2 \\ \hline
            IRQ4 & Последовательный порт 1 \\ \hline
            IRQ5 & Параллельный порт 2 \\ \hline
            IRQ6 & Драйвер НГМД \\ \hline
            IRQ7 & Параллельный порт 1 \\ \hline
            IRQ9, IRQ10, IRQ11, IRQ15 & None \\ \hline
            IRQ12 & Интерфейс мыши \\ \hline
            IRQ14 & Драйвер НЖМД \\ \hline
        \end{tabular}
        \captionsetup{skip=-5pt}
        \caption{\tiny{Линии прерываний на шине ISA}}
    % \end{table}
    % \begin{table}
    %     \centering
        \begin{tabular}{|p{3.4cm}|p{7.8cm}|}
            \hline
            \textbf{Линия прерывания} & \textbf{Типичное использование} \\ \hline
            IRQ0 & Системный таймер \\ \hline
            IRQ1 & Интерфейс клавиатуры \\ \hline
            IRQ2 & Прерывание от дополнительного контроллера PIC \\ \hline
            IRQ8 & Часы реального времени \\ \hline
            IRQ13 & Сопроцессор \\ \hline
        \end{tabular}
        \captionsetup{skip=-5pt}
        \caption{\tiny{Линии прерываний, подключенные к контроллерам прерываний}}
    \end{table}
    \end{footnotesize}
\end{frame}

% Слайд Intel 8086/8088 - команды
\begin{frame}{Intel 8086/8088: команды}
    \begin{itemize}
        \item Арифметические и логические инструкции
        \item Инструкции пересылки данных
        \item Инструкции работы со строками (до 64К)
        \item Инструкции передачи управления
        \item Инструкции управления процессором
    \end{itemize}
\end{frame}

% Слайд Intel 8086/8088 - арифметические и логические инструкции
\begin{frame}{Intel 8086/8088: команды}
    \begin{columns}[t,onlytextwidth]
        \begin{column}{0.5\textwidth}
            Арифметические и логические инструкции
            \begin{tiny}
            \begin{tabular}{|p{0.9cm}|p{5cm}|}
                \hline
                \multicolumn{2}{|c|}{\textbf{Логические}} \\ \hline
                NOT & \glqq NOT\grqq байта или слова\\ \hline
                AND & \glqq AND\grqq байт или слов\\ \hline
                OR & \glqq OR\grqq байт или слов\\ \hline
                XOR & \glqq Исключающее OR\grqq байт или слов\\ \hline
                TEST & \glqq AND\grqq байт или слов без записи результата\\ \hline
                \multicolumn{2}{|c|}{\textbf{Сдвиг}} \\ \hline
                SHL/SAL & Логический и арифметический сдвиг влево \\ \hline
                SHR & Логический сдвиг вправо \\ \hline
                SAR & Арифметический сдвиг вправо \\ \hline
                \multicolumn{2}{|c|}{\textbf{Циклический сдвиг}} \\ \hline
                ROL & Циклический сдвиг влево \\ \hline
                ROR & Циклический сдвиг вправо \\ \hline
                RCL & Циклический сдвиг влево c использованием CF \\ \hline
                RСR & Циклический сдвиг вправо c использованием CF \\ \hline
            \end{tabular}
            \end{tiny}
        \end{column}
        \begin{column}{0.5\textwidth}
            \begin{tiny}
            \begin{tabular}{|p{0.9cm}|p{5.4cm}|}
                \hline
                \multicolumn{2}{|c|}{\textbf{Сложение}} \\ \hline
                ADD & Сложение байта или слова\\ \hline
                ADС & Сложение с переносом (+ СF)\\ \hline
                INC & Увеличение на 1\\ \hline
                AAA & ACSII корректировка после сложения BCD\\ \hline
                DAA & Десятичная корректировка после сложения BCD\\ \hline
                \multicolumn{2}{|c|}{\textbf{Вычитание}} \\ \hline
                SUB & Вычитание \\ \hline
                SBB & Вычитание с заёмом (- CF) \\ \hline
                DEC & Уменьшение на 1 \\ \hline
                NEG & Изменение знака на противоположный \\ \hline
                CMP & Сравнение \\ \hline
                AAS & ACSII корректировка после вычитания BCD \\ \hline
                DAS & Десятичная корректировка после вычитания BCD \\ \hline
                \multicolumn{2}{|c|}{\textbf{Умножение}} \\ \hline
                MUL & Беззнаковое умножение \\ \hline
                IMUL & Целочисленное умножение \\ \hline
                AAM & ACSII корректировка после умножения BCD \\ \hline
                \multicolumn{2}{|c|}{\textbf{Деление}} \\ \hline
                DIV & Беззнаковое деление \\ \hline
                IDIV & Целочисленное деление \\ \hline
                AAВ & ACSII корректировка после деления BCD \\ \hline
                CBW & Преобразование байта в слово \\ \hline
                CWD & Преобразование слова в двойное слово \\ \hline
            \end{tabular}
            \end{tiny}
        \end{column}
    \end{columns}
\end{frame}

% Слайд Intel 8086/8088 - инструкции передачи управления
\begin{frame}{Intel 8086/8088: команды}
    \begin{columns}[t,onlytextwidth]
        \begin{column}{0.5\textwidth}
            Инструкции передачи управления
            \begin{tiny}
            \begin{tabular}{|p{1.8cm}|p{4.2cm}|}
                \hline
                \multicolumn{2}{|c|}{\textbf{Безусловный переход}} \\ \hline
                CALL & Вызов процедуры \\ \hline
                RET & Возврат из процедуры \\ \hline
                JMP & Переход по адресу \\ \hline
                \multicolumn{2}{|c|}{\textbf{Управление циклами}} \\ \hline
                LOOP & Переход по метке \\ \hline
                LOOPE/LOOPZ & Переход по метке, если равно/ноль \\ \hline
                LOOPE/LOOPZ & Переход по метке, если не равно/не ноль \\ \hline
                JCXZ & Переход, если регистр CX=0 \\ \hline
                \multicolumn{2}{|c|}{\textbf{Прерывания}} \\ \hline
                INT & Вызов прерывания \\ \hline
                INTO & Вызов прерывания, если переполнение \\ \hline
                IRET & Возврат из прерывания \\ \hline
            \end{tabular}
        \end{tiny}
        \end{column}
        \begin{column}{0.5\textwidth}
            \begin{tiny}
            \begin{tabular}{|p{0.9cm}|p{5.4cm}|}
                \hline
                \multicolumn{2}{|c|}{\textbf{Условный переход}} \\ \hline
                JA/JNBE & Переход, если выше/не ниже не равно (беззнаковый)\\ \hline
                JAE/JNB & Переход, если не ниже/выше или равно (беззнаковый)\\ \hline
                JB/JNAE & Переход, если ниже/не выше или не равно (беззнаковый)\\ \hline
                JBE/JNA & Переход, если ниже или равно/не выше (беззнаковый)\\ \hline
                JBС & Переход, если возник перенос\\ \hline
                JE/JZ & Переход, если равно/ноль\\ \hline
                JG/JNLE & Переход, если больше/не меньше и не равно\\ \hline
                JGE/JNL & Переход, если больше или равно/не меньше\\ \hline
                JL/JNGE & Переход, если меньше/не больше и не равно\\ \hline
                JLE/JNG & Переход, если меньше или равно/не больше\\ \hline
                JNC & Переход, если нет переноса\\ \hline
                JNE/JNZ & Переход, если не равно/не ноль\\ \hline
                JNO & Переход, если нет переполнения\\ \hline
                JNP/JPO & Переход, если нет четности/нечетная четность\\ \hline
                JNS & Переход, если нет знака\\ \hline
                JO & Переход, если есть переполнение\\ \hline
                JP/JPE & Переход, если есть четность/четность четная\\ \hline
                JS & Переход, если есть знак\\ \hline
            \end{tabular}
            \end{tiny}
        \end{column}
    \end{columns}
\end{frame}

% Слайд Intel 8086/8088 - инструкции пересылки данных
\begin{frame}{Intel 8086/8088: команды}
    \begin{columns}[t,onlytextwidth]
        \begin{column}{0.5\textwidth}
            Инструкции пересылки данных
        \end{column}
        \begin{column}{0.5\textwidth}
            \begin{figure}[р]
                \centering
                \includegraphics[height=6.4cm, keepaspectratio]{8086TransferInstruction.png}
            \end{figure}
        \end{column}
    \end{columns}
\end{frame}

% Слайд Intel 8086/8088 - инструкции работы со строками
\begin{frame}{Intel 8086/8088: команды}
    \begin{columns}[t,onlytextwidth]
        \begin{column}{0.5\textwidth}
            Инструкции работы со строками
        \end{column}
        \begin{column}{0.5\textwidth}
            \begin{figure}[р]
                \centering
                \includegraphics[height=6.4cm, keepaspectratio]{8086StringInstruction.png}
            \end{figure}
        \end{column}
    \end{columns}
\end{frame}

% Слайд Intel 8086/8088 - инструкции управления процессором
\begin{frame}{Intel 8086/8088: команды}
    \begin{columns}[t,onlytextwidth]
        \begin{column}{0.5\textwidth}
            Инструкции управления процессором
        \end{column}
        \begin{column}{0.5\textwidth}
            \begin{figure}[р]
                \centering
                \includegraphics[height=6.4cm, keepaspectratio]{8086CPUInstruction.png}
            \end{figure}
        \end{column}
    \end{columns}
\end{frame}

% Слайд Intel 8087 - сопроцессор 8087
\begin{frame}{Сопроцессор Intel 8087}
    \begin{itemize}
        \begin{footnotesize}
        \item Для увеличения быстродействия при выполнении вычислений с плавающей точкой
        \item Собственный набор инструкций
        \item Вычисление экспоненты, логарифмов, тригонометрических функций
        \item Расширенный 80-битный формат чисел
        \item CPU~--- общее управление процессом вычислений, NPX (Numeric Processor Extension)~--- только свои команды
        \end{footnotesize}
        \begin{figure}[р]
            \centering
            \includegraphics[height=2.6cm, keepaspectratio]{8087DataTypes.png}
        \end{figure}
\end{itemize}
\end{frame}

\end{document}