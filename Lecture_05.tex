\documentclass[aspectratio=169,14pt]{beamer}

\usepackage[utf8]{inputenc}
\usepackage[main=russian,english]{babel}
\usepackage[T1]{fontenc}
\usepackage[labelformat=empty]{caption}
\usepackage{multimedia}
\usepackage{xcolor}
\usepackage{setspace}

\usetheme{Singapore}

\definecolor{urlcolor}{HTML}{799B03} % цвет гиперссылок
\hypersetup{urlcolor=urlcolor, colorlinks=true}

\graphicspath{{../Images/}}

\title{Центральные процессоры: Pentium Pro - MIPS }

\begin{document}

% Слайд Intel Pentium Pro
\begin{frame}{Intel Pentium Pro}
    \begin{itemize}
        \begin{footnotesize}
        \begin{spacing}{0.8}
        \item 1995 год~--- шестое поколение (архитектура P6)~--- RISC ядро
        \item Частота 150-200 МГц, 500-350 нм, 5,5М ядро, 21М-67,5М транзисторов
        (совместно с кэшем)
        \item Шина адреса 36-битная
        \item Кэш второго уровня L2 (256Кб~- 1Мб)
        \item Микроархитектура динамического исполнения (Dynamic Execution
        Architecture)
        \item Пять параллельных блоков исполнения (два целочисленных, два FPU,
        один для операций с памятью)
        \item Двойная независимая шина
        \item Поддержка многопроцессорных систем
        \end{spacing}
        \end{footnotesize}
    \end{itemize}
    \includegraphics[height=2.4cm, keepaspectratio]{PentiumProFace.jpg}
    \includegraphics[height=2.4cm, keepaspectratio]{PentiumProUnFace.jpg}
    \includegraphics[height=2.4cm, keepaspectratio]{PentiumProOpened.jpg}
\end{frame}

% Слайд Pentium vs Pentium Pro
\begin{frame}{Intel Pentium vs Pentium Pro}
    Внутрь кристалла интегрирован кэш второго уровня,
    APIC (контроллер прерываний) и контроллер памяти

    \includegraphics[height=5.3cm, keepaspectratio]{PentiumVSPentiumPro.png}
\end{frame}

% Слайд Pentium Pro: внутренняя архитектура
\begin{frame}{Pentium Pro: внутренняя архитектура}
    \begin{figure}[htp]
        \centering
        \includegraphics[height=7.1cm, keepaspectratio]{PentiumProFunctionalUnits.png}
    \end{figure}
\end{frame}

% Слайд Pentium Pro: пул инструкций
\begin{frame}{Pentium Pro: пул инструкций}
    \begin{columns}[T,onlytextwidth]
        \begin{column}{0.6\textwidth}
            \begin{itemize}
                \begin{footnotesize}
                \item Длина конвейера~--- 12 стадий
                \item Разделение стадии исполнения на стадии Диспетчеризация / Исполнения и удаления
                \item Блок Диспетчеризации / Исполнения выбирает операции микрокода
                из пула в зависимости от их статуса и готовности операндов,
                после чего помещает их в очередь исполнения на соответствующий
                ресурс внутри этого блока. Результат помещается обратно в пул
                инструкций
                \item Блок удаления (Retire) фиксирует инструкции, удаляет
                их из пула и восстанавливает правильный порядок инструкций
                \end{footnotesize}
            \end{itemize}
        \end{column}
        \begin{column}{0.4\textwidth}
            \includegraphics[height=6.8cm, keepaspectratio]{PentiumProInstructionPool.png}
        \end{column}
    \end{columns}
\end{frame}

% Слайд Pentium Pro: микроархитектура динамического исполнения
\begin{frame}{Pentium Pro: микроархитектура динамического исполнения}
    \begin{itemize}
        \item Глубокое предсказание ветвлений
        \begin{itemize}
            \item динамическое предсказание: использование буферов BTB
            \item статическое предсказание: переходы назад сбудутся,
            вперед~--- нет
        \end{itemize}
        \item Динамический анализ потоков данных для поиска возможности
        выполнения инструкций вне порядка и восстановления их порядка
        следования
        \item Спекулятивные вычисления
    \end{itemize}
\end{frame}

% Слайд Pentium MMX
\begin{frame}{Pentium MMX}
    \begin{itemize}
        \item MMX (MultiMedia eXtension)~--- ориентирована на
        приложения мультимедиа, 2D/3D графику
        \item Реализация технологии SIMD (Single Instruction~--- Multiple
         Data)~--- одновременная обработка нескольких элементов
         данных за одну инструкцию
        \begin{itemize}
            \item Восемь 64-разрядных регистра (М0-М7)
            \item Одна инструкция может выполнять действия сразу
            над всеми комплектами операндов (8,4,2 или 1), которые
            размещены в этих регистрах
            \item 57 дополнительных инструкций
        \end{itemize}
    \end{itemize}
\end{frame}

% Слайд Pentium II
\begin{frame}{Pentium II}
    \begin{itemize}
        \item 1997 год~--- модифицированное ядро P6 (Pentium Pro) + блок ММХ
        \item Частота 233-400 МГц; 7,5~--- 27М транзисторов;
        350-180 нм
    \end{itemize}
    \includegraphics[height=5cm, keepaspectratio]{PentiumIIFace.jpg}
\end{frame}

% Слайд Pentium III
\begin{frame}{Pentium III}
    \begin{itemize}
        \item 1999 год~--- поколение P6, модифицированное ядро
        Deschutes, которое использовалось в Pentium II
        \item Частота 450-1400 ГГц;  9,5М-44М транзисторов; 250-130 нм
        \item Улучшенная технология SIMD (технология SSE)
        \item Pentium III на ядре Tualatin~--- аппаратная
        предвыборка данных
    \end{itemize}
    \includegraphics[height=3cm, keepaspectratio]{PentiumIIIFace.jpg}
    \includegraphics[height=3cm, keepaspectratio]{PentiumIIIUnface.jpg}
\end{frame}

% Слайд Pentium III: внутренняя архитектура
\begin{frame}{Pentium III: внутренняя архитектура}
    \begin{figure}[htp]
        \centering
        \includegraphics[height=7.1cm, keepaspectratio]{PentiumIIIFunctionalUnits.png}
    \end{figure}
\end{frame}

% Слайд Pentium III: SSE
\begin{frame}{Pentium III: SSE}
    \begin{itemize}
        \item SSE - Streaming SIMD Extensions – потоковое расширение SIMD
        \item Данная технология решает две проблемы MMX:
        \begin{itemize}
            \item При использовании MMX невозможно использовать
            инструкции сопроцессора (поскольку задействованы
            одинаковые регистры)
            \item MMX работает только с целыми числами
        \end{itemize}
        \item Восемь 128-битных регистра и набор инструкций, работающих со скалярными и упакованными типами данных
    \end{itemize}
\end{frame}

% Слайд Pentium 4
\begin{frame}{Pentium 4}
    \begin{itemize}
        \item 2000 год~--- архитектура седьмого поколения NetBurst (P68)
        \item Частота 1300-3800МГц, 42-188М транзисторов, технология 180-65 нм
        \item Гиперконвейер~--- большое число стадий, что дает
        возможность для повышения частоты
    \end{itemize}
    \begin{figure}[htp]
        \centering
        \includegraphics[height=4cm, keepaspectratio]{Pentium4Face.jpg}
    \end{figure}
\end{frame}

% Слайд Pentium 4: внутренняя архитектура
\begin{frame}{Pentium 4: внутренняя архитектура}
    \begin{figure}[htp]
        \centering
        \includegraphics[height=6.9cm, keepaspectratio]{Pentium4FunctionalUnits.png}
    \end{figure}
\end{frame}

% Слайд Pentium 4: Гиперконвейер
\begin{frame}{Pentium 4: Гиперконвейер}
    \begin{itemize}
        \item Глубина: 20 (31 на ядрах Prescott и Cedar) стадий, в связи
        с применением кэша последовательностей микроопераций, декодер
        вынесен за пределы конвейера
        \item Позволяет достигать высоких тактовых частот
        \item Недостатки:
        \begin{itemize}
            \item уменьшение удельной производительности (за один
            такт меньшее количество инструкций)
            \item существенные потери производительности при
            интрукциях перехода или кэш-промахе
        \end{itemize}
        \item Увеличенный буфер предсказания ветвлений
    \end{itemize}
\end{frame}

% Слайд Pentium 4: кэш последовательности микроопераций
\begin{frame}{Pentium 4: кэш последовательности микроопераций}
    \begin{itemize}
        \item Перед исполнением инструкции из набора x86 преобразуются
        в набор внутренних команд (микроопераций)
        \item Отказ от кэша кода и хранение в кэше
        последовательности микроопераций
        \item Операция cохранения (store) разделена на операции
        сохранения данных и сохранения адреса
    \end{itemize}
    \includegraphics[height=2.9cm, keepaspectratio]{Pentium4PipelineStructure.png}
\end{frame}

% Слайд Pentium 4: Hyper-Threading
\begin{frame}{Pentium 4: Hyper-Threading}
    \begin{columns}[T,onlytextwidth]
        \begin{column}{0.65\textwidth}
            \begin{itemize}
                \begin{tiny}
                \begin{spacing}{0.8}
                \item Для увеличения производительности используются различные подходы:
                конвейеризация, суперскалярность, увеличение кэш-памяти, исполнение вне порядка следования команд
                \item Hyper-Threading~-- иной подход: попытка более полно загрузить внутренние ресурсы процессора
                \item Процессор выполняет одновременно более одного потока (приложения)
                \item Физический процессор воспринимается операционной системой и приложениями как два логических процессора
                \item Необходима поддержка со стороны операционной системы
                \end{spacing}
                \end{tiny}
            \end{itemize}
        \end{column}
        \begin{column}{0.35\textwidth}
            \includegraphics[height=3.5cm, keepaspectratio]{Pentium4H-T.png}
        \end{column}
    \end{columns}
        \includegraphics[width=7.4cm]{Pentium4H-TPipeline.png}
\end{frame}

% Слайд Intel Core 2
\begin{frame}{Intel Core 2}
    \begin{itemize}
        \begin{footnotesize}
        \item 2006~---  Intel Core 2, 64-разрядные процессоры, набор команд Intel~64
        \item Несколько процессорных ядер
        \item Архитектура Intel Core
        \begin{itemize}
            \begin{footnotesize}
            \item Широкое динамическое исполнение
            \item Уменьшение потребления питания
            \item Улучшенный доступ к кэш-памяти
            \item Улучшенный доступ к памяти
            \item Выполнение 128 битный SIMD-инструкций за один такт
            \end{footnotesize}
        \end{itemize}
        \end{footnotesize}
    \end{itemize}
    \begin{figure}[htp]
        \centering
        \includegraphics[height=3.2cm, keepaspectratio]{Core2FloorPlan.png}
    \end{figure}
\end{frame}

% Слайд Intel Core: внутренняя архитектура
\begin{frame}{Intel Core: внутренняя архитектура}
    \begin{columns}
        \begin{column}{0.5\textwidth}
            \begin{figure}
                \centering
                \includegraphics[height=5.2cm, keepaspectratio]{CoreDuoWideDynamicExecution.png}
                \caption{Ядра Intel Core}
            \end{figure}
        \end{column}
        \begin{column}{0.5\textwidth}
            \begin{figure}
                \centering
                \includegraphics[height=5.2cm, keepaspectratio]{CoreMicroarchitecture.png}
                \caption{Архитектура ядра Intel Core}
            \end{figure}
        \end{column}
    \end{columns}
\end{frame}

% Слайд MIPS архитектура
\begin{frame}{MIPS архитектура}
    \begin{itemize}
        \begin{footnotesize}
        \begin{spacing}{0.8}
        \item MIPS (Microprocessor without Interlocked Pipeline Stages)~--- микропроцессор без блокирующегося конвейера
        \item Представлен в 1981 году, модификации MIPS I, MIPS II, MIPS III, MIPS IV, MIPS V, MIPS32 и MIPS64
        \item Смартфоны, маршрутизаторы, принтеры, игровые консоли и т.п.
        \item Типичный представитель RISC (Reduced Instruction Set Computers) архитектуры:
        \begin{itemize}
            \begin{footnotesize}
            \item короткий цикл
            \item эффективное использование площади кристалла из-за простого набора команд
            \item возможность применения новых технологий изготовления полупроводников
            \end{footnotesize}
        \end{itemize}
        \item R4000~--- 64 разрядный процессор, 64 разрядное ALU и FPU,
        64 разрядные регистры, 64 разрядное виртуальное адресное пространство
        \end{spacing}
        \end{footnotesize}
    \end{itemize}
    \includegraphics[height=2.8cm, keepaspectratio]{MIPSR4000Face.png}
    \includegraphics[height=2.8cm, keepaspectratio]{MIPSR4000UnFace.png}
\end{frame}

% Слайд MIPS внутренняя архитектура
\begin{frame}{MIPS внутренняя архитектура}
    \begin{columns}
        \begin{column}{0.35\textwidth}
            \begin{itemize}
                \item 32 регистра общего назначения
                \item 3 специальных регистра
                \item Инструкции фиксированной длины
            \end{itemize}
        \end{column}
        \begin{column}{0.65\textwidth}
            \includegraphics[height=7cm, keepaspectratio]{MIPSR4000FunctionalUnits.png}
        \end{column}
    \end{columns}
\end{frame}

% Слайд MIPS R4000: виртуальная память
\begin{frame}{MIPS R4000: виртуальная память}
    \begin{columns}
        \begin{column}{0.55\textwidth}
            \begin{itemize}
                \begin{footnotesize}
                \begin{spacing}{0.9}
                \item ASID (Address Space Identifier)~--- идентификатор адресного пространства уменьшает количество очисток TLB и соотносит адресное пространство с процессом
                \item Отдельные адресные пространства USER, KERNEL и SUPERVISOR
                \item Размер VPN (Virtual Page Number) зависит от размера страницы
                \end{spacing}
                \end{footnotesize}

                \includegraphics[width=5.5cm, keepaspectratio]{MIPSR4000MemoryAddress.png}
            \end{itemize}
        \end{column}
        \begin{column}{0.45\textwidth}
            \begin{figure}[c]
                \includegraphics[height=5.8cm, keepaspectratio]{MIPSR4000AddressSpacesKernelMode.png}
            \end{figure}
        \end{column}
    \end{columns}
\end{frame}

% Слайд MIPS R4000: конвейер
\begin{frame}{MIPS R4000: конвейер}
    \begin{itemize}
        \item Целочисленный
        \item Для операций с плавающей точкой (FPU)
    \end{itemize}

    \begin{figure}[c]
        \centering
        \includegraphics[width=7cm, keepaspectratio]{MIPSR4000IntegerPipeLine.png}
        \caption{\footnotesize{Целочисленный конвейер}}
    \end{figure}

    \begin{figure}[c]
        \centering
        \includegraphics[height=2cm, keepaspectratio]{MIPSR4000FPPipeLine.png}
        \caption{\footnotesize{Конвейер FPU}}
    \end{figure}
\end{frame}

% Слайд ARM Cortex-A8
\begin{frame}{ARM Cortex-A8}
    \begin{itemize}
        \item 32-х битный процессор, архитектура ARM v7-A
        \item Два конвейера
        \item Интегрированный кеш L2
        \item 600 MHz~--- 1 GHz и выше
        \item RICS архитектура
    \end{itemize}
\end{frame}

% Слайд ARM Cortex-A8: внутренняя архитектура
\begin{frame}{ARM Cortex-A8: внутренняя архитектура}
    \begin{figure}[c]
        \centering
        \includegraphics[width=9.2cm, keepaspectratio]{CortexA8FunctionalUnits.png}
    \end{figure}
\end{frame}


\end{document}
