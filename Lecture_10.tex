\documentclass[aspectratio=169,14pt]{beamer}

\usepackage[utf8]{inputenc}
\usepackage[main=russian,english]{babel}
\usepackage[T1]{fontenc}
\usepackage[labelformat=empty]{caption}
% \usepackage{listings}
\usepackage{multimedia}
\usepackage{xcolor}
% \usepackage{hyperref}
\usepackage{setspace}
\usepackage{verbatim}
\usepackage{multirow}

\usetheme{Singapore}

\definecolor{urlcolor}{HTML}{799B03} % цвет гиперссылок
\hypersetup{urlcolor=urlcolor, colorlinks=true}

\graphicspath{{../Images/}}

\title{Планирование процессов}

\begin{document}

% Слайд Планирование процессов
\begin{frame}{Планирование процессов}
    \begin{itemize}
        \item В многозадачных операционных системах возможны ситуации, когда
        более одного процесса находятся в состоянии готовности и хотят
        получить доступ к процессору
        \item Планировщик~--- часть операционной системы, которая реализует
        алгоритм планирования
        \item Существуют различные типы поведения процессов:
        \begin{itemize}
            \item ограничены вычислительными возможностями ЭВМ
            \item ограничены возможностями ввода/вывода
        \end{itemize}
    \end{itemize}
\end{frame}

% Слайд Когда планировать?
\begin{frame}{Когда планировать?}
    \begin{itemize}
        \item При завершении процесса (обязательно)
        \item При блокировании на ввод/вывод или примитивами синхронизации
        с <<засыпанием>> (обязательно)
        \item При создании нового процесса
        \item При прерывании ввода/вывода
        \item При прерывании от таймера
        \begin{itemize}
            \item \textbf{невытесняющий} алгоритм планирования: процесс
            выполняется до блокирования или завершения
            \item \textbf{вытесняющий} алгоритм планирования: процесс
            выполняется в течение некоторого времени, затем принудительно
            переключается
        \end{itemize}
    \end{itemize}
\end{frame}

% Слайд Окружение и категории алгоритмов планирования
\begin{frame}{Окружение и категории алгоритмов планирования}
    Среды (окружение), в которых происходит планирование:
    \begin{itemize}
        \item \textbf{Системы пакетной обработки}: нет пользователей, нужна
        производительность. Подходят невытесняющие алгоритмы или вытесняющие
        с длительным промежутком исполнения
        \item \textbf{Интерактивные системы}: необходимо вытеснение
        \item \textbf{Системы реального времени}: вытеснение необязательно, поскольку
        процессы короткие
    \end{itemize}
\end{frame}

% Слайд Цели алгоритмов планирования
\begin{frame}{Цели алгоритмов планирования}
\begin{itemize}
    \begin{spacing}{0.8}
    \begin{footnotesize}
    \item Цели зависят от среды (окружения), но выделяются и общие
    \item Все системы:
    \begin{itemize}
        \begin{tiny}
        \begin{spacing}{0.8}
        \item равноправие: предоставление каждому процессу справедливой доли
        процессорного времени
        \item применение политик: наблюдение за соблюдением установленной
        политики (пример политики: процессы безопасности могут быть запущены
        в любой момент, даже если нужно задержать другие процессы)
        \item баланс: обеспечение работой всех компонентов системы
        \end{spacing}
        \end{tiny}
    \end{itemize}
    \item Системы пакетной обработки:
    \begin{itemize}
        \begin{tiny}
        \begin{spacing}{0.8}
        \item пропускная способность: выполнение максимального количества
        заданий в единицу времени
        \item время оборота: минимизация времени, затрачиваемого на ожидание
        обслуживания и обработку задания
        \item коэффициент использования процессора: обеспечение постоянной
        занятости процессора
        \end{spacing}
        \end{tiny}
    \end{itemize}
    \item Интерактивные системы:
    \begin{itemize}
        \begin{tiny}
            \begin{spacing}{0.8}
            \item время отклика: быстрая реакция на запросы
            \item пропорциональность: соответствие ожиданиям пользователя
            (пользователи имеют распространенное представление, сколько
            выполняется та или иная задача)
            \end{spacing}
        \end{tiny}
    \end{itemize}
    \item Системы реального времени:
    \begin{itemize}
        \begin{tiny}
            \begin{spacing}{0.8}
            \item соответствие временным ограничениям: во избежание потери
            данных, поступающих непрерывно
            \item предсказуемость: избежание потери качества в мультимедиа
            системах (возможна потеря небольшого объема данных, но время
            исполнения процесса должно быть предсказуемо в целом)
            \end{spacing}
        \end{tiny}
    \end{itemize}
    \end{footnotesize}
    \end{spacing}
\end{itemize}
\end{frame}

% Слайд Планирование в системах пакетной обработки
\begin{frame}{Планирование в системах пакетной обработки}
    \begin{footnotesize}
    \begin{spacing}{0.8}
        \begin{itemize}
            \item Первым пришел – первым обслужен
            \begin{itemize}
                \begin{tiny}
                \begin{spacing}{0.8}
                    \item Процессы обслуживаются в порядке поступления запросов
                    на использование процессора.
                    \item Новые процессы помещаются в конец очереди
                    \item Заблокированные процессы помещаются в конец очереди
                    \item ~<<$+$>>: простота и легкость
                    \item ~<<$-$>>: неоптимальная загрузка ресурсов
                    (вычислительный процесс, изредка обращающийся к
                    вводу/выводу и несколько процессов со многими операциями
                    ввода/вывода; последние будут ждать первого)
                \end{spacing}
                \end{tiny}
            \end{itemize}
            \item Самое короткое задание – первое
            \begin{itemize}
                \begin{tiny}
                \begin{spacing}{0.8}
                    \item Предположение: время выполнения процессов известно
                    заранее (на рисунке A,B,C,D~--- задания, числа~--- время
                    исполнения)
                    \item ~<<$+$>>: минимальное время оборота
                    \item ~<<$-$>>: должны быть в наличии сразу все задания
                \end{spacing}
                \end{tiny}
            \end{itemize}
            \item Задание с наименьшим временем выполнения~--- следующее
            \begin{itemize}
                \begin{tiny}
                    \begin{spacing}{0.8}
                    \item Первым выбирается задание с наименьшим оставшимся
                    временем исполнения
                    \item Время выполнения заданий должно быть известно
                    \item Если вновь появившееся задание можно выполнить
                    быстрее~--- текущее приостанавливается и запускается новое
                    \end{spacing}
                \end{tiny}
            \end{itemize}
            \includegraphics[height=1.8cm, keepaspectratio]{OSPlanningShortFirst.png}

        \end{itemize}
    \end{spacing}
    \end{footnotesize}
\end{frame}

% Слайд Трехуровневое планирование
\begin{frame}{Трехуровневое планирование}
    \begin{itemize}
        \item Входная очередь хранится на жестком диске
        \item Планирование допуска:
        \begin{itemize}
            \item сочетание заданий для процессора и ввода/вывода
            \item короткие задания сначала
        \end{itemize}
        \item Планировщик памяти работает, когда не хватает оперативной памяти
        для всех процессов; перемещение заданий в/из памяти - долгий процесс
        \item Планировщик процессора использует подходящий алгоритм

        \includegraphics[height=3cm, keepaspectratio]{OS3LevelPlanning.png}
    \end{itemize}
\end{frame}

% Слайд Планирование в интерактивных системах
\begin{frame}{Планирование в интерактивных системах}
    \begin{itemize}
        \begin{spacing}{0.9}
        \item Только двухуровневое планирование
        \item Циклическое (карусельное) планирование
            \begin{itemize}
            \item каждому процессу предоставляется квант времени
            \item по истечении кванта времени происходит переключение на
            следующий процесс, а текущий отправляется в конец очереди
            \item прост в реализации (нужно поддерживать список готовых к
            исполнению процессов)
            \item важно выбрать размер кванта времени
            \end{itemize}
        \item Приоритетное планирование
        \begin{itemize}
            \begin{footnotesize}
            \begin{spacing}{0.8}
            \item каждому процессу присваивается приоритет и управление
            передается процессу с наибольшим приоритетом
            \item приоритет остальных процессов увеличивается с каждым
            тактом (или переключением между другими процессами), после
            выполнения приоритет восстанавливается до первоначального значения
            \item приоритет может задаваться статически, может изменяться
            динамически
            \item удобно считать приоритет в обратном порядке, т.е.
            минимальное числовое значение означает максимальный приоритет
            \end{spacing}
            \end{footnotesize}
        \end{itemize}
        \end{spacing}
    \end{itemize}
\end{frame}

% Слайд Планирование в интерактивных системах
\begin{frame}{Планирование в интерактивных системах}
    \begin{itemize}
        \item Многоуровневое планирование:
        \begin{itemize}
            \item процессы группируются в классы по приоритетам
            \item между классами применяется приоритетное планирование,
            внутри класса~--- циклическое
            \includegraphics[height=4cm, keepaspectratio]{OSPlanningPriorityClass.png}
        \end{itemize}
    \end{itemize}
\end{frame}

% Слайд Планирование в интерактивных системах
\begin{frame}{Планирование в интерактивных системах}
    \begin{columns}
        \begin{column}{0.5\textwidth}
            Многоуровневые очереди с обратной связью:
            \begin{itemize}
                \begin{footnotesize}
                \begin{spacing}{1.0}
                \item возможна миграция процессов между очередями приоритетов
                (за счет механизма обратной связи)
                \item переход из очереди 0 в очередь 1 и далее осуществляется,
                если процесс полностью израсходовал свой квант времени
                \item FCFS~--- <<первый пришел~--- первый обслужен>>
                \end{spacing}
                \end{footnotesize}
            \end{itemize}
        \end{column}
        \begin{column}{0.5\textwidth}
            \includegraphics[height=6.8cm, keepaspectratio]{OSMultiLevelQueue.png}
        \end{column}
    \end{columns}
\end{frame}

% Слайд
\begin{frame}{Планирование в интерактивных системах}
    Гарантированное планирование
    \begin{itemize}
        \item $N$ интерактивно работающих пользователя
        \item Гарантия: каждый из пользователей получит $1/N$ часть
        процессорного времени
        \item $T_i$~--- время нахождения пользователя в системе (время с
        начала сеанса), $t_i$~--- полученное время процессора
        \item Коэффициент справедливость для каждого пользователя: $t_i \cdot N / T_i$
        \item Квант времени получает пользователь (один из процессов
        пользователя) с минимальным коэффициентом справедливости
    \end{itemize}
\end{frame}

% Слайд Планирование в интерактивных системах
\begin{frame}{Планирование в интерактивных системах}
    Лотерейное планирование
    \begin{itemize}
        \item Процессам раздаются <<лотерейные билеты>>
        \item В момент принятия решения выбирается случайный билет и
        управление передается процессу, у которого этот билет находится
        \item Для реализации приоритетов билеты могут выдаваться неравномерно,
        в разных объемах
        \item Возможна передача билетов между процессами
        \item Легко реализуется точная балансировка нагрузки и обеспечение
        качества обслуживания
        \item Прототип планировщика реализован для микроядра Mach~3.0
    \end{itemize}
\end{frame}

% Слайд Планирование в системах реального времени
\begin{frame}{Планирование в системах реального времени}
    \begin{itemize}
        \item Свойства систем реального времени
        \begin{itemize}
            \item Жесткие системы реального времени (обязательно укладываться
            в сроки выполнения задания)
            \item Мягкие системы реального времени
        \end{itemize}
        \item Короткие, предсказуемые процессы, которые обрабатывают
        (реагируют) на внешние события
        \item Внешние события
        \begin{itemize}
            \item периодические
            \item непериодические
        \end{itemize}
    \end{itemize}
\end{frame}

% Слайд Планирование в системах реального времени
\begin{frame}{Планирование в системах реального времени}
    \begin{itemize}
        \begin{footnotesize}
        \item $C$ (computing)~--- время, необходимое для обработки события
        \item $R$~--- время впуска задания (совпадает с временем поступления
        события для упрощения моделей, пренебрегаем переключением)
        \item $T$~--- период поступления события
        \item $D$~--- время, до истечения которого должна закончится
        обработка события (совпадает с $T$ для упрощения модели)
        \item $C / T$~--- утилизация (загрузка) ЦПУ задачей
        \item $U = \sum {C_i / T_i}$ – утилизация ЦПУ всеми задачами
        \item Если $U > 1$ – перегрузка процессора
        \item Если $U \leq 1$ – все задания будут закончены к сроку
        \end{footnotesize}
        \includegraphics[height=3cm, keepaspectratio]{OSPeriodicTask.png}
    \end{itemize}
\end{frame}

% Слайд Планирование в системах реального времени: RMS
\begin{frame}{Планирование в системах реального времени: RMS}
    \begin{itemize}
        \begin{spacing}{0.9}
        \item RMS (Rate-monotonic Scheduling)~--- планирование с приоритетом,
        пропорциональным частоте
        \begin{itemize}
            \begin{footnotesize}
            \begin{spacing}{0.9}
                \item каждый периодический процесс должен быть завершен за
                время его периода
                \item ни один процесс не должен зависеть от любого другого
                \item каждому процессу требуется одинаковое процессорное время
                на каждом интервале
                \item у непериодических процессов нет жестких сроков
                \item прерывание происходит без накладных расходов (для
                упрощения модели системы)
            \end{spacing}
            \end{footnotesize}
        \end{itemize}
        \item Используется для процессов, удовлетворяющих условиям:
        \item Каждому процессу назначается приоритет, равный частоте
        возникновения событий процесса (чем меньше задача~-- тем выше
        приоритет). Всегда запускается процесс с наивысшим приоритетом,
        прерывая по необходимости работающий процесс
        \end{spacing}
    \end{itemize}
\end{frame}

% Слайд Планирование в системах реального времени: RMS
\begin{frame}{Планирование в системах реального времени: RMS}
    \begin{columns}
        \begin{column}{0.6\textwidth}
            \begin{itemize}
                \item Не всегда работает, даже если $U < 1$, например:
                \begin{itemize}
                    \item $T_1(2,5); T_2(4,7)$
                    \item $ U = 2/5 + 4/7 = 34/35$
                    \item $Pr(T_1)=1, Pr(T_2)=2$
                \end{itemize}
            \end{itemize}
        \end{column}
        \begin{column}{0.4\textwidth}
            \includegraphics[width=6cm, keepaspectratio]{OSRMSNotWork.png}
        \end{column}
    \end{columns}
    \begin{itemize}
        \begin{spacing}{0.86}
        \item Достаточное условие планирования: $U \leq n*(2^{1/n}-1)$
        \item ~<<+>>
        \begin{itemize}
            \item легкий в реализации
            \item стабильный (низкоприоритетные задания могут не успевать
            обрабатываться, но высокоприоритетные будут работать)
        \end{itemize}
        \item ~<<--->>
        \begin{itemize}
            \item низкая загрузка ЦПУ
            \item требование $D = T$
            \item работа только с независимыми заданиями
        \end{itemize}
        \item Все <<--->> можно решить, кроме низкой загрузки
        \end{spacing}
    \end{itemize}
\end{frame}

% Слайд Планирование в системах реального времени: EDF
\begin{frame}{Планирование в системах реального времени: EDF}
    \begin{footnotesize}
    \begin{spacing}{0.8}
    \begin{itemize}
        \item EDF (Earliest Deadline First) – вначале процессы с самым коротким
        временем окончания
        \item Когда поступает новая задача, происходит сортировка очереди
        задач таким образом, что ближайшая к своему завершению получает
        наивысший приоритет
        \item Текущая задача продолжает выполнятся, если только новая не
        получает наивысший приоритет
        \item Если любой другой алгоритм сможет распланировать задачи,
        то и EDF сможет
        \item $U \leq 1$ – необходимое и достаточное условие планирование
        \item ~<<+>>
        \begin{itemize}
            \begin{tiny}
            \begin{spacing}{1}
            \item простой в теории и реализации
            \item простая проверка на возможность планирования
            \item оптимальный
            \item наилучшая загрузка ЦПУ
            \end{spacing}
            \end{tiny}
        \end{itemize}
        \item ~<<--->>
        \begin{itemize}
            \begin{tiny}
            \begin{spacing}{1}
            \item сложно реализовать на практике, нужно быстро сортировать
            очередь
            \item нестабильный, нельзя предсказать, какие задачи могут быть
            невыполнены
            \end{spacing}
            \end{tiny}
        \end{itemize}
    \end{itemize}
    \end{spacing}
    \end{footnotesize}
\end{frame}

% Слайд Планирование в ОС Linux
\begin{frame}{Планирование в ОС Linux}
    \begin{spacing}{0.9}
    \begin{itemize}
        \item Три типа процессов:
        \begin{itemize}
            \begin{footnotesize}
            \item Интерактивные: постоянно взаимодействующие с пользователем,
            при поступлении команды (прерывания) от пользователя должны быть
            быстро <<разбужены>>.
            \item Пакетные процессы: работают в фоновом режиме и не нуждаются
            во взаимодействии с пользователем
            \item Процессы реального времени: жесткие требования к
            планированию, нельзя блокировать ради процессов с низким
            приоритетом
            \end{footnotesize}
        \end{itemize}
        \item Планировщик Linux использует эвристический алгоритм для
        определения интерактивного или пакетного процесса, предпочитает
        интерактивные процессы
        \item Реализованы три класса планирования:
        \begin{itemize}
            \begin{footnotesize}
            \item SCHED\_FIFO: процессы, работающие в реальном времени,
            планируются по принципу <<первым вошел---первым обслужен>>
            \item SCHED\_RR: процессы, работающие в реальном времени,
            планируются по круговому принципу
            \item SCHED\_NORMAL: обычные процессы
            \end{footnotesize}
        \end{itemize}
    \end{itemize}
    \end{spacing}
\end{frame}

% Слайд Планирование в ОС Linux: SCHED_NORMAL
\begin{frame}{Планирование в ОС Linux: SCHED\_NORMAL}
    \begin{footnotesize}
    \begin{spacing}{0.9}
    \begin{itemize}
        \item Каждый процесс имеет свой статический приоритет (от 100,
        наивысший, до 139)
        \item Статический приоритет определяет базовый квант времени
        \item Каждый процесс имеет динамический приоритет (от 100 до 139)
         = max(100, min(статический-бонус+5, 139))
        \item Бонус - число от 0 до 10, зависит от среднего времени сна
        процесса (предыдущая история процесса)
        \item Среднее время сна~--- среднее количество наносекунд,
        проведенных процессом в ожидании (среднее, поскольку в разных
        режимах ожидания время считается неравномерно), не более 1 секунды,
        уменьшается пока процесс работает
        \item Существует таблица соответствия между среднем временем сна
        и бонусом
        \item Среднее время сна учитывается при отнесении процесса к
        интерактивным или пакетным
    \end{itemize}
    \end{spacing}
    \end{footnotesize}
    \includegraphics[height=1.8cm, keepaspectratio]{OSLinuxProcessPlanningBQT.png}
\end{frame}

% Слайд Планирование в ОС Linux: SCHED_NORMAL
\begin{frame}{Планирование в ОС Linux: SCHED\_NORMAL}
    \begin{footnotesize}
    \begin{spacing}{0.8}
    \begin{itemize}
        \item для предотвращения голодания процесс, исчерпавший квант времени,
        замещается низкоприоритетным процессом, чей квант времени не истек;
        для реализации этого механизма поддерживается два непересекающихся
        набора выполняющихся процессов:
        \begin{itemize}
            \begin{footnotesize}
            \item \textbf{активные}: не исчерпали своего кванта времени, им
            разрешено работать
            \item \textbf{с истекшим квантом}: процессам запрещено работать,
            пока у всех активных не закончатся кванты времени
            \end{footnotesize}
        \end{itemize}
        \item повышение интерактивных процессов:
        \begin{itemize}
            \begin{tiny}
            \begin{spacing}{0.8}
            \item активный пакетный процесс, исчерпавший квант, обязательно переходит в список процессов с истекшим квантом
            \item активный интерактивный процесс, исчерпавший квант, обычно остается активным, планировщик выделяет ему новый квант
            \item активный интерактивный процесс, исчерпавший квант, переносится в список процессов с истекшим квантом, если более «старый» процесс с истекшим квантом ждет достаточно долго или если процесс с истекшим квантом имеет более высокий приоритет
            \end{spacing}
            \end{tiny}
        \end{itemize}
        \includegraphics[height=2.2cm, keepaspectratio]{OSLinuxRunQueueStructure.png}
    \end{itemize}
    \end{spacing}
    \end{footnotesize}
\end{frame}

% Слайд Планирование в ОС Linux: процессы реального времени
\begin{frame}{Планирование в Linux: процессы реального времени}
    \begin{footnotesize}
    \begin{itemize}
        \item Каждый процесс реального времени имеет свой приоритет реального
        времени, от 1 до~99
        \item Планировщик выбирает процессы с высоким приоритетом
        \item Процессы реального времени всегда считаются активными
        \item Процессы с одинаковым приоритетом обслуживаются в круговой очереди
        \item Процесс реального времени замещается:
        \begin{itemize}
            \begin{footnotesize}
            \item При появлении процесса с высоким приоритетом
            \item При блокировки, остановке, уничтожении, добровольном
            освобождении
            \item Процесс работает в реальном времени по дисциплине
            SCHED\_RR и исчерпал свой квант времени
            \end{footnotesize}
        \end{itemize}
        \item Продолжительность базового кванта времени у процессов реального
        времени, работающих по принципу SCHED\_RR, зависит от статического
        приоритета а не от приоритета реального времени
    \end{itemize}
    \end{footnotesize}
\end{frame}

% Слайд Планирование в ОС Windows
\begin{frame}{Планирование в ОС Windows}
    \begin{footnotesize}
    \begin{itemize}
        \item Выделены две группы приоритетов: реального времени и переменные
        \item Вытесняющий планировщик с учетом приоритетов
        \item Если появляется поток с более высоким приоритетом, текущий поток
        вытесняется
        \item В классе потоков реального времени все потоки имеют
        фиксированный приоритет (от 16 до 31); потоки с одинаковым
        приоритетом располагаются в круговой очереди
        \item В классе переменных приоритетов
        \begin{itemize}
            \begin{tiny}
            \item Поток начинает работать с базовым приоритетом (1-15)
            \item Приоритет увеличивается
            \begin{itemize}
            \begin{tiny}
                \item когда завершается операция ввода/вывода на связанном с ней потоке, слагаемое зависит от устройства (1 – для диска, 2 – для последовательной линии, 6 – для клавиатуры и 8 – для звуковой карты)
                \item при разблокировании на семафоре (+2, +1)
                \item при пробуждении графического потока при доступном оконном вводе
            \end{tiny}
            \end{itemize}
            \item Приоритет уменьшается, когда поток исчерпал свой квант (-1, но до базового уровня)
            \item Если окно становится окном переднего плана, все его потоки получают более длительный квант времени
            \item Потоки с одинаковым приоритетом располагаются в круговой очереди
            \end{tiny}
        \end{itemize}
    \end{itemize}
    \end{footnotesize}
\end{frame}

% Слайд Планирование в ОС Windows
\begin{frame}{Планирование в ОС Windows}
    \begin{figure}[htp]
        \centering
        \includegraphics[width=11cm, keepaspectratio]{OSWindowsPlanning.jpg}
    \end{figure}
\end{frame}

\end{document}