\documentclass[aspectratio=169,14pt]{beamer}

\usepackage[utf8]{inputenc}
\usepackage[main=russian,english]{babel}
\usepackage[T1]{fontenc}
\usepackage[labelformat=empty]{caption}
% \usepackage{listings}
\usepackage{multimedia}
\usepackage{xcolor}
% \usepackage{hyperref}
\usepackage{setspace}
\usepackage{verbatim}
\usepackage{multirow}

\usetheme{Singapore}

\definecolor{urlcolor}{HTML}{799B03} % цвет гиперссылок
\hypersetup{urlcolor=urlcolor, colorlinks=true}

\graphicspath{{../Images/}}

\title{Управление памятью}

\begin{document}

% Слайд Управление памятью
\begin{frame}{Управление памятью}
    \begin{spacing}{0.9}
    \begin{itemize}
        \item Память~--- важнейший ресурс ЭВМ, требующий тщательного
        управления
        \item Иерархия памяти
        \begin{itemize}
            \item регистры
            \item кэш-память
            \item ОЗУ
            \item внешняя память
        \end{itemize}
        \item Менеджер памяти (модуль управления памятью)~--- часть
        ОС, отвечающая за управление памятью
        \begin{itemize}
            \item Следит, какая часть памяти используется,
            а какая свободна
            \item Выделяет/освобождает память для процессов
            \item Управляет обменом память-диск
        \end{itemize}
        \item Типы менеджеров памяти:
        \begin{itemize}
            \item Без перемещения (без использования внешней памяти)
            \item С перемещением
        \end{itemize}
    \end{itemize}
    \end{spacing}
\end{frame}

% Слайд Однозадачная система без подкачки на диск
\begin{frame}{Однозадачная система без подкачки на диск}
    \begin{itemize}
        \item В каждый момент времени работает только
        одна программа
        \item Память разделяется между ОС и программой
        \item При запуске программы ОС копирует её с диска
        в память и передает ей управление. По окончании загружает
        другую программу поверх первой.

        \includegraphics[height=4.2cm, keepaspectratio]{OSMemorySingleTask.png}
    \end{itemize}
\end{frame}

% Слайд Многозадачная система c фиксированными разделами
\begin{frame}{Многозадачная система c фиксированными разделами}
    \begin{footnotesize}
    \begin{spacing}{0.8}
    \begin{itemize}
        \item Память разделяется на несколько разделов, возможно
        разного размера
        \begin{itemize}
        \begin{tiny}
            \item общая очередь процессов ко всем разделам
            \item отдельные очереди к разделам
        \end{tiny}
        \end{itemize}
        \item Недостаток нескольких очередей~--- возможное простаивание
        раздела
        \item Алгоритмы планирования в случае одной очереди:
        \begin{itemize}
            \begin{tiny}
            \begin{spacing}{0.8}
            \item первый пришел~--- первый обслужен, циклический
            \item выбирается задача, которая максимальной займет раздел;
            выбор осуществляется при освобождении раздела; возможна
            дискриминация <<небольших>> задач (решение~--- выделение
            специального раздела <<небольшого>> раздела).
            \end{spacing}
            \end{tiny}
        \end{itemize}
        \item OS/MFT (Multiprogramming with Fixed number Tasks)
        \includegraphics[height=2.6cm, keepaspectratio]{OSMemoryFixedPartitions.png}
    \end{itemize}
    \end{spacing}
    \end{footnotesize}
\end{frame}

% Слайд Многозадачная система с динамическим разделами
\begin{frame}{Многозадачная система с динамическим разделами}
    \begin{footnotesize}
    \begin{spacing}{0.8}
    \begin{itemize}
        \item Недостатки фиксированных разделов:
        \begin{itemize}
            \begin{footnotesize}
            \item ограничение на количество задач;
            \item неэффективное использование в случае небольших задач.
            \end{footnotesize}
        \end{itemize}
        \item Решение: динамическое количество разделов переменного размера
        \begin{itemize}
            \begin{footnotesize}
            \item изначально вся память пуста (кроме занятой ОС)
            \item процессы последовательно загружаются в память, начиная
            с адреса после ОС
            \end{footnotesize}
        \end{itemize}
        \item Недостаток: фрагментация~--- наличие большого числа небольших
        разделов, в которые не умещается процесс. Решается с помощью
        уплотнения (перемещение всех занятых участков, чтобы свободная
        память образовывала единую область) (трудоемкая задача)
    \end{itemize}
    \begin{figure}[htp]
        \centering
        \includegraphics[height=2.6cm, keepaspectratio]{OSMemoryDynamicPartitions.png}
    \end{figure}
    \end{spacing}
    \end{footnotesize}

\end{frame}

% Слайд Многозадачная система с динамическим разделами: рост разделов
\begin{frame}{Многозадачная система с динамическим разделами: рост разделов}
    \begin{itemize}
        \item Возможна ситуация, когда процессу нужен дополнительный
        объем памяти
        \item Возможное решение:
        \begin{itemize}
            \item заранее выделять больше места
            \item расширять раздел, если нет возможности для
            расширения~--- перемещать в новый раздел
        \end{itemize}
        \includegraphics[height=3.2cm, keepaspectratio]{OSMemoryGrowPartitions.png}
    \end{itemize}
\end{frame}

% Слайд Дополнительные задачи ОС
\begin{frame}{Дополнительные задачи ОС}
    \begin{footnotesize}
    \begin{itemize}
        \item Использование сложных моделей размещения процессов в памяти
        ставит перед ОС дополнительные задачи:
        \begin{itemize}
            \begin{footnotesize}
            \item настройка адресов
            \item защита адресного пространства
            \end{footnotesize}
        \end{itemize}
        \item Глобальное решение: оснащение процессора специальными
        регистрами
        \begin{itemize}
            \begin{footnotesize}
            \item базовый (указывает на начало адресного пространства процесса)
            \item предельный (указывает конец или размер адресного
            пространства процесса)
            \end{footnotesize}
        \end{itemize}
        \includegraphics[height=3.6cm, keepaspectratio]{OSMemoryAddressSetup.png}
    \end{itemize}
    \end{footnotesize}
\end{frame}

% Слайд Учет свободной памяти: битовые массивы
\begin{frame}{Учет свободной памяти: битовые массивы}
    \begin{itemize}
        \item Вся память разбивается на блоки
        \item Выделяется специальный массив, в котором каждый бит означает,
        что соответствующий по порядку блок занят (значение~1) или свободен
        (значение~0)
        \item При выделении памяти происходит поиск последовательности
        нулей нужной длины
        \item Недостаток~--- долгий поиск (искомая последовательность может
        пересекать границы слов и нужно многократное сравнение слов)
    \end{itemize}
    \begin{figure}[htp]
        \centering
        \includegraphics[height=1.2cm, keepaspectratio]{OSMemoryBitsArray.png}
    \end{figure}
\end{frame}

% Слайд Учет свободной памяти: связные списки
\begin{frame}{Учет свободной памяти: связные списки}

    \begin{footnotesize}
        \begin{itemize}
            \item Создается список, каждый элемент которого содержит
            \begin{itemize}
                \begin{footnotesize}
                \item признак занят (P) или свободен (H) фрагмент
                памяти
                \item адрес начала фрагмента
                \item длина фрагмента
                \end{footnotesize}
            \end{itemize}
            \item Алгоритмы выделения блока памяти:
            \begin{itemize}
            \begin{footnotesize}
                \item первый подходящий участок
                \item следующий подходящий участок, стартует не с
                начала списка, а с того места, на котором остановились
                в предыдущий раз (немного медленнее, показало
                моделирование)
                \item самый подходящий участок (медленнее, сильная
                фрагментация маленькими фрагментами)
                \item самый неподходящий участок, расчет на наличие больших
                остатков (моделирование показало, что работает не очень
                хорошо)
            \end{footnotesize}
            \end{itemize}
            \includegraphics[height=1cm, keepaspectratio]{OSMemoryPH.png}
        \end{itemize}
    \end{footnotesize}
\end{frame}

% Слайд Использование внешней памяти
\begin{frame}{Использование внешней памяти}
    \begin{itemize}
        \item Часто возникают ситуации, когда процесс надолго занят операцией
        ввода/вывода. В этом случае целесообразно выгрузить его во внешнюю
        память, освободив ОЗУ
        \item Своппинг (подкачка)~--- процесс целиком выгружается/загружается
        на диск (во внешнюю память)
        \item Виртуальная память~--- процесс частично может быть загружен в
        память
    \end{itemize}
    \begin{figure}[htp]
        \centering
        \includegraphics[height=2.9cm, keepaspectratio]{OSMemorySwapping.png}
    \end{figure}
\end{frame}

% Слайд Виртуальная память: страничная организация
\begin{frame}{Виртуальная память: страничная организация}
    \begin{footnotesize}
    \begin{itemize}
        \item Страничная организация памяти~--- способ реализации виртуальной
        памяти
        \item Страницы~--- части, на которые разбивается пространство
        виртуальных адресов
        \item Страничные блоки~--- единицы физической памяти
        \item Страницы имеют фиксированный размер. Передача между ОЗУ и
        диском осуществляется в страницах.
        \item Страничное прерывание~--- происходит, если процесс обратился к
        странице, которая не загружена в ОЗУ
    \end{itemize}
    \end{footnotesize}
    \begin{figure}[htp]
        \centering
        \includegraphics[height=3.4cm, keepaspectratio]{OSMemoryVirtualPhysicalAddresses.png}
    \end{figure}
\end{frame}

% Слайд Таблицы страниц
\begin{frame}{Таблицы страниц}
    \begin{itemize}
        \item Таблицы страниц используются для хранения соответствия
        адресов виртуальной страницы и страничного блока
        \item Таблица страниц может быть размещена:
        \begin{itemize}
            \item в ОЗУ (дешево, долго);
            \item в аппаратных регистрах (буфер быстрого преобразования,
            TLB: translation lookaside buffer).
        \end{itemize}
   \end{itemize}
   \begin{figure}[htp]
    \centering
    \includegraphics[width=13cm, keepaspectratio]{OSMemoryPageTableRecord.png}
    \end{figure}
\end{frame}

% Слайд Использование таблицы страниц
\begin{frame}{Использование таблицы страниц}
    Пример использования таблицы страниц в системе из 16 страниц по 4Кб
    \begin{figure}[htp]
        \centering
        \includegraphics[height=6cm, keepaspectratio]{OSMemorySimpePageTableUse.png}
    \end{figure}
\end{frame}

% Слайд Многоуровневые таблицы страниц
\begin{frame}{Многоуровневые таблицы страниц}
    Многоуровневые таблицы используются для решения проблемы
    большого размера таблицы страниц
    \begin{figure}[htp]
        \centering
        \includegraphics[height=6cm, keepaspectratio]{OSMemoryTwoLevelPageTable.png}
    \end{figure}
\end{frame}

% Слайд TLB
\begin{frame}{TLB}
    \begin{footnotesize}
    \begin{itemize}
        \item Буфер быстрого преобразования используется для повышения
        быстродействия преобразования виртуального адреса в физический
        \item Используется ассоциативный доступ (поиск одновременно во
        всех ячейках TLB)
    \end{itemize}
    \end{footnotesize}
    \begin{figure}[htp]
        \centering
        \includegraphics[height=4.4cm, keepaspectratio]{OSMemoryTLB.png}
        \includegraphics[height=4.4cm, keepaspectratio]{OSMemoryTLBOnly.png}
    \end{figure}
\end{frame}

% Слайд Алгоритмы замещения страниц
\begin{frame}{Алгоритмы замещения страниц}
    \begin{itemize}
        \item Во время возникновения ошибки отсутствия страниц операционная
        система должна выбрать страницу для удаления, чтобы освободить место
        для затребованной страницы
        \item Произвольный выбор страницы для удаления приводит к
        существенному снижению производительности
        \item Оптимальный алгоритм замещения:
        \begin{itemize}
            \item каждая страница помечается количеством команд, которые
            должны быть выполнены до обращения к этой странице
            \item выгружается страница с максимальной меткой.
        \end{itemize}
        \item Алгоритм нереализуем, поскольку неизвестно, когда произойдет
        обращение к странице
    \end{itemize}
\end{frame}

% Слайд Алгоритм NRU
\begin{frame}{Алгоритм NRU}
    \begin{spacing}{0.9}
    \begin{itemize}
        \item \textbf{NRU} (Not Recently Used)~--- не использовавшаяся в
        последнее время
        \item Идея: если страница не использовалась в последнее время,
        значит, она не будет использоваться в дальнейшем
        \item Для каждой страницы устанавливаются два бита
        \begin{itemize}
            \item $R = 1$, если было чтение из страницы; периодически $R$
            обнуляется
            \item $M = 1$, если была модификация страницы
        \end{itemize}
        \item Все страницы делятся на 4 класса:
        \begin{itemize}
            \item класс 0: не было обращений и изменений ($R=0, M=0$)
            \item класс 1: обращений не было, страница изменена ($R=0, M=1$)
            \item класс 2: обращение было, страница не изменена ($R=1, M=0$)
            \item класс 3: было обращение, страница изменена ($R=1, M=1$)
        \end{itemize}
        \item Выбирается произвольная страница из непустого минимального
        класса
    \end{itemize}
    \end{spacing}
\end{frame}

% Слайд Алгоритм FIFO
\begin{frame}{Алгоритм FIFO}
    \begin{itemize}
        \item \textbf{FIFO} (First in~--- first out)~--- первым пришел, первым ушел
        \item Идея: если страница <<старая>>, то к ней не было и не будет
        обращений
        \item Поддерживается список всех страниц, удаляется первая из этого
        списка
        \item Недостаток: можно удалить часто используемую страницу
    \end{itemize}
\end{frame}

% Слайд Алгоритм «второй шанс»
\begin{frame}{Алгоритм <<второй шанс>>}
    \begin{itemize}
        \item Идея: если страница <<старая>>, то к ней не было и не будет
        обращений; дополнительно проверятся бит $R$
        \item Если $R = 0$, страница удаляется
        \item Если $R = 1$, $R$ обнуляется и страница попадает в конец
        списка (обновляется время ее загрузки)
        \includegraphics[height=3.6cm, keepaspectratio]{OSMemorySecondChance.png}
    \end{itemize}
\end{frame}

% Слайд Алгоритм часов
\begin{frame}{Алгоритм часов}
    \begin{itemize}
        \item Алгоритм <<второй шанс>> не эффективен, требует модификации
        списка
        \item Идея: страницы хранятся к кольцевом списке и добавляется
        указатель на текущую страницу~--- кандидат для удаления
        \item По сути, отличается от <<второго шанса>> реализацией
        \includegraphics[height=3.8cm, keepaspectratio]{OSMemoryClock.png}
    \end{itemize}
\end{frame}

% Слайд Алгоритм LRU
\begin{frame}{Алгоритм LRU}
    \begin{spacing}{0.8}
    \begin{footnotesize}
    \begin{itemize}
        \item LRU (Least Recently Used)~--- дольше всего не использовавшаяся
        страница
        \item Идея: страницы, к которым ранее не было обращений, не потребуются
        в дальнейшем
        \item Реализуем, но трудоемок:
        \begin{itemize}
            \begin{tiny}
            \item либо поддерживать список страниц в нужном порядке
            \item либо использовать аппаратный счетчик, который увеличивается
            после каждой команды
            \end{tiny}
        \end{itemize}
        \item Вариант аппаратной реализации: матрица обращения
        \begin{itemize}
            \begin{tiny}
            \item для $n$ страниц используется матрица $n*n$
            \item при доступе к страничному блоку $k$ всей $k$-ой строке
            присваиваются единицы, а $k$-му столбцу присваиваются нули
            \item строка с наименьшим двоичным значением является дольше
            всего неиспользуемой
            \end{tiny}
        \end{itemize}
        \includegraphics[height=3cm, keepaspectratio]{OSMemoryLRUMatrix.png}
    \end{itemize}
    \end{footnotesize}
    \end{spacing}
\end{frame}

% Слайд Модель рабочего набора
\begin{frame}{Модель рабочего набора}
    \begin{tiny}
    \begin{spacing}{0.4}
    \begin{itemize}
        \item В момент запуска процесса нужны страницы отсутствуют в памяти
        \item Через некоторое время в памяти скапливается достаточное
        количество необходимых процессу страниц и он начинает работать с
        небольшим количеством страничных прерываний
        \item Процессы характеризуются локальностью обращений, во время
        выполнения любой фазы процесс обращается к небольшой части
        собственных страниц
        \item Множество страниц, которое процесс использует в данный момент,
        называется рабочим набором
        \item Системы замещения страниц пытаются отслеживать рабочий набор
        процесса и обеспечивают его нахождение в памяти еще о запуска
        процесса~--- модель рабочего процесса
        \item $k$~--- количество обращений к памяти. Пусть для каждого
        момента времени $t$ есть набор, включающий все страницы,
        $к$ которым было $k$ последних обращений к памяти
        \item $w(k,t)$~--- рабочий набор
        \includegraphics[width=11cm, keepaspectratio]{OSMemoryWorkSet.png}
    \end{itemize}
    \end{spacing}
    \end{tiny}
\end{frame}

% Слайд Модель рабочего набора: wsclock
\begin{frame}{Модель рабочего набора: wsclock}
    \begin{footnotesize}
    \begin{itemize}
        \item Из-за асимптотического поведения $w(k,t)$ содержимое рабочего
        набора нечувствительно к значению $k$; существует большое количество
        значений $k$, при которых рабочие наборы одинаковы
        \item Для реализации механизмов рабочего набора операционная
        система должна отслеживать, какие страницы в нем находятся
        \item Возможная реализация~--- использовать механизм <<старения>>
        страниц
        \begin{itemize}
            \begin{footnotesize}
            \item у каждой страницы есть счетчик нахождения в памяти (время
            последнего использования)
            \item старший бит счетчика может означать присутствие страницы в
            рабочем наборе
            \item если за $n$ последовательных тактов обращения к странице не
            происходит, она убирается из рабочего набора
            \end{footnotesize}
        \end{itemize}
        \item \textbf{wsclock}~--- алгоритм <<часы>> с дополнительной проверкой
        присутствия страницы в рабочем наборе
    \end{itemize}
    \end{footnotesize}
\end{frame}

% Слайд Linux 2.6 Алгоритм PFRA, логика утилизации
\begin{frame}{Linux 2.6 Алгоритм PFRA, логика утилизации}
    \begin{figure}[htp]
        \centering
        \includegraphics[height=7cm, keepaspectratio]{OSMemoryPFRAUtilizationLogic.png}
    \end{figure}
\end{frame}

\end{document}