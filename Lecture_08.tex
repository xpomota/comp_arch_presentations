\documentclass[aspectratio=169,14pt]{beamer}

\usepackage[utf8]{inputenc}
\usepackage[main=russian,english]{babel}
\usepackage[T1]{fontenc}
\usepackage[labelformat=empty]{caption}
\usepackage{multimedia}
\usepackage{xcolor}
% \usepackage{hyperref}
\usepackage{setspace}
% \usepackage{verbatim}
% \usepackage{multirow}

\usetheme{Singapore}

\definecolor{urlcolor}{HTML}{799B03} % цвет гиперссылок
\hypersetup{urlcolor=urlcolor, colorlinks=true}

\graphicspath{{../Images/}}

\title{История ОС. Основные понятия ОС}

\begin{document}

% Слайд Введение в операционные системы
\begin{frame}{Введение в операционные системы}
    \begin{itemize}
        \item В составе компьютера выделяют три основных блока: процессор,
        память, устройства ввода-вывода
        \item Написание программ, которые корректно и оптимально работают
        с этим блоками, является крайне трудной задачей
        \item Операционная система~--- специальный уровень программного
        обеспечения, отвечает за управление всеми устройствами
        компьютера и обеспечивает пользователя простыми, доступными
        средствами для работы с аппаратурой
    \end{itemize}
\end{frame}

% Слайд Уровни абстракций ЭВМ
\begin{frame}{Уровни абстракций ЭВМ}
    \begin{columns}
        \begin{column}{0.65\textwidth}
            \begin{itemize}
                \begin{footnotesize}
                \begin{spacing}{1}
                    \item На нижнем уровне находятся физические устройства~---
                    микропроцессоры как наборы интегральных схем, микросхемы
                    доступа к памяти, физические устройства ввода/вывода и т.п.
                    \item Далее следует микроархитектурный уровень, на котором
                    физические устройства рассматриваются в виде блоков,
                    выполняющих элементарные операции, рассмотренные на
                    цифровом логическом уровне
                    \item Для того, чтобы пользователь получил доступ к
                    функциональным блокам уровня микроархитектуры,
                    предлагается набор инструкций, или набор команд,
                    называемый машинным языком
                \end{spacing}
                \end{footnotesize}
            \end{itemize}
        \end{column}
        \begin{column}{0.35\textwidth}
            \includegraphics[width=5cm, keepaspectratio]{CompStructureDetail.png}
        \end{column}
    \end{columns}
\end{frame}

% Слайд Структура компьютера и ПО
\begin{frame}{Структура компьютера и ПО}
    \begin{itemize}
        \item Программы и приложения на верхнем уровне обеспечивают потребности
        людей, которые используют компьютеры для решения своих задач
        \item Промежуток между этими уровнями достаточно большой и без
        наличия дополнительных уровней абстракций решение практических
        задач становится очень и очень трудным
        \item Дальнейшее изложение курса будет посвящено уровню операционных
        систем
    \end{itemize}
\end{frame}

% Слайд Назначение ОС
\begin{frame}{Назначение ОС}
    \begin{itemize}
        \item \textbf{Управление ресурсами компьютера}. ОС обеспечивает
        организованное и контролируемое распределение ресурсов компьютера
        между различными приложениями (процессами).
        \begin{itemize}
            \item Ресурсы разделяются во времени
            \item Ресурсы разделяются в пространстве
        \end{itemize}
        \item \textbf{Расширение возможностей ЭВМ}. Работа с оборудованием на
        уровне машинного языка примитивна и трудоемка. Программистам гораздо
        проще и удобнее использовать абстракции более высокого уровня.
        Операционная система предоставляет такого уровня абстракции.
    \end{itemize}
\end{frame}

% Слайд История ОС: I поколение
\begin{frame}{История ОС: I поколение}
    \begin{itemize}
        \item Программы на первых ЭВМ реализовывались в виде коммутационных
        панелей и представляли собой схемы для прямых численных вычислений
        \item На смену коммутационным панелям пришли перфокарты
        \item Отсутствуют языки программирования, даже ассемблер
        \item Уровень развития оборудования пока не позволяет говорить об
        операционных системах
    \end{itemize}
\end{frame}

% Слайд История ОС: II поколение
\begin{frame}{История ОС: II поколение}
    \begin{itemize}
        \begin{footnotesize}
        \begin{spacing}{0.8}
        \item Транзисторы: существенно выросли возможности ЭВМ. Появляются
        большие ЭВМ, называемые мейнфреймами. Складывается четкое разделение
        труда между проектировщиками, сборщиками, операторами, программистами
        и обслуживающим персоналом.
        \item Вместо машинных кодом применяют мнемонические обозначения и
        используют специальные программы~--- сборщики программ из небольших
        фрагментов кода~--- ассемблеры.
        \item Появляется FORTRAN~--- язык программирования высокого уровня,
        имеющий транслятор.
        \item Процесс разработки программы
        \begin{itemize}
            \begin{footnotesize}
            \item Программист записывал задание на бумаге
            \item Затем переносил задание на перфокарты
            \item Колода перфокарт передавалась оператору для ввода.
            \item По окончании работы программы результаты распечатывались на принтере.
            \item Если в процессе расчетов был необходим компилятор языка FORTRAN, то оператор загружал его отдельно.
            \end{footnotesize}
        \end{itemize}
        \end{spacing}
        \end{footnotesize}
    \end{itemize}
\end{frame}

% Слайд История ОС: системы пакетной обработки
\begin{frame}{История ОС: системы пакетной обработки}
    \begin{itemize}
        \begin{footnotesize}
        \begin{spacing}{0.8}
            \item Пакет~--- набор заданий (различных программ).
            \item С перфокарт на недорогих компьютерах задания переписывались
            на магнитную ленту.
            \item Затем магнитная лента передавалась для выполнения на
            <<большом>>, производительном компьютере~--- мейнфрейме.
            \item Выходные данные также записывались на ленту, после окончания
            задания данные с этой ленты распечатывались на принтере на
            отдельной недорогой ЭВМ.
            \item Для формирования задания использовались управляющие
            перфокарты: \$ЗАДАНИЕ, \$FORTRAN, \$ЗАГРУЗИТЬ, \$ЗАПУСТИТЬ, \$КОНЕЦ.
            \item Управляющие перфокарты~--- предшественники современных языков
            программирования и интерпретаторов команд.
            \item Fortran Monitor System, IBSYS
        \end{spacing}
        \end{footnotesize}
        \includegraphics[height=2.5cm, keepaspectratio]{PacketJobProcess.png}
    \end{itemize}
\end{frame}

% Слайд История ОС: Fortran Monitor System
\begin{frame}{История ОС: Fortran Monitor System}
    \begin{itemize}
        \item \textbf{Fortran Monitor System} представлял собой набор небольших
        процедур, которые помещались между отдельными заданиями или частями
        заданий в колоде перфокарт или на магнитной ленте
        \item Процедуры позволяли записывать текущее задание или его результаты
        на диск, загружать следующее в очереди задание в память и передавать
        ему управление
        \item В состав \textbf{FMS} включался ассемблер, позволяющий создавать код
        программы, пригодный для загрузки и запуска и позволяющий работать
        совместно с кодом на языке Fortran
    \end{itemize}
\end{frame}

% Слайд История ОС: III поколение
\begin{frame}{История ОС: III поколение}
    \begin{itemize}
        \begin{footnotesize}
        \begin{spacing}{1}
        \item Совместные периферийные операции в режиме подключения~---
        способность считывания с перфокарт на диск по мере их поступления в
        машинный зал (подкачка данных, спулинг, spooling~--- SPOOL
        (Simultaneous Peripheral Operation On Line). Когда текущее задание
        заканчивалось, операционная система могла загрузить новое задание с
        диска в освободившийся раздел памяти.
        \item Многозадачность~--- возможность одновременного использования
        различных ресурсов машины (процессор, память, устройства ввода/вывода)
        разными приложениями.
        \item Системы разделения времени~--- вариант многозадачности, в
        которой у каждого пользователя есть свой терминал.
        \item OS/360
        \item CTSS (Compatible Time Sharing System), система с разделением
        времени, разработана в MIT
        \end{spacing}
        \end{footnotesize}
    \end{itemize}
\end{frame}

% Слайд История ОС: IV поколение
\begin{frame}{История ОС: IV поколение}
    \begin{itemize}
        \begin{spacing}{0.9}
        \item Появление больших интегральных схем привело к созданию
        микрокомпьютеров для персонального использования
        \item \textbf{CP/M} (Control Program for Microcomputers)~--- работа с диском,
        запуск программ (для систем на базе Intel 8080)
        \item \textbf{DOS}, \textbf{MSDOS} (Disk Operation System)~--- для систем на базе
        Intel~x86
        \item Графический пользовательский интерфейс (окна, меню, мышь)~--- Даг
        Энгельбарт, 60-е годы двадцатого века
        \item \textbf{Apple Macintosh}~--- дружеский пользовательский интерфейс
        \item \textbf{Windows}~---ОС с графическим пользовательским
        интерфейсом, сначала поверх MSDOS, затем самостоятельная
        \end{spacing}
    \end{itemize}
\end{frame}

% Слайд История ОС: дерево UNIX
\begin{frame}{История ОС: дерево UNIX}
    \begin{figure}[htp]
        \centering
        \includegraphics[width=10.4cm, keepaspectratio]{UnixHistory.png}
        % \captionsetup{skip=-5pt}
        % \caption{\tiny{Микросхемы поддержки Intel 8088}}
    \end{figure}
\end{frame}

% Слайд Основные понятия и функции ОС
\begin{frame}{Основные понятия и функции ОС}
    \begin{itemize}
        \item Процесс~--- программа во время её выполнения (абстракция,
        описывающая выполняющуюся программу)
        \item Многозадачность~--- одновременное нахождение в памяти
        нескольких процессов
        \item Адресное пространство процесса~--- адреса памяти, которые
        использует/может использовать процесс
        \item Файл~--- именованный набор данных
        \item Файловая система~--- способ (метод) организации файлов, включая
        отображение файлов на физический носитель
        \item Управление процессами
        \item Управление памятью
        \item Управление устройствами ввода/вывода
    \end{itemize}
\end{frame}

% Слайд Системные вызовы
\begin{frame}{Системные вызовы}
    \begin{itemize}
        \begin{footnotesize}
        \begin{spacing}{0.9}
        \item Интерфейс между пользовательскими программами и операционной
        системой стоится в основном на абстракциях
        \item Преимущества дополнительного программного слоя:
        \begin{itemize}
            \begin{footnotesize}
            \item облегчается программирование, потому что программисты
            избавлены от необходимости изучать низкоуровневые характеристики
            аппаратных устройств
            \item повышается безопасность системы, поскольку ядро может
            проверить корректность запроса на уровне интерфейса до выполнения
            этого запроса
            \item программы становятся более переносимыми, позволяя
            компилировать и корректно выполнять их в каждом ядре
            операционной системы, предлагающем такой же набор интерфейсов
            \end{footnotesize}
        \end{itemize}
        \item Системный вызов~--- обращение прикладной программы к ядру
        операционной системы для выполнения какой-либо операции
        \item Приложения не могут получать доступ напрямую к системным ресурсам
        \item API (Application Programmer Interface, интерфейс прикладного
        программирования) представляет собой определение функции, описывающее,
        как получить определенный сервис или услугу. Системный вызов~---
        непосредственное обращение к ядру ОС
        \end{spacing}
        \end{footnotesize}
    \end{itemize}
\end{frame}

% Слайд Выполнение системного вызова
\begin{frame}{Выполнение системного вызова}
    \begin{itemize}
        \begin{footnotesize}
        \item Помещение параметров системного вызова на стек
        \item Помещение номера системного вызова в регистр
        \item Переключение в режим ядра
        \begin{itemize}
            \begin{footnotesize}
            \item или прерывание int \$0x80
            \item или инструкция sysenter (>Pentium II)
            \end{footnotesize}
        \end{itemize}
        \item Проверка допустимости системного вызова
        \item Вызов служебной процедуры, ассоциированной с номером системного вызова, который хранится в регистре
        \item Возврат к вызывающей процедуре
        \end{footnotesize}
        \includegraphics[height=3.2cm, keepaspectratio]{SysCallExecution.png}
    \end{itemize}
\end{frame}

% Слайд Категории системных вызовов
\begin{frame}{Категории системных вызовов}
    \begin{columns}
        \begin{column}{0.5\textwidth}
            \begin{itemize}
                \item \footnotesize{Управление процессами}
                \begin{itemize}
                    \begin{footnotesize}
                    \item load
                    \item execute, end (exit)
                    \item abort/создание процесса (fork в Unix-like,
                    NtCreateProcess в WindowsNT Native API)
                    \item get/set process attributes
                    \item wait время, события, signal события
                    \item allocate, free memory
                    \end{footnotesize}
                \end{itemize}
                \item \footnotesize{Работа с файлами}
                \begin{itemize}
                    \begin{footnotesize}
                    \item create file, delete file
                    \item open, close
                    \item read, write, reposition
                    \item get/set file attributes
                    \end{footnotesize}
                \end{itemize}
            \end{itemize}
        \end{column}
        \begin{column}{0.5\textwidth}
            \begin{itemize}
                \begin{spacing}{0.9}
                \item \footnotesize{Управление устройствами}
                \begin{itemize}
                    \begin{footnotesize}
                    \item request device, release device
                    \item read, write, reposition
                    \item get/set device attributes
                    \item logically attach or detach devices
                    \end{footnotesize}
                \end{itemize}
                \item \footnotesize{Работа с информацией}
                \begin{itemize}
                    \begin{footnotesize}
                    \item get/set time or date
                    \item get/set system data
                    \item get/set process, file, or device attributes
                    \end{footnotesize}
                \end{itemize}
                \item \footnotesize{Связь, коммуникация}
                \begin{itemize}
                    \begin{footnotesize}
                    \item create, delete communication connection
                    \item send, receive messages
                    \item transfer status information
                    \item attach or detach remote devices
                    \end{footnotesize}
                \end{itemize}
                \end{spacing}
            \end{itemize}
        \end{column}
    \end{columns}
\end{frame}

% Слайд Структура ОС: монолитная
\begin{frame}{Структура ОС: монолитная}
    \begin{itemize}
        \begin{footnotesize}
        \begin{spacing}{0.9}
        \item В общем случае структура отсутствует, ОС реализуется в виде
        набора процедур, вызывающих друг друга
        \item Некоторая структура наблюдается при реализации системных
        вызовов: выполняется специальная инструкция перехвата управления
        (вызов ядра или вызов супервизора), переключает машину в режим ядра
        \item Такая реализация позволяет выделить следующие уровни:
        \begin{itemize}
        \begin{footnotesize}
        \begin{spacing}{0.8}
            \item Главная программа вызывает требуемую сервисную процедуру
            \item Набор сервисных процедур, выполняющих системные вызовы
            \item Набор утилит, обслуживающих служебные процедуры
        \end{spacing}
        \end{footnotesize}
        \end{itemize}
        \item Для каждого системного вызова существует реализующая его
        сервисная процедура
        \end{spacing}
        \end{footnotesize}
        \includegraphics[height=2.8cm, keepaspectratio]{OSMonoStructure.png}
    \end{itemize}
\end{frame}

% Слайд Структура ОС: многоуровневая
\begin{frame}{Структура ОС: многоуровневая}
    \begin{itemize}
        \begin{footnotesize}
        \begin{spacing}{0.8}
        \item Организация ОС в виде иерархии уровней
        \item Первая ОС с такой структурой: THE (Э.Дейкстра)
        \begin{itemize}
        \begin{footnotesize}
        \begin{spacing}{0.8}
            \item 5 Оператор
            \item 4 Пользовательские программы
            \item 3 Управление вводом/выводом
            \item 2 Взаимодействие <<оператор-процесс>> (между консолью
            оператора и процессами)
            \item 1 Управление памятью и барабаном~--- выделение процессам
            пространства в ОЗУ и магнитном барабане (виртуальная память)
            \item 0 Выделение процессора и многозадачность
        \end{spacing}
        \end{footnotesize}
        \end{itemize}
        \item MULTICS
        \begin{itemize}
        \begin{footnotesize}
            \item Уровни~--- серия концентрических колец
            \item Внутренние кольца более привилегированные, чем внешние
            \item Вызов внешним кольцом процедуры внутреннего
            осуществляется эквивалентно системному вызову
        \end{footnotesize}
        \end{itemize}
        \item THE: многоуровневая схема только как конструктивное решение,
        все части системы собраны в одном файле
        \item MULTICS: механизм разделения колец действовал во время
        исполнения на аппаратном уровне, можно запустить программу в
        нужном кольце (при наличии привилегий)
        \end{spacing}
        \end{footnotesize}
    \end{itemize}
\end{frame}

% Слайд Структура ОС: многоуровневая, UNIX
\begin{frame}{Структура ОС: многоуровневая, UNIX}
    \includegraphics[width=7cm, keepaspectratio]{OSUNIXMultiLevelStructure.png}
    \includegraphics[width=6.8cm, keepaspectratio]{OSUnixKernel.png}
\end{frame}

% Слайд Структура ОС: многоуровневая, WINDOWS 2000
\begin{frame}{Структура ОС: многоуровневая, WINDOWS~2000}
    \begin{figure}[htp]
        \centering
        \includegraphics[width=10.2cm, keepaspectratio]{OSWin2000Structure.png}
    \end{figure}
\end{frame}

% Слайд Структура ОС: виртуальные машины VM/370
\begin{frame}{Структура ОС: виртуальные машины VM/370}
    \begin{itemize}
        \begin{footnotesize}
        \item CP/CMS (позже VM/370) система с разделением времени для IBM/360
        \begin{itemize}
        \begin{footnotesize}
            \item многозадачность
            \item расширенная машина с более удобным интерфейсом
            доступа к оборудованию
        \end{footnotesize}
        \end{itemize}
        \item Монитор виртуальной машины: предоставляет верхнему уровню не
        одну, а несколько виртуальных машин
        \item Виртуальная машина не является расширенной, а предоставляет
        точную аппаратную копию, включая ввод/вывод, прерывания и т.п.
        \item На разных виртуальных машинах могут запускаться разные ОС
        (OS/360~--- пакетные задания, CMS~--- Conversational Monitor
        System, система диалоговой обработки)
        \item В настоящее время~--- VMWare, VirtualBox, \ldots
        \end{footnotesize}
    \end{itemize}
    \includegraphics[height=2.4cm, keepaspectratio]{OSVM370.png}
\end{frame}

% Слайд Структура ОС: экзоядро
\begin{frame}{Структура ОС: экзоядро}
    \begin{itemize}
        \begin{footnotesize}
        \item В VM/370 каждый процесс пользователя получает точную копию
        настоящей машины
        \item На Pentium, в режиме виртуальной машины 8086, каждый
        пользовательский процесс получает точную копию машины
        \item Развитие идеи: система, в которой каждый пользователь
        получает абсолютную копию машины, но со своим подмножеством
        ресурсов (память, диск, \ldots)
        \item На нижнем уровне в режиме ядра работает экзоядро
        (exokernel)~--- программа для распределения ресурсов для
        виртуальных машин и проверки их использования.
        \item Преимущества:
        \begin{itemize}
        \begin{footnotesize}
            \item нет уровня отображения ресурсов
            \item отделение многозадачности (в экзоядре) от операционной
            системы пользователя осуществляется с меньшими затратами,
            необходимо лишь не допускать вмешательства работы виртуальных
            машин
        \end{footnotesize}
        \end{itemize}
        \item Идея осталась на уровне исследовательского проекта (ExOS, Nemesis)
        \end{footnotesize}
    \end{itemize}
\end{frame}

% Слайд Структура ОС: клиент-сервер
\begin{frame}{Структура ОС: клиент-сервер}
    \begin{itemize}
        \begin{footnotesize}
        \item В современных ОС существует тенденция переноса кода на верхние
        уровни и минимизация ядра
        \item Решение большинства задач операционной системы перекладывается
        на пользовательские процессы
        \item Пользовательский процесс (клиент) посылает запрос серверному
        процессу, который его обрабатывает и высылает ответ обратно
        \item Задача ядра~--- управление взаимодействием между клиентами
        и серверами
        \item Некоторые функции невозможно выполнить из пространства
        пользователя
        \begin{itemize}
            \begin{tiny}
            \begin{spacing}{0.8}
                \item возможно запускать такие процессы в режиме ядра с сохранением модели взаимодействия клиент-сервер (ранние версии MINIX, драйверы компилировались в ядро, но запускались как отдельные процессы)
                \item встраивание в ядро минимальных механизмов обработки, но вынос «политических» решений в пользовательское пространство (например, операции чтения/записи с диска в ядре, а проверка доступа – в пространстве пользователя). MINIX3, драйверы находятся в пользовательском пространстве и посылают ядру специальные вызовы.
            \end{spacing}
            \end{tiny}
        \end{itemize}
    \end{footnotesize}
    \includegraphics[height=2.4cm, keepaspectratio]{OSClientServerModel.png}
    \end{itemize}
\end{frame}
\end{document}