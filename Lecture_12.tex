\documentclass[aspectratio=169,14pt]{beamer}

\usepackage[utf8]{inputenc}
\usepackage[main=russian,english]{babel}
\usepackage[T1]{fontenc}
\usepackage[labelformat=empty]{caption}
\usepackage{listings}
\usepackage{multimedia}
\usepackage{xcolor}
% \usepackage{hyperref}
\usepackage{setspace}
\usepackage{verbatim}
% \usepackage{fancyvrb}
% \DefineShortVerb{\|} % определение | для инлайн-кода

\usepackage{multirow}

\usetheme{Singapore}

\definecolor{urlcolor}{HTML}{799B03} % цвет гиперссылок
\hypersetup{urlcolor=urlcolor, colorlinks=true}

\graphicspath{{../Images/}}

\title{Файловые системы}

\begin{document}

% Слайд Файловые системы
\begin{frame}{Файловые системы}
    \begin{itemize}
        \item Необходимо постоянное хранение данных
        \item Используется внешняя память
        \item Хранение данных организовано в виде файлов; часть операционной
        системы, управляющая файлами, называется файловой системой
        \item Файл~--- абстрактный механизм, именованный набор данных.
        Представляет собой способ сохранения информации на диске и её
        считывание. От пользователя скрываются детали: способ и место
        хранения информации, детали работы дисков и т.п.
    \end{itemize}
\end{frame}

% Слайд Файлы
\begin{frame}{Файлы}
    \begin{footnotesize}
        \begin{itemize}
            \item Именование файлов
            \begin{itemize}
                \begin{tiny}
                \item при создании файла ему дается имя, правила именования
                (длина имени, используемые символы, зависимость от регистра
                и т.п.) зависят от системы
                \item часто имя состоит из двух частей, разделенных точкой:
                собственно имя и расширения, которое позволяет пользователю
                быстро определить категорию содержимого файла (тип файла:
                исполняемый, изображение, музыка, программа на определенном
                языке и т.п.)
                \end{tiny}
            \end{itemize}
            \item Доступ к файлам:
            \begin{itemize}
                \begin{tiny}
                \item последовательный (чтение и запись только по порядку от
                начала к концу, поддерживалась <<перемотка>>)
                \item произвольный (байты файла могут быть прочитаны в
                произвольном порядке)
                \end{tiny}
            \end{itemize}
            \item Атрибуты файлов (метаданные): дополнительная информация
            (кроме имени), описывающая свойства файлов
            \begin{itemize}
            \begin{tiny}
                \item защита (кто и какие операции может выполнять)
                \item флаги: только чтение, скрытый, системный, \ldots
                \item дата и время создания
                \item время последнего изменения
                \item текущий размер
                \item максимальный размер
                \item \ldots
            \end{tiny}
            \end{itemize}
        \end{itemize}
    \end{footnotesize}
\end{frame}

% Слайд Операции с файлами
\begin{frame}{Операции с файлами}
    \begin{footnotesize}
    \begin{itemize}
        \item Создание/удаление
        \item Открытие/закрытие (считывание в атрибутов файла, создание связи с процессом – файловый дескриптор/сохранение данных из кэша, удаление связи с процессом)
        \item Чтение/запись (осуществляется с текущей позиции, которая хранится в файловом дескрипторе)
        \item Позиционирование в файле (изменение текущей позиции)
        \item Получение/установка атрибутов файла
        \item Переименование
        \item Блокирование
        \item Операции с файлами реализуются в виде системных вызовов
    \end{itemize}
    \end{footnotesize}
\end{frame}

% Слайд Каталоги
\begin{frame}{Каталоги}
    \begin{itemize}
        \item Каталог (директория, папка) – способ организации и упорядочения файлов. Зачастую сами являются файлами
        \item Простейшая форма – один каталог, в котором расположены все файлы – корневой каталог
        \item Двухуровневая система – корневой каталог и каталоги для каждого пользователя
        \item Иерархическая – каталоги организованы в виде дерева
        \item Указание расположения файла в каталоге/каталогах:
        \begin{itemize}
            \item абсолютное (от корневого каталога): /usr/src/app/test.c c:\textbackslash Users\textbackslash PC\textbackslash Desktop\textbackslash test.c
            \item относительное (от текущего каталога): .. – родительский каталог, . – текущий каталог
        \end{itemize}
    \end{itemize}
\end{frame}

% Слайд Операции с каталогами
\begin{frame}{Операции с каталогами}
    \begin{itemize}
        \item Создание/удаление
        \item Открытие/закрытие (для выполнения последующей операции чтения каталога)
        \item Чтение следующей записи в каталоге
        \item Переименование
        \item Связывание файлов
        \item Удаление связи файлов
        \item \ldots
    \end{itemize}
\end{frame}

% Слайд Реализация файлов
\begin{frame}{Реализация файлов}
    \begin{footnotesize}
    \begin{spacing}{0.8}
        \begin{itemize}
            \item Главная задача файловой системы~--- обеспечить соответствие
            между файлом (который идентифицируется своим именем) и набором
            секторов на диске, в которых хранятся данные
            \item Простейшая схема~--- файл занимает набор смежных секторов
            \begin{itemize}
                \begin{footnotesize}
                \item легко реализуется, быстро работает
                \item диск дефрагментируется, нет возможности расширения файла
                \item CD-ROM, DVD
                \end{footnotesize}
            \end{itemize}
            \item Связанные списки~--- первое слово каждого блока является
            указателем на следующий блок
            \begin{itemize}
                \begin{footnotesize}
                \item достаточно хранить адрес начального блока
                \item медленный произвольный доступ (необходимо прочитать
                данные с начала файла)
                \end{footnotesize}
            \end{itemize}
            \includegraphics[height=2.2cm, keepaspectratio]{OSFilesLinkedList.png}
        \end{itemize}
    \end{spacing}
    \end{footnotesize}
\end{frame}

% Слайд Реализация файлов - FAT
\begin{frame}{Реализация файлов - FAT}
    \footnotesize{Связные списки с индексацией (указатели на следующие блоки
    хранятся в отдельной таблице, эта таблица загружается в
    память)~--- таблица размещения файлов (File Allocation Table, FAT)}
    \begin{columns}
        \begin{column}{0.6\textwidth}
            \begin{spacing}{0.8}
            \begin{tiny}
                \begin{itemize}
                        \item FAT содержит столько элементов, сколько есть
                        физических секторов на диске (блоков или кластеров
                        в случае объединения нескольких последовательных секторов)
                        \item С каждым файлом связан начальный блок хранения
                        данных на диске, номер которого одновременно служит
                        индексом в таблице FAT для определения следующего блока
                        \item В примере: файл A хранится в блоках 4,7,2,10,12;
                        файл B хранится в блоках 6,3,11,14
                        \item Специальные значения в ячейках FAT могут указывать
                        на окончание файла (последний блок в цепочке) или
                        определять, свободен ли блок
                        \item Необходимо хранить в памяти всю таблицу,
                        размер ее может быть большой
                \end{itemize}
            \end{tiny}
            \end{spacing}
        \end{column}
        \begin{column}{0.4\textwidth}
            \includegraphics[height=5cm, keepaspectratio]{OSFilesFAT.png}
        \end{column}
    \end{columns}
\end{frame}

% Слайд Реализация файлов – индексные узлы (i-узлы)
\begin{frame}{Реализация файлов – индексные узлы (i-узлы)}
    \begin{footnotesize}
        \begin{itemize}
            \item С каждым файлом связывается структура данных, называемая
            индексным узлом
            \item Эта структура содержит атрибуты и адреса блоков файла
            \item Для решения проблемы роста файла последний дисковый адрес
            резервируется не для блока данных, а для адреса косвенного блока
            \includegraphics[height=5.2cm, keepaspectratio]{OSFilesInodes.png}
        \end{itemize}
    \end{footnotesize}
\end{frame}

% Слайд Реализация каталогов
\begin{frame}{Реализация каталогов}
    \begin{itemize}
        \item Каталог состоит из списка элементов, по одному на файл;
        содержит имя файла, атрибуты, и
        \begin{itemize}
            \item дисковый адрес начала файла (FAT)
            \item номер первого блока (связный список)
            \item номер i-узла
        \end{itemize}
        \item Атрибуты можно хранить в i-узлах
    \end{itemize}
\end{frame}

% Слайд Производительность
\begin{frame}{Производительность}
    \begin{itemize}
        \item Кэширование~--- набор блоков диска хранится в памяти. При
        запросе блока происходит проверка его наличия в кэше. Размер кэша
        ограничен, используются алгоритмы замещения блоков, аналогичные
        виртуальной памяти.
        \item Опережающее чтение блока~--- последовательное считывание блоков
        до того, как они понадобятся.
        \item Снижение времени перемещения блока головок:
        \begin{itemize}
            \item помещение блоков, к которым высока вероятность обращения
            близко друг к другу, на один цилиндр
            \item помещение i-узлов не в начале диска, а перед каждым файлом
            \item помещение i-узлов в середине диска
        \end{itemize}
    \end{itemize}
\end{frame}

% Слайд Примеры файловых систем: CP/M
\begin{frame}{Примеры файловых систем: CP/M}
    \begin{footnotesize}
        \begin{itemize}
            \begin{spacing}{0.8}
            \item CP/M~--- операционная система, изначально работала на
            процессорах Intel 8080, Z80
            \item Один каталог. Файловая запись~--- набор данных, описывает
            один файл
            \item CP/M v1.4 (для дисков 250К, 77~дорожек, 26~секторов по 128~байт
            на дорожку, 2 резервных дорожки, два 1К блока для файловых записей,
            двести сорок 1К блоков данных)
            \begin{itemize}
                \begin{footnotesize}
                \begin{spacing}{0.8}
                \item статус файла (SS, 1 байт, существует, удален, скрыт)
                \item имя файла (Fn, 8 байт)
                \item расширение (Tn, 3 байта)
                \item номер файловой записи (EX, 1 байт, для файлов более 16К)
                \item резерв (Sn, 2 байта)
                \item количество записей (RC, 1 байт), занятых файлом в данной
                файловой записи, одна запись~--- 128 байт, служит для указания
                размера файла (с округлением до 128~байт).
                % Какое число хранится в поле для полностью занятой записи?
                \item размещение файла (AL, 16 байт, список номеров блоков)
                \end{spacing}
                \end{footnotesize}
                \includegraphics[height=0.9cm, keepaspectratio]{OSFilesCPM14DirectoryEntry.png}
            \end{itemize}
        \end{spacing}
        \end{itemize}
    \end{footnotesize}
\end{frame}

% Слайд Примеры файловых систем: MSDOS
\begin{frame}{Примеры файловых систем: MSDOS}
    \begin{itemize}
        \item MSDOS~--- изначально 16-битный клон CP/M c FAT12
        \item Иерархическая файловая система (MS-DOS 2.0)
        \item Смежные сектора диска логически объединяются в кластеры, файл
        хранится в кластерах
        \item Обязательно есть корневой каталог, размещается на диске
        непосредственно после таблицы FAT (FAT12/FAT16)
        \item Пустые кластера в FAT помечаются 0, последний кластер файла
        помечается меткой EOC (End Of Clusterchain) (FAT12: >=0x0FF8, 0x0FFF;
        FAT16: >=0xFFF8;\ldots), кластер с поврежденным блоком - 0x0FF7
    \end{itemize}
\end{frame}

% Слайд FAT32: файловая запись
\begin{frame}{FAT32: файловая запись}
    \begin{footnotesize}
        \begin{itemize}
            \begin{spacing}{0.8}
            \item Структура файловой записи FAT32
            \begin{itemize}
                \begin{tiny}
                \item DIR\_Name~--- имя файла (8.3)
                \item DIR\_Attr~--- атрибуты файла (каталог, только чтение,
                скрытый, системный, архивный, \ldots)
                \item DIR\_CrtTimeTenth, DIR\_CrtTime, DIR\_CrtDate,
                DIR\_LstAccDate, DIR\_WrtDate~--- метки времени и даты
                \item DIR\_FstClustHI, DIR\_FstClustLO~--- первый и второй байт
                номера кластера, где начинается содержимое файла (одновременно
                индекс в FAT для номера второго кластера)
                \item DIR\_FileSize~--- размер файла
                \end{tiny}
            \end{itemize}
            \item Создание каталога
            \begin{itemize}
                \begin{tiny}
                \item DIR\_FileSize = 0
                \item Каталогу отводится один кластер данных (корневой каталог
                в FAT12/FAT16 занимает фиксированное количество секторов)
                \item Создаются две записи: . и .. для текущего и родительского каталога
                \end{tiny}
            \end{itemize}
            \end{spacing}
            \includegraphics[height=1.6cm, keepaspectratio]{OSFilesFAT32FileEntry.png}
        \end{itemize}
    \end{footnotesize}
\end{frame}

% Слайд Примеры файловых систем: NTFS
\begin{frame}{Примеры файловых систем: NTFS}
    \begin{itemize}
        \item NTFS (New Technology File System)~--- файловая система для
        семейства операционных систем Winndows NT
        \item Информация о файлах хранится в MFT (Master File Table)~--- главная
        файловая таблица
        \item Поддерживает разграничение доступа к данным для пользователей и
        групп пользователей
        \item Позволяет назначать дисковые квоты для пользователей (3.0)
        \item Файлы хранятся в кластерах (512~байт~--- 64Кб)
        \includegraphics[height=2cm, keepaspectratio]{OSFilesNTFSPartition.png}
    \end{itemize}
\end{frame}

% Слайд NTFS, структура MFT
\begin{frame}{NTFS, структура MFT}
    \begin{columns}
        \begin{column}{0.65\textwidth}
            \begin{spacing}{0.8}
            \begin{footnotesize}
                \begin{itemize}
                    \item Полная структуризация: всё есть файл; MFT~--- файл
                    \item MFT представляет собой набор записей по 1Кб
                    \item Каждая запись описывает один файл, но один файл может
                    описываться более чем одной записью
                    \item Первые 16~записей (описывают метафайлы, размещаются
                    в корневом каталоге, начинаются с символа \$)~--- служебные
                    \begin{itemize}
                        \begin{footnotesize}
                            \item журнал (\$LogFile): данные для осуществления транзакции
                            \item файл тома (\$Volume): серийный номер, время
                            создания и т.п. для раздела (тома)
                            \item определения атрибутов (\$AttrDef): список
                            допустимых атрибутов в записях MFT
                            \item таблица преобразования регистра (\$UpCase):
                            для преобразования имен файлов/каталогов
                        \end{footnotesize}
                    \end{itemize}
                \end{itemize}
            \end{footnotesize}
            \end{spacing}
        \end{column}
        \begin{column}{0.35\textwidth}
            \includegraphics[height=5.2cm, keepaspectratio]{OSFilesNTFSMFT.png}
        \end{column}
    \end{columns}
\end{frame}

% Слайд NTFS, запись MFT
\begin{frame}{NTFS, запись MFT}
    \begin{footnotesize}
        \begin{itemize}
            \item Запись в MFT представляет собой совокупность атрибутов,
            каждый из которых хранится как отдельный поток (streams) байтов
            \item Атрибут состоит из тела и заголовка; бывают резидентными
            (значения хранятся в записи MFT) и нерезидентными
            \item Запись состоит из набора атрибутов
            \item Типовые атрибуты:
            \begin{itemize}
                \begin{footnotesize}
                \item \$STANDARD\_INFORMATION: стандартная информация о файле
                (время, права доступа)
                \item \$FILE\_NAME: полное имя файла
                \item \$SECURITY\_DESCRIPTOR: дескриптор безопасности и списки
                прав доступа (ACL)
                \item \$DATA: основные данные файла
                \item \$BITMAP: карта свободного пространства
                \item \$ATTRIBUTE\_LIST: расположение дополнительных записей MFT,
                когда потоки не умещаются в одну запись MFT
                \end{footnotesize}
        \end{itemize}
        \end{itemize}
    \end{footnotesize}
\end{frame}

% Слайд NTFS, запись MFT: примеры
\begin{frame}{NTFS, запись MFT: примеры}
    \begin{columns}
        \begin{column}{0.2\textwidth}
            Запись для файла
        \end{column}
        \begin{column}{0.8\textwidth}
            \includegraphics[height=3.8cm, keepaspectratio]{OSFilesNTFSMFTEntry_1.png}
        \end{column}
    \end{columns}
    \begin{columns}
        \begin{column}{0.2\textwidth}
            Запись для каталога
        \end{column}
        \begin{column}{0.8\textwidth}
            \includegraphics[height=3.8cm, keepaspectratio]{OSFilesNTFSMFTEntry_2.png}
        \end{column}
    \end{columns}
\end{frame}

% Слайд Ext2 – Extended Filesystem
\begin{frame}{Ext2 – Extended Filesystem}
    \begin{itemize}
        \item При создании файловой системы можно изменить размер блока
        \item Объединяет блоки диска в группы, каждая группа включает блоки
        данных и индексные дескрипторы (для ускорения доступа к данным)
        \item Заранее выделяет блоки данных для файла
        \item Улучшенные алгоритмы обновления (изменения, удаления) файлов~---
        минимизация сбоев системы
        \item Поддержка автоматической проверки непротиворечивости при загрузке
    \end{itemize}
\end{frame}

% Слайд Ext2
\begin{frame}{Ext2}
    Раздел разбит на группы блоков, каждый из которых содержит один из
    элементов информации:
    \begin{footnotesize}
        \begin{itemize}
            \item копия суперблока файловой системы (обновляется при
            проверке файловой системы)
            \item копия дескрипторов групп блоков (обновляется при проверке
            файловой системы)
            \item битовая карта блоков данных
            \item битовая карта индексных дескрипторов
            \item данные файлов
        \end{itemize}
        \includegraphics[height=3cm, keepaspectratio]{OSFilesExt2PartitionCrop.png}
    \end{footnotesize}
\end{frame}

% Слайд Ext2: таблица индексных дескрипторов
\begin{frame}{Ext2: таблица индексных дескрипторов}
    Таблица индексных дескрипторов состоит из ряда блоков, каждый из которых
    содержит определенное количество дескрипторов (по 128 байт)
    \begin{footnotesize}
        \begin{itemize}
            \begin{spacing}{0.8}
            \item тип файла и права доступа
            \item ИД владельца
            \item длина файла
            \item время обращения, изменения, удаления файла
            \item количество блоков данных в файле
            \item флаги файла
            \item указатели на блоки данных (массив  из указателей на
            логические номера блоков, обычно 15); первые 12~--- содержат
            логические номера, 12,13 и 15 элементы~--- косвенная адресация;
            логический номер блока занимает 4 байта
            \item список доступа к файлу (указывает на блок, содержащий
            расширенные атрибуты)
            \item список доступа к каталогу
            \item адрес фрагмента
            \item \ldots
            \end{spacing}
    \end{itemize}
    \end{footnotesize}
\end{frame}

% Слайд Ext2: i-блоки
\begin{frame}{Ext2: i-блоки}
    \includegraphics[height=7cm, keepaspectratio]{OSFilesExt2IBlock.png}
\end{frame}

% Слайд Ext2: каталоги
\begin{frame}{Ext2: каталоги}
    \begin{itemize}
        \item Каталог~--- cпециальный тип файла, содержат имена и номера
        индексных дескрипторов
        \item Элементы каталога имеют переменную длину и содержат:
        \begin{itemize}
            \item номер индексного дескриптора
            \item длина элемента каталога
            \item длина имени файла
            \item тип файла
            \item имя файла
        \end{itemize}
        \includegraphics[height=2.8cm, keepaspectratio]{OSFilesExt2Directory.png}
    \end{itemize}
\end{frame}

% Слайд Ext3
\begin{frame}{Ext3}
    \begin{itemize}
        \item Совместимость с Ext2
        \item Журналирование: каждое изменение выполняется за два шага
        \begin{itemize}
            \item вначале копии всех блоков сохраняются в журнале
            \item затем блоки сохраняются в файловой системе
            \item если сбой произошел до фиксирования в журнале~--- выдается ошибка
            \item если сбой происходит после фиксирования в журнале~--- повторная запись
        \end{itemize}
    \end{itemize}
\end{frame}

% Ext4
\begin{frame}{Ext4}
    \begin{itemize}
        \item 48 битная адресация блоков (файл до 16ТB, файловая система до 1EB)
        \item Неограниченное число подкаталогов
        \item Вместо отображения блоков используется отображение <<экстентов>>~---
        набора последовательных блоков
        \item Изменяемый размер i-узла (по-умолчанию~--- 256~байт, хранение
        времени с повышенной точностью, доп. аттрибутов непосредственно в i-yзле)
        \item Усовершенствования при выделении блоков
    \end{itemize}

\end{frame}
\end{document}