\documentclass[aspectratio=169,14pt]{beamer}

\usepackage[utf8]{inputenc}
\usepackage[main=russian,english]{babel}
\usepackage[T1]{fontenc}
\usepackage[labelformat=empty]{caption}
\usepackage{multimedia}
\usepackage{xcolor}
\usepackage{hyperref}
\usepackage{setspace}

\usetheme{Singapore}

\definecolor{urlcolor}{HTML}{799B03} % цвет гиперссылок
\hypersetup{urlcolor=urlcolor, colorlinks=true}

\graphicspath{{../Images/}}

\title{Цифровой логический уровень}

\begin{document}

% Слайд цифровая схема, вентиль
\begin{frame}{Цифровой логический уровень}
    \begin{itemize}
        \pause
        \item Цифровая схема~--- схема, в которой есть только два логических значения сигнала.
        \pause
        \begin{columns}[T,onlytextwidth]
            \begin{column}{0.5\textwidth}
                \begin{figure}[htp]
                    \centering
                    \includegraphics[width=4cm,keepaspectratio]{AnalogSignal.png}
                    \caption{\tiny{аналоговый сигнал}}
                    \label{fig:AnalogSignal}
                \end{figure}
            \end{column}
            \begin{column}{0.5\textwidth}
                \begin{figure}[htp]
                    \centering
                    \includegraphics[width=4cm, keepaspectratio]{DigitalSignal.png}
                    \caption{\tiny{цифровой сигнал}}
                    \label{fig:DigitalSignal}
                \end{figure}
            \end{column}
        \end{columns}
            \pause
        \item Вентиль~--- электронное устройство, которое может вычислять различные функции от двузначных сигналов, преобразуя соответствующим образом входные сигналы в выходной логический сигнал.
    \end{itemize}
\end{frame}

% Слайд вентиль
\begin{frame}{Цифровой логический уровень}
    \begin{itemize}
        \pause
        \item Вентиль на базе электромагнитного реле
        \pause
        \begin{spacing}{0.8}
        \item
        \begin{small}
        Под действием силы, которая возникает в катушке при появлении
        сигнала на входе, верхняя пластина смещается и замыкает выходную цепь.
        Входное напряжение может быть малым, а выходное~--- большим.
        \end{small}
        \item
        \begin{small}
        Если в качестве выхода использовать верхний контакт, то получим
        двухпозиционное реле (double-throw). У него два электрически
        противоположных выхода: когда на одном есть напряжение, на другом
        его нет.
        \end{small}
        \end{spacing}
        \pause
        \begin{columns}[T,onlytextwidth]
            \begin{column}{0.5\textwidth}
                \begin{figure}[htp]
                    \centering
                    \includegraphics[height=3.2cm,keepaspectratio]{ReleOff.png}
                    \caption{\tiny{реле разомкнуто, на выходе сигнала нет}}
                    \label{fig:ReleOff}
                \end{figure}
            \end{column}
            \begin{column}{0.5\textwidth}
                \begin{figure}[htp]
                    \centering
                    \includegraphics[height=3.2cm, keepaspectratio]{ReleOn.png}
                    \caption{\tiny{реле замкнуто, на выходе есть сигнал}}
                    \label{fig:ReleOn}
                \end{figure}
            \end{column}
        \end{columns}
            \pause
        \item Вентиль~--- электронное устройство, которое может вычислять различные функции от двузначных сигналов, преобразуя соответствующим образом входные сигналы в выходной логический сигнал.
    \end{itemize}
\end{frame}

% Слайд вентиль И
\begin{frame}{Цифровой логический уровень}
    \begin{columns}[T,onlytextwidth]
        \begin{column}{0.5\textwidth}
            \begin{itemize}
                \item Вентиль <<И>> (AND)
                \item Реализуется при подключении двух реле последовательно
                \item Вычисляет элементарную логическую функцию <<И>>
            \end{itemize}
        \end{column}
        \begin{column}{0.5\textwidth}
            \begin{figure}[htp]
                \centering
                \includegraphics[height=3.5cm, keepaspectratio]{ReleDTSerial.png}
                \caption{\tiny{последовательное подключение двух реле}}
                \label{fig:ReleDTSerial}
            \end{figure}
        \end{column}
    \end{columns}
    \begin{figure}[htp]
        \centering
        \includegraphics[height=1.2cm,keepaspectratio]{GateAND.png}
        \caption{\tiny{изображение вентиля <<И>> на схеме}}
        \label{fig:GateAND}
    \end{figure}
\end{frame}

% Слайд вентиль ИЛИ
\begin{frame}{Цифровой логический уровень}
    \begin{columns}[T,onlytextwidth]
        \begin{column}{0.5\textwidth}
            \begin{itemize}
                \item Вентиль <<ИЛИ>> (OR)
                \item Реализуется при подключении двух реле параллельно
                \item Вычисляет элементарную логическую функцию <<ИЛИ>>
            \end{itemize}
        \end{column}
        \begin{column}{0.5\textwidth}
            \begin{figure}[htp]
                \centering
                \includegraphics[height=3.5cm, keepaspectratio]{ReleDTParallel.png}
                \caption{\tiny{последовательное подключение двух реле}}
                \label{fig:ReleDTParallel}
            \end{figure}
        \end{column}
    \end{columns}
    \begin{figure}[htp]
        \centering
        \includegraphics[height=1.2cm,keepaspectratio]{GateOR.png}
        \caption{\tiny{изображение вентиля <<ИЛИ>> на схеме}}
        \label{fig:GateOR}
    \end{figure}
\end{frame}

% Слайд Инвертор, ИЛИ-НЕ, И-НЕ, исключающее ИЛИ
\begin{frame}{Цифровой логический уровень}
    \begin{columns}[T,onlytextwidth]
        \begin{column}{0.5\textwidth}
            \begin{itemize}
                \item Одно двухпозиционное реле является инвертором, который меняет сигнал на противоположный
                \item Вентиль <<ИЛИ-НЕ>> (NOR)
                \item Вентиль <<И-НЕ>> (NAND)
                \item Вентиль <<Исключающее ИЛИ>> (Exclusive OR)
            \end{itemize}
        \end{column}
        \begin{column}{0.5\textwidth}
            \begin{figure}[htp]
                \vspace{-25pt}
                \centering
                \includegraphics[height=0.8cm, keepaspectratio]{Inverter.png}
                \captionsetup{skip=0pt}
                \caption{\tiny{инвертор}}
                \label{fig:Inverter}
            \end{figure}
            \begin{figure}[htp]
                \vspace{-25pt}
                \centering
                \includegraphics[height=0.8cm, keepaspectratio]{GateNOR.png}
                \caption{\tiny{<<ИЛИ-НЕ>>}}
                \label{fig:GateNOR}
            \end{figure}
            \begin{figure}[htp]
                \vspace{-25pt}
                \centering
                \includegraphics[height=0.8cm, keepaspectratio]{GateNAND.png}
                \caption{\tiny{<<И-НЕ>>}}
                \label{fig:GateNAND}
            \end{figure}
            \begin{figure}[htp]
                \vspace{-25pt}
                \centering
                \includegraphics[height=0.8cm, keepaspectratio]{GateExclOR.png}
                \caption{\tiny{<<Исключающее ИЛИ>>}}
                \label{fig:GateExclOR}
            \end{figure}
        \end{column}
    \end{columns}
\end{frame}

% Слайд Шеннон
\begin{frame}{Цифровой логический уровень}
    \begin{itemize}
        \item Впервые идею об использовании электрических устройств для вычисления
        логических выражений предложил Клод Шеннон (1937~год)
        \item Таблицы истинности вентилей
        \begin{small}
        \begin{columns}[T,onlytextwidth]
            \begin{column}{0.3\textwidth}
                \begin{table}
                    \centering
                    \begin{tabular}{c|cc}
                        \textbf{<<И>>} & 0 & 1 \\ \hline
                        0 & 0 & 0 \\
                        1 & 0 & 1
                    \end{tabular}
                \end{table}
            \end{column}
            \begin{column}{0.3\textwidth}
                \begin{table}
                    \centering
                    \begin{tabular}{c|cc}
                        \textbf{<<ИЛИ>>} & 0 & 1 \\ \hline
                        0 & 0 & 1 \\
                        1 & 1 & 1
                    \end{tabular}
                \end{table}
            \end{column}
            \begin{column}{0.3\textwidth}
                \begin{table}
                    \centering
                    \begin{tabular}{c|cc}
                        \textbf{<<ИЛИ-НЕ>>} & 0 & 1 \\ \hline
                        0 & 1 & 0 \\
                        1 & 0 & 0
                    \end{tabular}
                \end{table}
            \end{column}
        \end{columns}
        \begin{columns}[T,onlytextwidth]
            \begin{column}{0.5\textwidth}
                \begin{table}
                    \centering
                    \begin{tabular}{c|cc}
                        \textbf{<<И-НЕ>>} & 0 & 1 \\ \hline
                        0 & 1 & 1 \\
                        1 & 1 & 0
                    \end{tabular}
                \end{table}
            \end{column}
            \begin{column}{0.5\textwidth}
                \begin{table}
                    \centering
                    \begin{tabular}{c|cc}
                        \textbf{<<Исключающее или>>} & 0 & 1 \\ \hline
                        0 & 0 & 1 \\
                        1 & 1 & 0
                    \end{tabular}
                \end{table}
            \end{column}
        \end{columns}
        \end{small}
    \end{itemize}
\end{frame}

% Слайд Суммирование
\begin{frame}{Цифровой логический уровень}
    \begin{itemize}
        \item Суммирование
        \begin{columns}[T,onlytextwidth]
            \begin{column}{0.5\textwidth}
                \begin{table}
                    \centering
                    \begin{tabular}{c|cc}
                        \textbf{<<+>>} & 0 & 1 \\ \hline
                        0 & 0 & 1 \\
                        1 & 1 & 0
                    \end{tabular}
                    \caption{\tiny{Таблица истинности для сложения однобитных чисел}}
                \end{table}
            \end{column}
            \begin{column}{0.5\textwidth}
                \begin{figure}[htp]
                    \centering
                    \includegraphics[height=0.8cm, keepaspectratio]{GateExclOR.png}
                    \caption{\tiny{Реализуется вентилем <<Исключающее ИЛИ>>}}
                \end{figure}
            \end{column}
        \end{columns}

        \item Перенос
        \begin{columns}[T,onlytextwidth]
            \begin{column}{0.5\textwidth}
                \begin{table}
                    \centering
                    \begin{tabular}{c|cc}
                        \textbf{<<+>>} & 0 & 1 \\ \hline
                        0 & 0 & 0 \\
                        1 & 0 & 1
                    \end{tabular}
                    \caption{\tiny{Таблица истинности для операции переноса}}
                \end{table}
            \end{column}
            \begin{column}{0.5\textwidth}
                \begin{figure}[htp]
                    \centering
                    \includegraphics[height=0.8cm, keepaspectratio]{GateAND.png}
                    \caption{\tiny{Реализуется вентилем <<И>>}}
                \end{figure}
            \end{column}
        \end{columns}
    \end{itemize}
\end{frame}

% Слайд полусумматор
\begin{frame}{Цифровой логический уровень}
    \begin{itemize}
        \item Объединив два вентиля, получим сумматор одноразрядных двоичных чисел. Такая схема называется
        полусумматором, так как не может использоваться для сложения многоразрядных чисел потому
        что не учитывает возможный перенос от предыдущего разряда.
        \begin{columns}[T,onlytextwidth]
            \begin{column}{0.5\textwidth}
                \begin{figure}[htp]
                    \centering
                    \includegraphics[height=2.2cm, keepaspectratio]{HalfAdder1bit.png}
                    \caption{\tiny{Полусумматор на основе вентилей <<Исключающее ИЛИ>> и <<И>> }}
                \end{figure}
            \end{column}
            \begin{column}{0.5\textwidth}
                \begin{figure}[htp]
                    \centering
                    \includegraphics[height=2.2cm, keepaspectratio]{HalfAdder1bitQuad.png}
                    \caption{\tiny{Схематичное изображение полусумматора}}
                \end{figure}
            \end{column}
        \end{columns}
    \end{itemize}
\end{frame}

% Слайд сумматор
\begin{frame}{Цифровой логический уровень}
    \begin{itemize}
        \item Сумматор~--- объединение двух полусумматоров.
        \begin{spacing}{0.8}
        \item Перенос разряда может возникнуть только в одном из полусумматоров.
        В этой связи для сложения переносов достаточно объединить
        соответствующие выходы <<П>> полусумматоров вентилем <<ИЛИ>>.
        \item Многобитный сумматор~--- объединение однобитных сумматоров
        \item На практике используют различные виды сумматоров с быстрым параллельным переносом
        \end{spacing}
        \begin{figure}[htp]
            \centering
            \includegraphics[height=2.1cm, keepaspectratio]{Adder1bitSchema.png}
            \caption{\tiny{Схема сумматора}}
        \end{figure}
    \end{itemize}
\end{frame}

% Слайд вычитание
\begin{frame}{Цифровой логический уровень}
    \begin{itemize}
        \item Вычитание
        \begin{itemize}
            \begin{footnotesize}
            \item $a-b=r$; разрядность $а = n; a > b$
            \item Для десятичных чисел: $a - b = a - b + 10^n - 10^n = a + (10^n - 1 - b) + 1 - 10^n$
            \item Собственно вычитание заменяется двумя сложениями и двумя вычитаниями без заимствования разряда
            \item $10^n - 1 - b$ называется дополнением до $9$, поскольку $10^n - 1$ состоит из девяток
            \item В случае двоичных чисел осуществляется дополнение до 1,
            что идентично инверсии числа (замены 0 на 1 и наоборот)
            \end{footnotesize}
        \end{itemize}
        \item Шаги вычитания
        \begin{itemize}
            \begin{footnotesize}
            \item Инверсия вычитаемого
            \item Сложение уменьшаемого и инверсии вычитаемого
            \item Добавление 1
            \item В условиях ограниченной разрядности вычитание степени разряда не требуется
            \end{footnotesize}
        \end{itemize}
    \end{itemize}
\end{frame}

% Слайд отрицательные числа
\begin{frame}{Цифровой логический уровень}
    \begin{itemize}
        \item Представление отрицательных чисел в двоичном виде
        \begin{footnotesize}
            \begin{itemize}
                \item Для восьми бит: 0 = 0, 01 = 1, 10 = 2, \ldots, 01111111 = 127,
                10000000 = -128, 10000001 = -127, \ldots, 11111111 = -1
                \item Такой способ называется дополнением до 2 (дополнение до 1 и добавление 1)
                \item Старший значащий бит называется знаковым разрядом, равен 1 для отрицательных
                чисел и 0 для положительных
                \item При выполнении операций необходимо следить за переполнением
                или исчезновением разрядов
                \item Результат сложения положительных и отрицательных чисел неверен,
                если знаковые разряды операндов совпадают, а знаковый результат от них отличается
            \end{itemize}
        \end{footnotesize}
\end{itemize}
\end{frame}

% Слайд комбинационные схемы
\begin{frame}{Цифровой логический уровень}
    \begin{itemize}
        \item Комбинационные (комбинаторные) схемы
        \begin{itemize}
            \item При конструировании схем напрямую из вентилей будет возникать много вводов и выводов
            \item Комбинационные схемы~--- схемы, в которых выходные сигналы
            определяются текущими входными сигналами
            \item Комбинационные схемы используются в вычислительных цепях для формирования входных
            сигналов и для подготовки данных, которые подлежат сохранению
            \item Комбинационные схемы~--- это устройства без <<памяти>>, выходные сигналы зависят
            только от входных
        \end{itemize}
    \end{itemize}
\end{frame}

% Слайд мультиплексор
\begin{frame}{Цифровой логический уровень}
    \begin{columns}[T,onlytextwidth]
        \begin{column}{0.6\textwidth}
            \begin{itemize}
            \item Мультиплексор
                \begin{itemize}
                    \item Схема с $2^n$ входами, одним выходом и \textbf{n} линиями управления
                    \item Один из входов соединяется с выходом в зависимости от значения на линиях управления
                    \item Возможна реализация различных табличных функций
                    \item Преобразование параллельного кода в последовательный
                \end{itemize}
            \end{itemize}
        \end{column}
        \begin{column}{0.4\textwidth}
            \begin{figure}[htp]
                \centering
                \includegraphics[height=2.8cm, keepaspectratio]{MultiplexorScheme.png}
                \captionsetup{skip=-5pt}
                \caption{\tiny{Mультиплексор}}
            \end{figure}
            \begin{figure}
                \centering
                \includegraphics[height=2cm]{MuxMajorityFunction.png}
                \captionsetup{skip=-5pt}
                \caption{\tiny{Функция большинства}}
            \end{figure}
        \end{column}
    \end{columns}
\end{frame}

% Слайд декодер
\begin{frame}{Цифровой логический уровень}
    \begin{columns}[T,onlytextwidth]
        \begin{column}{0.6\textwidth}
            \begin{itemize}
            \item Декодер
                \begin{itemize}
                    \item Схема с \textbf{n} входами, $2^n$ выходами
                    \item Устанавливается в 1 выход, соответствующий числу (значению) на входе
                    \item Использование: выбор схемы выполнения в зависимости от команды
                \end{itemize}
            \end{itemize}
        \end{column}
        \begin{column}{0.4\textwidth}
            \begin{figure}[htp]
                \centering
                \includegraphics[height=3.6cm, keepaspectratio]{DecoderAction.png}
                \caption{\tiny{Работа декодера}}
            \end{figure}
        \end{column}
    \end{columns}
\end{frame}

% Слайд компаратор
\begin{frame}{Цифровой логический уровень}
    \begin{itemize}
        \item Компаратор
        \begin{itemize}
            \item Сравнивает два слова, поступившие на вход
        \end{itemize}
    \end{itemize}
    \begin{figure}[htp]
        \centering
        \includegraphics[height=5cm, keepaspectratio]{Comparer.png}
        \caption{\tiny{Компаратор}}
    \end{figure}
\end{frame}

% Слайд Арифметико---логическое устройство
\begin{frame}{Цифровой логический уровень}
    \begin{columns}[T,onlytextwidth]
        \begin{column}{0.5\textwidth}
            \begin{itemize}
            \item Арифметико---логическое устройство
                \begin{itemize}
                    \item Выполнение операций <<НЕ>> над одним словом; <<И>>, <<ИЛИ>>, сложение над двумя входными словами
                    \item INVA~--- инверсия А
                    \item ENA~--- разрешение А
                    \item ENB~--- разрешение B
                \end{itemize}
            \end{itemize}
        \end{column}
        \begin{column}{0.5\textwidth}
            \begin{figure}[htp]
                \centering
                \includegraphics[height=5cm, keepaspectratio]{ALU.png}
                \caption{\tiny{Арифметико---логическое устройство}}
            \end{figure}
        \end{column}
    \end{columns}
\end{frame}

% Слайд тактовый генератор
\begin{frame}{Цифровой логический уровень}
    \begin{itemize}
        \item Тактовый генератор вызывает серию одинаковых по длительности и частоте импульсов
            \begin{itemize}
                \begin{footnotesize}
                \item Используется для контроля времени выполнения операций
                \item Временной интервал между началом одного и началом другого
                импульса~--- такт (период)
                \item Начала такта~--- фронт, окончание такта~--- спад
                \item Синхронные и асинхронные (разное время высокого и низкого уровня сигнала) генераторы
                \end{footnotesize}
            \end{itemize}
            \begin{columns}[T,onlytextwidth]
                \begin{column}{0.66\textwidth}
                    \begin{figure}[htp]
                        \centering
                        \includegraphics[height=1.8cm, keepaspectratio]{ClockGenerator.png}
                        \caption{\tiny{Тактовый генератор и значение на выходе}}
                    \end{figure}
                \end{column}
                \begin{column}{0.33\textwidth}
                    \begin{figure}[htp]
                        \centering
                        \includegraphics[height=1.8cm, keepaspectratio]{ClockGeneratorAsync.png}
                        \caption{\tiny{Формирование асинхронных импульсов}}
                    \end{figure}
                \end{column}
            \end{columns}
    \end{itemize}
\end{frame}

% Слайд SR-защелка
\begin{frame}{Цифровой логический уровень}
    Защелка
    \begin{columns}[T,onlytextwidth]
        \begin{column}{0.5\textwidth}
            \begin{footnotesize}
                \begin{itemize}
                \item Необходима схема, которая запоминает <<предыдущее>> входное значение
                \item Основана на двух вентилях <<НЕ-ИЛИ>>
                \item SR~--- защелка: S~--- установка (setting), R~--- сброс (reset)
            \end{itemize}
            $S=0; R=0; Q=0 \Rightarrow \overline{Q} = 1 \Rightarrow Q=0$

            $S=0; R=0; Q=1 \Rightarrow \overline{Q} = 0 \Rightarrow Q=1$

            $S=1; R=0; Q=0 \Rightarrow \overline{Q} = 0 \Rightarrow Q=1$

            $S=1; R=0; Q=1 \Rightarrow \overline{Q} = 0 \Rightarrow Q=1$

            $S=0; R=1; Q=0 \Rightarrow \overline{Q} = 1 \Rightarrow Q=0$

        \end{footnotesize}
        \end{column}
        \begin{column}{0.5\textwidth}
            \begin{figure}[htp]
                \centering
                \includegraphics[height=2.5cm, keepaspectratio]{SRLatch.png}
                \caption{\tiny{SR~--- защелка}}
            \end{figure}
            \begin{table}
                \centering
                \begin{footnotesize}
                \begin{tabular}{c|cc}
                    \textbf{<<НЕ-ИЛИ>>} & 0 & 1 \\ \hline
                    0 & 1 & 0 \\
                    1 & 0 & 0
                \end{tabular}
                \end{footnotesize}
            \end{table}
        \end{column}
    \end{columns}
\end{frame}

% Слайд синхронная SR-защелка
\begin{frame}{Цифровой логический уровень}
    \begin{itemize}
        \item Синхронная $SR$~--- защелка
        \begin{itemize}
            \item Защелка, которая меняет состояние только в определенные моменты времени
            \item Меняет состояние только когда значение синхровхода = 1 (включение, стробирование)
        \end{itemize}
        \begin{figure}[htp]
            \centering
            \includegraphics[height=3.6cm, keepaspectratio]{SRLatchSync.png}
            \caption{\tiny{Синхронная $SR$~--- защелка}}
        \end{figure}
    \end{itemize}
\end{frame}

% Слайд проблема синхронной SR-защелки
\begin{frame}{Цифровой логический уровень}
    \begin{itemize}
        \item Проблема синхронной $SR$~--- защелки
        \begin{itemize}
            \item $S=1; R=1 \Rightarrow$ схема становится недетерминированной,
            когда $S$ и $R$ вернуться к 0
            \item Состояние выхода зависит от того, на каком из входов $S$ или $R$
            раньше появится 0
        \end{itemize}
    \end{itemize}
\end{frame}

% Слайд проблема D-защелка
\begin{frame}{Цифровой логический уровень}
    \begin{itemize}
        \item Синхронная $D$-защелка
        \begin{itemize}
            \item Единственный вход $D$
            \item Запоминает значение, когда есть сигнал на синхровходе
            \item Память объемом 1 бит
            \item Реализуется 11 транзисторами, возможны более сложные
            схемы с меньшим числом транзисторов
        \end{itemize}
        \begin{figure}[htp]
            \centering
            \includegraphics[height=3.4cm, keepaspectratio]{DSyncLatch.png}
            \caption{\tiny{Синхронная $D$~--- защелка}}
        \end{figure}
    \end{itemize}
\end{frame}

% Слайд проблема триггер
\begin{frame}{Цифровой логический уровень}
    \begin{itemize}
        \item Триггер (flip-flop)
        \begin{itemize}
            \item Для сохранения сигнала на входе в заданный момент времени
            \item Срабатывает, когда происходит переход сигнала с 0 на 1 (фронт)
            и наоборот (спад)
            \item Защелка запускается уровнем сигнала, триггер запускается
            перепадом сигнала
        \end{itemize}
        \begin{columns}[T,onlytextwidth]
            \begin{column}{0.4\textwidth}
                \begin{figure}[htp]
                    \centering
                    \includegraphics[height=2cm, keepaspectratio]{Trigger.png}
                    \caption{\tiny{Триггер}}
                \end{figure}
            \end{column}
            \begin{column}{0.6\textwidth}
                \begin{figure}[htp]
                    \centering
                    \includegraphics[width=3.4cm, keepaspectratio]{TriggerTiming.png}
                    \caption{\tiny{Срабатывание триггера}}
                \end{figure}
            \end{column}
        \end{columns}
    \end{itemize}
\end{frame}
\end{document}