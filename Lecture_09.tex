\documentclass[aspectratio=169,14pt]{beamer}

\usepackage[utf8]{inputenc}
\usepackage[main=russian,english]{babel}
\usepackage[T1]{fontenc}
\usepackage[labelformat=empty]{caption}
% \usepackage{listings}
\usepackage{multimedia}
\usepackage{xcolor}
% \usepackage{hyperref}
\usepackage{setspace}
\usepackage{verbatim}
\usepackage{multirow}

\usetheme{Singapore}

\definecolor{urlcolor}{HTML}{799B03} % цвет гиперссылок
\hypersetup{urlcolor=urlcolor, colorlinks=true}

\graphicspath{{../Images/}}

\title{История ОС. Основные понятия ОС}

\begin{document}

% Слайд Процессы
\begin{frame}{Процессы}
    \begin{itemize}
        \begin{footnotesize}
        \item Процесс~--- абстрактное понятие, описывающее выполнение
        программы процессором
        \item Процесс~--- выполняемая программа, включая значения
        регистров, адресное пространство, стек, открытые файлы и прочие ресурсы
        \item Многозадачность~--- возможность одновременного использования
        различных ресурсов машины (процессор, память, устройства ввода/вывода)
        разными приложениями
        \item В многопроцессорных (многоядерных) системах выполнение процессов
        действительно происходит одновременно
        \item В системах с одним процессором для реализации многозадачности
         в памяти должны находиться несколько процессов. В определённые
         моменты времени происходит переключение между ними, т.е. выбор
         одного из процессов и его выполнение
        \end{footnotesize}
    \end{itemize}
\end{frame}

% Слайд Многозадачность
\begin{frame}{Многозадачность}
    \begin{figure}[htp]
        \centering
        \includegraphics[height=7cm, keepaspectratio]{OSMultiTaskAdvantage.png}
    \end{figure}
\end{frame}

% Слайд Создание процесса
\begin{frame}{Создание процесса}
    \begin{itemize}
        \item При инициализации системы процесс создаётся загрузчиком
        путём загрузки кода (части операционной системы) и передачи
        ему управления
        \item Последующее создание процессов происходит с помощью
        системных вызовов, например, при:
        \begin{itemize}
            \item вводе команды в консоли (командой строке);
            \item двойном щелчке на иконке;
            \item непосредственном использовании системного вызова
            в коде.
        \end{itemize}
        \item Системный вызов \textbf{fork}~--- создание процесса,
        аналогичного родительскому
        \item Системный вызов \textbf{execv (execl)}~--- загрузка
        новой программы в память
    \end{itemize}
\end{frame}

% Слайд Завершение процесса
\begin{frame}{Завершение процесса}
    \begin{itemize}
        \item Способы завершения процесса
        \begin{itemize}
            \item Нормальное завершение, добровольное
            \item Завершение вследствие ошибки, добровольное
            \item Завершение вследствие фатальной ошибки,
            принудительное
            \item Уничтожение другим процессом, принудительное
        \end{itemize}
        \item Системные вызовы \textbf{exit}, \textbf{kill}
    \end{itemize}
\end{frame}

% Слайд Состояние процесса
\begin{frame}{Состояние процесса}
    \begin{itemize}
        \begin{footnotesize}
            \item Действие (процесс использует процессор)
            \item Готовность (процесс временно приостановлен, чтобы
            позволить выполняться другому процессу)
            \item Блокировка (процесс не может быть запущен прежде, чем
            произойдет какое-либо внешнее событие)
            \item Переход 1 осуществляется при выполнении операции ввода/вывода
            или при системном вызове (\textbf{sleep})
            \item Переход 4 осуществляется при возникновении внешнего события
            (окончание операции ввода/вывода, системный вызов \textbf{wakeup})
            \item Переходы 2, 3 выполняет планировщик~--- специальная подсистема
            внутри ОС
            \begin{figure}[htp]
                \centering
                \includegraphics[height=3cm, keepaspectratio]{OSProccessStateTypical.png}
            \end{figure}
        \end{footnotesize}
    \end{itemize}
\end{frame}

% Слайд Жизненный цикл процесса
\begin{frame}{Жизненный цикл процесса}
    \begin{itemize}
        \begin{footnotesize}
            \item В режиме ядра исполняется только код ядра, не процесса
            \item Zombie~--- завершившийся процесс, для которого процесс-родитель
            не может принять сигнал о завершении процесса)
        \end{footnotesize}
        \begin{figure}[htp]
            \centering
            \includegraphics[height=5.6cm, keepaspectratio]{OSProcessLifeTime.png}
        \end{figure}
    \end{itemize}
\end{frame}

% Слайд Реализация процессов
\begin{frame}{Реализация процессов}
    \begin{itemize}
        \begin{footnotesize}
            \item Таблица процессов~--- каждый элемент таблицы описывает
            отдельный процесс
            \item \textbf{MINIX3}~--- три раздела в таблице процессов для
            разных целей
        \end{footnotesize}
    \end{itemize}
    \begin{tiny}
        \begin{tabular}{p{3.6cm}p{4.4cm}p{4.4cm}}
            \hline
            \textbf{Ядро} & \textbf{Управление процессами} & \textbf{Управление файлами} \\ \hline
            Регистры & Указатель на текстовый сегмент & Маска UMASK \\
            Счетчик команд & Указатель на сегмент данных & Корневой каталог \\
            Слово состояния программы & Указатель на сегмент стека &  Рабочий каталог \\
            Указатель стека &  Статус завершения &  Дескрипторы файлов \\
            Состояние процесса & Состояние сигналов & Реальный идентификатор пользователя\\
            Текущий приоритет &  Идентификатор процесса & Эффективный идентификатор пользователя \\
            Максимальный приоритет &  Родительский процесс & Реальный идентификатор группы \\
            Оставшееся число тактов &  Группа процессов & Эффективный идентификатор группы \\
            Размер кванта &  Процессорное время дочернего процесса & Управление \\
            Использованное процессорное время & Реальный идентификатор пользователя (UID) & Область сохранения для чтения/записи \\
            Указатели очереди сообщений & Эффективный идентификатор пользователя & Параметры системного вызова \\
            Биты действующих сигналов &  Реальный идентификатор группы (GUID) & Различные битовые флаги \\
            Различные флаги & Эффективный идентификатор группы &  \\
            Имя процесса & Файловая информация о совместном использовании текста &  \\
            & Битовый маски для сигналов &  \\
            & Различные битовые флаги &  \\
            & Имя процесса &  \\
        \end{tabular}
    \end{tiny}
\end{frame}

% Слайд Переключение процессов
\begin{frame}{Переключение процессов}
    \begin{itemize}
        \begin{spacing}{0.8}
        \begin{footnotesize}
            \item Дескриптор процесса~--- информационная структура, в которой
            представлена внешняя (по отношению к процессу) информация
            \begin{itemize}
                \begin{tiny}
                \begin{spacing}{0.9}
                    \item Идентификатор процесса
                    \item Состояние процесса
                    \item Информация о привилегиях процесса
                \end{spacing}
            \end{tiny}
            \end{itemize}
            \item Контекст процесса~--- содержит информацию о внутреннем
            состоянии процесса и аппаратуры
            \begin{itemize}
                \begin{tiny}
                \begin{spacing}{1.0}
                    \item Содержимое регистров
                    \item Информация об открытых файлах
                    \item Информация о работе внешних устройств
                \end{spacing}
                \end{tiny}
            \end{itemize}
            \item Аппаратный контекст~--- набор данных, который должен быть
            загружен в регистры до возобновления работы процесса
            \item Linux
            \begin{itemize}
                \begin{tiny}
                \begin{spacing}{1.0}
                    \item Переключение глобального каталога страниц
                    с целью установки адресного пространства
                    \item Переключение стека режима ядра и аппаратного
                    контекста
                \end{spacing}
                    \begin{itemize}
                        \begin{tiny}
                        \begin{spacing}{1.0}
                            \item В ранних версиях: аппаратно, far jmp (переход
                            на селектор дескриптора сегмента нового процесса,
                            автоматически сохраняется старый аппаратный
                            контекст и загружается новый из сегмента
                            состояния задачи TSS
                            \item В версии 2.6: пошаговое переключение
                            выполняемое с помощью последовательности
                            инструкций mov, обеспечивается строгий контроль
                            допустимости загружаемых данных
                            \item Время переключения приблизительно одинаково
                        \end{spacing}
                        \end{tiny}
                    \end{itemize}
                \begin{spacing}{1.0}
                    \item Сохранение и загрузка регистров FPU, MMX, XMM
                \end{spacing}
                \end{tiny}
            \end{itemize}
        \end{footnotesize}
        \end{spacing}
    \end{itemize}
\end{frame}

% Слайд Потоки
\begin{frame}{Потоки}
    \begin{itemize}
        \item Переключение между процессами очень длительная задача
        \item Традиционно у каждого процесса есть адресное пространство и
        один поток управления
        \item Возможно добавление в процесс нескольких потоков управления
        \item Такие потоки управления называются программными потоками
        или легковесными процессами
    \end{itemize}
    \begin{figure}[htp]
        \centering
        \begin{tiny}
            \begin{tabular}{p{4cm}p{4cm}}
                \hline
                \textbf{Процесс} & \textbf{Поток} \\ \hline
                Адресное пространство & Счетчик команд \\
                Глобальные переменные & Регистры \\
                Открытые файлы & Стек \\
                Дочерние процессы &  Состояние \\
                Действующие аварийные сигналы & \\
                Сигналы и их обработчики &  \\
                Учетная информация &  \\
            \end{tabular}
        \end{tiny}
    \end{figure}
\end{frame}

% Слайд Межпроцессное взаимодействие
\begin{frame}{Межпроцессное взаимодействие}
    \begin{itemize}
        \begin{footnotesize}
            \item IPC~--- InterProcess Coomunication
            \item Аспекты проблемы
            \begin{itemize}
                \begin{footnotesize}
                \item Передача информации между процессами
                \item Контроль над деятельностью процессов (гарантия, что
                процессы не пересекутся в критических ситуация)
                \item Согласование действий процессов
                \end{footnotesize}
            \end{itemize}
            \item Для обмена информацией между процессам можно использовать
            общую память, файлы и т.п.
            \item Необходимо предусматривать взаимное исключение~--- запрет
            одновременной записи и чтения совместно используемых данных более
            чем одним процессом (если один процесс использует общие данные,
            другому процессу это запрещено)
            \item Критическая область (секция)~--- часть программы, в которой
            есть обращение к совместно используемым данным
            \item Задача: избежать нахождение двух процессов в критических
            областях
        \end{footnotesize}
    \end{itemize}
\end{frame}

% Слайд Взаимное исключение с активным ожиданием
\begin{frame}[containsverbatim]{Взаимное исключение с активным ожиданием}
    \begin{itemize}
        \item Запрещение прерываний (при входе в критическую область)
        \item Добавление переменных блокировки: возникает проблема атомарной
        проверки и изменения такой переменной
        \item Строгое чередование (переменная отслеживает, чья очередь войти
        в критическую область)
    \end{itemize}
    \begin{columns}
        \begin{column}{0.5\textwidth}
        \begin{footnotesize}
        \begin{verbatim}
    /* Процесс 1 */
    while(TRUE){
        while(turn!=0)
        critical_region();
        turn=1;
        noncritical_region();
    }
        \end{verbatim}
        \end{footnotesize}
        \end{column}
        \begin{column}{0.5\textwidth}
            \begin{footnotesize}
                \begin{verbatim}
    /* Процесс 2 */
    while(TRUE){
        while(turn!=1)
        critical_region();
        turn=0;
        noncritical_region();
    }
                \end{verbatim}
                \end{footnotesize}
                \end{column}
    \end{columns}
\end{frame}

% Слайд Алгоритм Петерсона
\begin{frame}[containsverbatim]{Алгоритм Петерсона}
    \begin{itemize}
        \item 1981, Гарри Петерсон существенно улучшил программный алгоритм
        контроля входа в критическую секцию
        \item Без строгого чередования
    \end{itemize}
    \begin{tiny}
        \begin{verbatim}
#define FALSE 0
#define TRUE 1
#define N 2                                                 // Количество процессов
int turn;                                                   // Чья очередь
int interested[N] = {0};                                    // Индикатор "интереса" критической областью

void enter_region(int process)
{
    int other;
    other = 1 - process;                                    // Номер другого процесса
    interested[process] = TRUE;
    turn = process;                                         // Занимаем очередь
    while (turn == process && interested[other] == TRUE);   // Пустой цикл
    /* вход в критическую секцию осуществлён */
}

void leave_region(int process)                              // Покидание критической секции
{
    interested[process] = FALSE;
}
        \end{verbatim}
        \end{tiny}
\end{frame}

% Слайд Алгоритм Петерсона, последовательный вход в секцию
\begin{frame}[containsverbatim]{Алгоритм Петерсона, последовательный вход в секцию}
    \begin{columns}
        \begin{column}{0.32\textwidth}
            \begin{tiny}
                \begin{verbatim}
#define FALSE 0
#define TRUE 1
#define N 2
int turn;
int interested[N] = {0};

void enter_region(int process)
{
    int other;
    other = 1 - process;
    interested[process] = TRUE;
    turn = process;
    while (turn == process && interested[other] == TRUE);
}

void leave_region(int process)
{
    interested[process] = FALSE;
}
                \end{verbatim}
            \end{tiny}
        \end{column}
        \begin{column}{0.68\textwidth}
            \begin{figure}[htp]
                \centering
                \begin{tiny}
                    \begin{tabular}{p{0.5cm}|p{2.4cm}|p{2.4cm}}
                        \hline
                        \textbf{Время} & \textbf{Поток 0} & \textbf{Поток 0} \\ \hline
                        $t_{1}$ & int other = 1; & \\ \hline
                        $t_{2}$ & interested[0] = TRUE; & \\ \hline
                        $t_{3}$ & turn = 0; &   \\ \hline
                        $t_{4}$ & while (turn == 0 \&\& interested[1]); & \\ \hline
                        $t_{5}$ & \multirow{5}{=}{Критическая секция} & int other = 0; \\ \cline{1-1}\cline{3-3}
                        $t_{6}$ & & interested[1] = TRUE \\ \cline{1-1}\cline{3-3}
                        $t_{7}$ & & turn = 1; \\ \cline{1-1}\cline{3-3}
                        $t_{8}$ & & while (turn == 1 \&\& interested[0]); \\ \cline{1-1}\cline{3-3}
                        $t_{9}$ & & while (turn == 1 \&\& interested[0]); \\ \hline
                        $t_{10}$ & interested[0] = FALSE & \\ \hline
                        $t_{11}$ & & \multirow{2}{=}{Критическая секция} \\ \cline{1-2}
                        $t_{12}$ & & \\ \hline
                        $t_{13}$ & & interested[1] = FALSE \\ \hline
                    \end{tabular}
                \end{tiny}
            \end{figure}
        \end{column}
    \end{columns}
\end{frame}


% Слайд Алгоритм Петерсона, попытка одновременного входа в секцию
\begin{frame}[containsverbatim]{Алгоритм Петерсона, попытка одновременного входа в секцию}
    \begin{columns}
        \begin{column}{0.32\textwidth}
            \begin{tiny}
                \begin{verbatim}
#define FALSE 0
#define TRUE 1
#define N 2
int turn;
int interested[N] = {0};

void enter_region(int process)
{
    int other;
    other = 1 - process;
    interested[process] = TRUE;
    turn = process;
    while (turn == process && interested[other] == TRUE);
}

void leave_region(int process)
{
    interested[process] = FALSE;
}
                \end{verbatim}
            \end{tiny}
        \end{column}
        \begin{column}{0.68\textwidth}
            \begin{figure}[htp]
                \centering
                \begin{tiny}
                    \begin{tabular}{p{0.5cm}|p{2.4cm}|p{2.4cm}}
                        \hline
                        \textbf{Время} & \textbf{Поток 0} & \textbf{Поток 0} \\ \hline
                        $t_{1}$ & int other = 1; & \\ \hline
                        $t_{2}$ & & int other = 0; \\ \hline
                        $t_{3}$ &  & interested[1] = TRUE;   \\ \hline
                        $t_{4}$ & interested[0] = TRUE; & \\ \hline
                        $t_{5}$ & turn = 0; & \\ \hline
                        $t_{6}$ & & turn = 1; \\ \hline
                        $t_{7}$ & & while (turn == 1 \&\& interested[0]); \\ \hline
                        $t_{8}$ & & while (turn == 1 \&\& interested[0]); \\ \hline
                        $t_{9}$ & while (turn == 0 \&\& interested[1]); & \\ \hline
                        $t_{10}$ & \multirow{3}{=}{Критическая секция} & while (turn == 1 \&\& interested[0]); \\ \cline{1-1}\cline{3-3}
                        $t_{11}$ & & while (turn == 1 \&\& interested[0]); \\ \cline{1-1}\cline{3-3}
                        $t_{12}$ & & while (turn == 1 \&\& interested[0]); \\ \hline
                        $t_{13}$ & interested[0] = FALSE; & \\ \hline
                        $t_{14}$ & & \multirow{2}{=}{Критическая секция} \\ \cline{1-2}
                        $t_{15}$ & & \\ \hline
                        $t_{16}$ & & interested[1] = FALSE \\ \hline
                    \end{tabular}
                \end{tiny}
            \end{figure}
        \end{column}
    \end{columns}
\end{frame}

% Слайд Аппаратная реализация взаимного исключения
\begin{frame}[containsverbatim]{Аппаратная реализация взаимного исключения}
    \begin{itemize}
        \begin{footnotesize}
        \item Реализуется с помощью команды \textbf{TSL} (Test and Set Lock)
        \item В регистр считывается значение ячейки памяти, а в ячейке
        сохраняется ненулевое число, регистр и адрес ячейки передаются
        в качестве параметров инструкции
        \item Атомарная операция, в время её выполнения не может
        произойти переключения контекста
        \item Пример:
        \begin{itemize}
            \begin{footnotesize}
            \item \textbf{REGISTER}~--- имя регистра для хранения промежуточного значения
            \item \textbf{LOCK}~--- ячейка памяти, общая для процессов, желающих
            войти в критическую секцию
            \end{footnotesize}
        \end{itemize}
        \end{footnotesize}
    \end{itemize}
    \begin{tiny}
        \begin{verbatim}
enter_region;
    TSL REGISTER, LOCK
    CMP REGISTER, #0
    JNE enter_region        // Если в ячейке было ненулевое значение,
                            // значит какой-то процесс заблокировал вход в критическую секцию
    RET

leave_region:
    MOVE LOCK, #0
    RET
        \end{verbatim}
    \end{tiny}
\end{frame}

% Слайд Аппаратная реализация взаимного исключения - спинлок
\begin{frame}[containsverbatim]{Аппаратная реализация взаимного исключения - спинлок}
    \begin{itemize}
        \begin{footnotesize}
            \item Spinlock (циклическая блокировка)~--- низкоуровневый
            примитив синхронизации, применяемый для реализации взаимного
            исключения
            \item Представляет собой переменную в памяти и реализуется
            на атомарных операциях, которые должны присутствовать в
            системе команд процессора
            \item Пример для x86: инструкция xchg производит обмен
            содержимого двух операндов
        \end{footnotesize}
    \end{itemize}
    \begin{tiny}
        \begin{verbatim}
mov eax, spinlock_address
mov ebx, SPINLOCK_BUSY

wait_cycle:
xchg [eax], ebx
cpm ebx, SPINLOCK_FREE
jnz wait_cycle

; < начало критической секции

mov eax, spinlock_address
mov ebx, SPINLOCK_FREE
xchg [eax], ebx
        \end{verbatim}
    \end{tiny}
\end{frame}

% Слайд Cемафоры
\begin{frame}{Cемафоры}
    \begin{spacing}{0.9}
    \begin{itemize}
        \begin{footnotesize}
            \item Алгоритмы активного ожидания:
            \begin{itemize}
            \begin{footnotesize}
                \item расходуют ресурсы процессора
                \item возможна инверсия приоритетов: задача с высоким
                приоритетом выполнения ждет освобождения критической
                секции у задачи с низким приоритетом
            \end{footnotesize}
            \end{itemize}
            \item Существуют примитивы, блокирующие процесс в случае
            запрета входа в критическую секцию
            \item \textbf{sleep}~--- системный вызов, блокирующий процесс
            \item \textbf{wakeup}~--- пробудить процесс
            \item 1965 год: Дейсктра предложил обобщение этих примитивов,
            предложив новый тип переменной <<семафор>>
            \item Значение семафора может быть нулем или положительным
            числом
            \item Над семафорами возможны две операции:
            \begin{itemize}
            \begin{footnotesize}
                \item \textbf{down}: сравнивает значение семафора с нулем.
                Если значение больше нуля, оно уменьшается на единицу,
                иначе процесс блокируется
                \item \textbf{up}: увеличивает значение семафора. Если с
                этим семафором связаны заблокированные (ожидающие) процессы,
                происходит пробуждение одного из них
            \end{footnotesize}
            \end{itemize}
            \item Операции \textbf{down} и \textbf{up} атомарны
        \end{footnotesize}
    \end{itemize}
    \end{spacing}
\end{frame}

% Слайд Мьютексы
\begin{frame}{Мьютексы}
    \begin{itemize}
    \begin{footnotesize}
        \item Мьютексы~--- упрощенная версия семафора
        \item Без подсчета, только для управления взаимными
        исключениями
        \item Мьютекс~--- переменная, которая может находиться в
        одном из двух состояний: заблокирована / свободна
        \item Над мьютексами возможны две операции:
        \begin{itemize}
            \begin{footnotesize}
            \item \textbf{mutex\_lock}: проверяет, заблокирован ли мьютекс
            и блокирует его, иначе процесс «засыпает»
            \item \textbf{mutex\_unlock}: разблокирует мьютекс и передает
            управление заблокированному процессу, связанному с этим
            мьютексом (если такой есть)
            \end{footnotesize}
        \end{itemize}
        \item Операции \textbf{mutex\_lock} и \textbf{mutex\_unlock} атомарны
        \item Используется для управления взаимными исключениями в
        потоках
    \end{footnotesize}
    \end{itemize}
\end{frame}

% Слайд Мониторы
\begin{frame}[containsverbatim]{Мониторы}
    \begin{spacing}{1}
    \begin{itemize}
        \begin{tiny}
        \item Использование примитивов~--- сложный процесс,
        возможно появление трудно обнаружимых ошибок
        \item Монитор~--- примитив синхронизации высокого уровня
        \item Монитор~--- набор процедур, переменных и других
        структур данных, объединенных в особый модуль или пакет
        \item Процессы могут вызывать процедуры монитора, но у процедур
        вне монитора нет доступа ко внутренним структурам данных
        монитора
        \item Свойство монитора: при обращении к монитору активным
        может быть только один процесс
        \item Обычно при вызове монитора первые несколько команд
        проверяют, нет ли в мониторе активных процессов
        \item Бринч Ханcен, Чарльз Хоар
        \end{tiny}
    \end{itemize}
    \end{spacing}
    \begin{tiny}
        \begin{verbatim}
            monitor account {
                int balance := 0

                function withdraw(int amount){
                    if amount <0 then error "Отрицательный счет"
                    else if balance < amount the error "Недостаточно средств"
                    else balance := balance - amount
                }
                function deposit(int amount){
                    if amount < 0 then error "Отрицательная сумма"
                    else balance := balance + amount
                }
            }
        \end{verbatim}
    \end{tiny}
\end{frame}

% Слайд Передача сообщений
\begin{frame}{Передача сообщений}
    \begin{itemize}
        \item Рассмотренные примитивы не применимы в распределенных системах
        \item Вариант: передача сообщений
        \begin{itemize}
            \item send (destination, message): посылает сообщение приемнику
            \item recieve(source, message): получает сообщение от источника
        \end{itemize}
        \item Проблемы:
        \begin{itemize}
            \item Подтверждение приема
            \item Аутентификация
            \item Производительность
        \end{itemize}
    \end{itemize}
\end{frame}

% Слайд Барьеры
\begin{frame}{Барьеры}
    \begin{itemize}
        \begin{footnotesize}
        \item Барьер~--- метод синхронизации в распределённых вычислениях, при
        котором выполнение параллельного алгоритма или его части можно
        разделить на несколько этапов, разделённых барьерами
        \item Идея заключается в том, чтобы в определенной точке ожидания
        собралось заданное число потоков управления
        \item Функции
        \begin{itemize}
            \begin{footnotesize}
            \item Инициализация и удаление (разрушение) барьеров
            \item Синхронизация на барьере
            \item Инициализация и удаление (разрушение) атрибутных объектов
            барьеров
            \item Опрос и установка атрибутов барьеров
            \end{footnotesize}
        \end{itemize}
        \item Когда у барьера собирается заданное число потоков, одному из них
        передается именованная константа, а другим~--- нули
        \item После этого барьер возвращается в начальное (инициализированное)
        состояние, а выделенный поток может выполнить соответствующие
        объединительные действия.
        \end{footnotesize}
    \end{itemize}
\end{frame}

\end{document}