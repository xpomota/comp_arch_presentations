\documentclass[aspectratio=169,14pt]{beamer}

\usepackage[utf8]{inputenc}
\usepackage[main=russian,english]{babel}
\usepackage[T1]{fontenc}
\usepackage[labelformat=empty]{caption}
\usepackage{multimedia}
\usepackage{xcolor}
\usepackage{setspace}

\usetheme{Singapore}

\definecolor{urlcolor}{HTML}{799B03} % цвет гиперссылок
\hypersetup{urlcolor=urlcolor, colorlinks=true}

\graphicspath{{../Images/}}

\title{Видеосистема}

\begin{document}

% Слайд Видеосистема
\begin{frame}{Видеосистема}
    \begin{itemize}
        \item Предназначена для вывода обновляющейся информации
        (текстовой/символьной и графической) в удобном виде на
        экран монитора
        \item Состоит из монитора и графического адаптера
        \item Графический адаптер~--- устройство, которое служит для
        программного формирования графических и текстовых изображений
        и передачи соответствующего сигнала на монитор
        \item Видеосистема ориентирована на растровый метод построения
        изображений~--- по точкам
    \end{itemize}
\end{frame}

% Слайд Электроно-лучевая трубка
\begin{frame}{Электронно-лучевая трубка}
    \begin{itemize}
        \begin{footnotesize}
        \item CRT (Cathode Ray Tube)~--- монитор на электронно-лучевой трубке
        \item Электронный прожектор (пушка) <<выстреливает>> поток электронов на экран
        \item Поверхность экрана со стороны потока электронов покрыта люминесцентным слоем, который обеспечивает кратковременную засветку части экрана (точка), на которую попадает поток электронов
        \item Поток совершает горизонтальные и вертикальные перемещения под управлением генераторов горизонтальной и вертикальной развертки
        \item Поток оставляет след только во время прямого хода и гасится во время обратного
        \item Сетка служит для управления интенсивностью потока (вплоть до полного гашения)
        \end{footnotesize}
        \begin{figure}
            \centering
            \includegraphics[height=2.5cm, keepaspectratio]{CRTGun.png}
            \includegraphics[height=2.5cm, keepaspectratio]{CRTScan.png}
        \end{figure}
    \end{itemize}
\end{frame}

% Слайд Жидкокристаллический монитор
\begin{frame}{Жидкокристаллический монитор}
    \begin{itemize}
        \begin{footnotesize}
        \item LCD (Liquid Crystal Display)~--- дисплей на основе жидких
        кристаллов
        \item Жидкие кристаллы~--- органические молекулы, которые
        двигаются как жидкость, но имеют структуру, как у кристалла
        \item Молекулы жидких кристаллов расположены между двумя
        поляризационными фильтрами таким образом, что изменяют
        поляризацию светового потока, после чего он проходит через
        второй фильтр без изменений
        \item Если приложить напряжение на электроды, то
        жидкокристаллические молекулы меняют свою пространственную
        ориентацию, что приводит к изменению прозрачности всей
        структуры
        \end{footnotesize}
        \begin{figure}
            \centering
            \includegraphics[height=3.0cm, keepaspectratio]{LCDSubpixel.png}
        \end{figure}
    \end{itemize}
\end{frame}

% Слайд Видеопамять
\begin{frame}{Видеопамять}
    \begin{columns}
        \begin{column}{0.5\textwidth}
            \begin{itemize}
                \begin{footnotesize}
                \begin{spacing}{0.9}
                \item Видеопамять~--- внутренняя оперативная память,
                отведенная для хранения данных, которые используются
                для формирования изображения на экране дисплея
                \item Обновление изображения осуществляется с
                частотой от 50 до 100Гц. Комфортная частота
                обновления дисплея~--- 25 кадров в секунду
                \item Чересстрочная развёртка~--- каждый кадр
                разбивается на два полукадра, составленные из строк,
                выбранных через одну
                \item Прогрессивная развёртка~--- все строки кадра
                передаются последовательно
                \end{spacing}
            \end{footnotesize}
            \end{itemize}
        \end{column}
        \begin{column}{0.5\textwidth}
            \begin{tiny}
                \begin{tabular}{|p{0.8cm}|p{2cm}|p{3cm}|}
                    \hline
                    \textbf{Адрес} & \textbf{Название} & \textbf{Назначение} \\ \hline
                    000000-07FFFF & 512КБ system memory board & Начальные 512КБ памяти на системной плате. \\ \hline
                    080000-09FFFF & 128КБ system memory board & Память на системной плате (512КБ-640КБ). Может быть отключена переключателем J10. \\ \hline
                    0A0000-0BFFFF & 128КБ video RAM & Зарезервировано для буфера видеоадаптера. \\ \hline
                    0С0000-0DFFFF & 128КБ I/O expansion ROM & Зарезервировано для ПЗУ адаптеров ввода/вывода. \\ \hline
                    0E0000-0EFFFF & 64КБ reserved on system board & Дубликат кода из адресов FE0000. \\ \hline
                    0F0000-0FFFFF & 64КБ ROM on the system board & Дубликат кода из адресов FF0000. \\ \hline
                    100000-FDFFFF & Maximum memory 15M & Каналы ввода/вывода в память на микросхемах расширения памяти (640КБ-15М). \\ \hline
                    FE0000-FEFFFF & 64КБ reserved on system board & Дубликат кода из адресов 0E0000. \\ \hline
                    FF0000-FFFFFF & 64КБ reserved on system board & Дубликат кода из адресов 0F0000. \\ \hline
                \end{tabular}
            \end{tiny}
            % \includegraphics[height=5.0cm, keepaspectratio]{80286SystemMemoryMap.png}
        \end{column}
    \end{columns}
\end{frame}

% Слайд Адаптер EGA/VGA
\begin{frame}{Адаптер EGA/VGA}
    \begin{itemize}
        \begin{footnotesize}
        \begin{spacing}{0.9}
        \item \textbf{Видеопамять}. В видеопамяти размещаются данные,
        отображаемые адаптером на экране дисплея. Видеопамять находится
        в адресном пространстве процессора и программы могут непосредственно
        производить с ней обмен данными.
        \item \textbf{Графический контроллер}. Служит для обмена данными
        между центральным процессором компьютера и видеопамятью.
        \item \textbf{Последовательный преобразователь}. Выбирает из
        видеопамяти один или несколько байт, преобразует их в
        последовательный поток битов и затем передает контроллеру
        атрибутов.
        \item \textbf{Контроллер атрибутов}. Преобразует информацию о
        цветах из формата, в котором она хранится в видеопамяти, в
        формат, необходимый для ЭЛТ.
        \item \textbf{Контроллер ЭЛТ}. Генерирует временные синхросигналы,
        управляющие ЭЛТ.
        \item \textbf{Синхронизатор}. Управляет всеми временными параметрами
        видеоадаптера. Синхронизатор также управляет доступом процессора
        к цветовым слоям видеоадаптера.
        \end{spacing}
        \end{footnotesize}
    \end{itemize}
\end{frame}

% Слайд Видеопамять VGA
\begin{frame}{Видеопамять VGA}
    \begin{itemize}
        \begin{footnotesize}
        \begin{spacing}{1}
        \item Видеопамять (256КБ) разделена на четыре банка (слоя) по 64КБ.
        Прямое отображение всей видеопамяти в адресное пространство
        процессора невозможно (объем видеопамяти больше 64КБ).
        \item Слои расположены в одном (одинаковом) адресном пространстве,
        по одному и тому же адресу доступны сразу четыре байта из
        разных слоев.
        \item Начальный адрес и размер буфера может изменяться в
        зависимости от видеорежима (A0000H, B0000H, B8000H; 32КБ-64КБ).
        \item Возможна запись сразу во все слои одной инструкцией
        процессора.
        \item Можно запрещать и разрешать запись в отдельные слои.
        \item Чтение разрешено только из одного слоя, определяется
        соответствующим регистром.
        \item Каждый из слоев разбивается на страницы (в разных
        режимах разное количество страниц). Активная страница служит
        источником изображения на экране. Активная страница определяется
        адресом в специальном регистре~--- возможно переключение страниц.
        \end{spacing}
        \end{footnotesize}
    \end{itemize}
\end{frame}

% Слайд Видеопамять VGA - режим 0 (текстовый)
\begin{frame}{Видеопамять VGA~--- режим 0 (текстовый)}
    Режимы 0 и 1: 16 цветов, текстовый, 40x25, символ 8x8
    \begin{columns}
        \begin{column}{0.6\textwidth}
            \begin{figure}
                \centering
                \includegraphics[width=7cm, keepaspectratio]{VGAVideoMemoryTextMode.png}
                \caption{\tiny{Программирование видеоадаптеров CGA, EGA и VGA / А.В.~Фролов, Г.В.~Фролов.~- 2-е изд., стер.~- М.~: АО~"Диалог-МИФИ", 1994. - 287~с. http://www.frolov-lib.ru/books/bsp/v03/index.html}}
            \end{figure}
        \end{column}
        \begin{column}{0.4\textwidth}
            \includegraphics[height=6cm, keepaspectratio]{VGAMemoryMode01.png}
        \end{column}
    \end{columns}
\end{frame}

% Слайд Видеопамять VGA - режим 10H (графический)
\begin{frame}{Видеопамять VGA~--- режим 10H (графический)}
    Режим 10H: графический, 640x350, 16~цветов (4~бита на цвет),
    каждый слой содержит бит изображения, С3~--- интенсивность
    \begin{columns}
        \begin{column}{0.65\textwidth}
            \begin{figure}
                \centering
                \includegraphics[width=8cm, keepaspectratio]{VGAVideoMemory10Mode.png}
                \caption{\tiny{Программирование видеоадаптеров CGA, EGA и VGA / А.В.~Фролов, Г.В.~Фролов.~- 2-е изд., стер.~- М.~: АО~"Диалог-МИФИ", 1994. - 287~с. http://www.frolov-lib.ru/books/bsp/v03/index.html}}
            \end{figure}
        \end{column}
        \begin{column}{0.35\textwidth}
            \includegraphics[height=5.8cm, keepaspectratio]{VGAMemoryMode10.png}
        \end{column}
    \end{columns}
\end{frame}


% Слайд Видеопамять VGA - режим 13H (графический)
\begin{frame}{Видеопамять VGA~--- режим 13H (графический)}
    Режим 13H: только VGA, графический, 320x200, 256 цветов
    \begin{figure}
        \centering
        \includegraphics[height=6cm, keepaspectratio]{VGAMemoryMode13.png}
    \end{figure}
\end{frame}

% Слайд Текстовый режим
\begin{frame}{Текстовый режим}
    \begin{itemize}
        \begin{footnotesize}
        \begin{spacing}{0.8}
        \item Экран представлен в виде решетки знакомест, в каждом из знакомест
        может находится один символ из ограниченного набора символов
        \item Изображение генерируется динамически из изображений символов
        \item Две области видеопамяти:
        \begin{itemize}
            \begin{footnotesize}
            \item Текстовый буфер (массив по количеству знакомест). Для каждого
            знакоместа хранится код символа и его атрибуты (цвет символа, цвет
            фона, флаги инверсии, яркости, подчеркивания, мигания и т.п.)
            \item Шрифт (изображение всех символов из набора)
            \end{footnotesize}
        \end{itemize}
        \item В видеоадаптере есть два счетчика: Y (строки)  и X (пиксели в
        строке). Эти счетчики делятся с остатком на размер знакоместа.
        Полученные частные~--- координаты в текстовом буфере,
        остатки~--- координаты в шрифте.
        \item По полученным координатам извлекается код символа и атрибуты, а
        также изображение символа (0~--- в этой позиции фон, 1~--- изображение)
        \end{spacing}
        \end{footnotesize}
    \end{itemize}
    \begin{figure}
        \centering
        \includegraphics[height=2.2cm, keepaspectratio]{CharGenerator.png}
    \end{figure}
\end{frame}

% Слайд Графический режим
\begin{frame}{Графический режим}
    \begin{itemize}
        \item В этом режиме осуществляется управление свечением каждой точки
        на экране
        \item Каждой точке экрана (пикселю, PEL, Picture Element)
        соответствует ячейка специальной памяти~--- видеопамять
        \item Изображение формируется в памяти графического адаптера
        центральным процессором, что требует пересылки большого объема
        информации.
        \item Видеопамять большую часть времени занята выдачей информации
        схемам регенерации изображений, освобождается только во время
        обратного хода луча
    \end{itemize}
\end{frame}

% Слайд Графический ускоритель
\begin{frame}{Графический ускоритель}
    \begin{itemize}
        \begin{footnotesize}
        \item Для решения проблем передачи большого количества данных в
        видеопамять можно:
        \begin{itemize}
            \begin{footnotesize}
            \item повышать ее быстродействие;
            \item расширять разрядности шин графического адаптера;
            \item кэшировать видеопамять; в этом случае данные будут записаны
            как в видеопамять, так и в кэш, а при считывании из этой
            области обращение будет только к кэшу;
            \item сократить текущий объём информации, передаваемой
            графическому адаптеру.
            \end{footnotesize}
        \end{itemize}
        \item Графический ускоритель~--- адаптер, в составе которого
        есть специализированный процессор, формирующий изображение в
        видеопамяти
        \item Графический процессор осуществляет \textbf{рендеринг}
        (получение изображения по модели) графического конвейера
        OpenGL или DirectX
        \item Команды рисования графических примитивов, заливка,
        копирование блока (прокрутка), аппаратная поддержка окон,
        панорамирование
        \end{footnotesize}
    \end{itemize}
\end{frame}

% Слайд OpenGL
\begin{frame}{OpenGL}
    \begin{itemize}
        \begin{footnotesize}
        \begin{spacing}{0.9}
        \item OpenGL (Open Graphics Library) — спецификация, определяющая
        платформонезависимый (независимый от языка программирования)
        программный интерфейс для написания приложений, использующих
        двумерную и трёхмерную компьютерную графику
        \item Более 300 функций для рисования сложных трёхмерных сцен из
        простых примитивов
        \item Основным принципом работы OpenGL является получение наборов
        векторных графических примитивов в виде точек, линий и
        треугольников с последующей математической обработкой полученных
        данных и построением растровой картинки на экране и/или в памяти
        \item Векторные трансформации и растеризация выполняются
        графическим конвейером (graphics pipeline)
        \item Абсолютное большинство команд OpenGL попадает в одну из
        двух групп: либо они добавляют графические примитивы на вход
        в конвейер, либо конфигурируют конвейер на различное исполнение
        трансформаций
        \item Рендеринг~--- процесс получения изображения по модели
        \item ~<<Шейдеры>>~--- программы, предназначенные для выполнения
        на процессорах видеокарты
        \end{spacing}
        \end{footnotesize}
    \end{itemize}
\end{frame}

% Слайд Конвейер OpenGL
\begin{frame}{Конвейер OpenGL}
    \begin{columns}
        \begin{column}{0.8\textwidth}
            \begin{itemize}
                \begin{footnotesize}
                \begin{spacing}{0.8}
                \item Vertex Specification~--- упорядоченный список
                вершин, которые определяют границы
                примитивов (треугольники, линии, точки).
                \item Vertex Shader~--- основная обработку отдельной вершины.
                \item Tesselation~--- шейдер для разделения на более мелкие примитивы.
                \item Geometry Shader~--- шейдер преобразования графических
                примитивов.
                \item Vertex Post-Processing~--- после обработки шейдерами вершины
                подвергаются обработке фиксированными функциями (Transform
                Feedback, Clippping)
                \item Primitive assembly~--- сборка обработанных вершин и их
                компоновка в последовательность примитивов.
                \item Rasterization~--- получение точечного изображения
                \item Fragment Shader~--- применение шейдеров к отдельным
                фрагментам изображения
                \item Per-Sample Operation~--- дополнительная обработка фрагментов
                изображения (выбраковочные тесты, цветосмешение, маскирование).
                \end{spacing}
                \end{footnotesize}
            \end{itemize}
        \end{column}
        \begin{column}{0.2\textwidth}
            \includegraphics[height=6.2cm, keepaspectratio]{OpenGLRenderingPipeline.png}
        \end{column}
    \end{columns}
\end{frame}

% Слайд Характеристики внешней памяти
\begin{frame}{Характеристики внешней памяти}
    \begin{itemize}
        \item Ёмкость (объем)~--- Кб-Тб
        \item Сменные или фиксированные носители
        \item Время доступа~--- усредненный интервал от выдачи запроса до
        фактического начала передачи данных
        \item Для устройств с подвижными носителями время доступа
        (Access Time) = время позиционирования, поиска (Seek Time)~+
        задержка, ожидание времени подхода участка (Latency)
        \item Скорость передачи данных~--- производительность обмена
        после выполнения поиска данных
    \end{itemize}
\end{frame}

% Слайд Дисковые накопители
\begin{frame}{Дисковые накопители}
    \begin{itemize}
        \begin{footnotesize}
            \item Информация располагается на дорожках (трека), каждый трек
            разбит на сектора фиксированного размера.
            \item Сектор~--- минимальный блок информации для чтения/записи
            (512-4096 байт)
            \item Преамбула~--- служебная информация: позволяет головке
            синхронизироваться перед операциями чтения/записи
            \item Цилиндр~--- совокупность <<одинаковых>> (по удалению
            от центра) треков на всех головках
            \item Головка для каждой поверхности своя
        \end{footnotesize}
    \end{itemize}
    \begin{figure}
        \centering
        \includegraphics[height=3.2cm, keepaspectratio]{HDDTrack.png}
        \includegraphics[height=3.2cm, keepaspectratio]{HDDSeveralHeads.png}
    \end{figure}
\end{frame}

% Слайд Геометрия дискового накопителя
\begin{frame}{Геометрия дискового накопителя}
    \begin{itemize}
        \begin{footnotesize}
        \item Физическая геометрия: адресация CHS (Cylinder Head
        Sector)~--- цилиндр, головка, сектор
        \item Логическая геометрия: физическая геометрия не вписывается
        в ограничения интерфейса; кроме того, появляются диски с
        дорожками с разным количеством секторов. Максимальные номера
        секторов и головок берутся равными 63 и 255 (ограничения
        прерывания BIOS INT 13H), число цилиндров подбирается в
        зависимости от емкости диска.
        \item LBA (Logical block addressing)~--- механизм логической
        адресации. Каждый блок, адресуемый на жестком диске, имеет
        свой LBA-номер.
        \end{footnotesize}
    \end{itemize}
    $$LBA=[(c\times H+h)\times S] + (s-1),$$
    \begin{footnotesize}где $c, h, s$~--- номера цилиндра, головки, сектора,
    $H$~--- количество головок, $S$~--- количество секторов на одной дорожке
    \end{footnotesize}
\end{frame}

% Слайд Контроллер жесткого диски
\begin{frame}{Контроллер жесткого диска}
    \begin{itemize}
        \begin{footnotesize}
        \item Контроллер~--- микросхема (набор микросхем), которая
        управляет диском
        \item Функции:
            \begin{itemize}
                \begin{footnotesize}
                \item Обработка команд (READ, WRITE, FORMAT)
                \item Управление перемещением кронштейна с головками
                \item Преобразование сигнала из аналогового в цифровой
                \item Обнаружение и исправление ошибок
                \item Буферизация и кэширование секторов
                \item Пропуск поврежденных секторов
                \end{footnotesize}
            \end{itemize}
        \item Форматирование диска
        \begin{itemize}
            \begin{footnotesize}
            \item низкоуровневое: базовая разметка области хранения данных,
            выполняет изготовитель
            \item разбиение на разделы: разбивает диск на логические
            разделы
            \item высокоуровневое форматирование: создание файловой
            системы
            \end{footnotesize}
        \end{itemize}
    \end{footnotesize}
\end{itemize}
\end{frame}

% Слайд Параметры дисковых накопителей
\begin{frame}{Параметры дисковых накопителей}
    \begin{itemize}
        \begin{footnotesize}
        \item Интерфейс (ATA, SATA, SCSI,~\ldots)
        \item Ёмкость (форматированная, неформатированная)
        \item Физический размер (форм-фактор)
        \item Время произвольного доступа (random access time)~---
        среднее время, за которое винчестер выполняет операцию
        позиционирования головки чтения/записи на произвольный участок
        магнитного диска (2,5 до 16 мс)
        \item Скорость вращения шпинделя (spindle speed)~--- количество
        оборотов шпинделя в минуту. От этого параметра в значительной
        степени зависят время доступа и средняя скорость передачи
        данных (5400~--- 15000 об/c)
        \item Скорость передачи данных (Transfer Rate) при
        последовательном доступе
        \item Среднее время поиска
        \item Максимальное время поиска
        \end{footnotesize}
    \end{itemize}
\end{frame}

% Слайд Функции BIOS 13H, 19H
\begin{frame}{Функции BIOS 13H, 19H}
    \begin{itemize}
        \begin{footnotesize}
        \item Сервис 13H работает на уровне физических устройств и
        изначально адресовался непосредственно к цилиндрам, головкам и
        секторам (позднее появилась подфункция, принимавшими номер
        сектора как 64-битное целое число (LBA) без деления на CHS)
        \begin{itemize}
            \begin{footnotesize}
            \item Состояние устройства
            \item Чтение сектора с диска в память
            \item Запись сектора из памяти на диск
            \item Получение параметров устройства
            \end{footnotesize}
        \end{itemize}
        \item Начальный загрузчик 19H
        \begin{itemize}
            \begin{footnotesize}
            \begin{spacing}{0.9}
            \item Происходит чтение 512 байт первого сектора (Master Boot
            Record) в ОЗУ по адресу 0x00007C00
            \item Выполняется инструкция перехода по этому адресу
            \item Этот код читает и анализирует таблицу разделов
            жёсткого диска, а затем, в зависимости от вида загрузчика,
            либо передаёт управление загрузочному коду активного раздела
            жёсткого диска, либо самостоятельно загружает ядро с диска
            (например, сетевого или съёмного) в оперативную память и
            передаёт ему управление.
            \end{spacing}
            \end{footnotesize}
        \end{itemize}
        \end{footnotesize}
    \end{itemize}
\end{frame}

% Слайд Главная загрузочная запись
\begin{frame}{Главная загрузочная запись}
    \begin{itemize}
        \begin{footnotesize}
        \item Главная загрузочная запись (Master Boot Record, MBR)~--- код
        и данные, необходимые для последующей загрузки операционной
        системы и расположенные в первых физических секторах (чаще всего
        в самом первом) на жёстком диске или другом устройстве хранения
        информации.
        \item Cодержит небольшой фрагмент исполняемого кода, таблицу
        разделов (partition table) и специальную сигнатуру.
        \end{footnotesize}
    \end{itemize}
    \begin{figure}
        \centering
        \includegraphics[height=3.8cm, keepaspectratio]{MBRStructure.png}
        \includegraphics[height=3.8cm, keepaspectratio]{PartitionTableStructure.png}
    \end{figure}
\end{frame}

% Слайд GPT https://habr.com/ru/post/347002/
\begin{frame}{GUID Partition Table (GPT)}
    Только LBA адресация, 64 разряда на адрес. Используется в UEFI (Unified
    Extensible Firmware Interface)~--- замене BIOS.
    \begin{tiny}
    \begin{tabular}{|p{1cm}|p{1.2cm}|p{3cm}|}
        \hline
        \textbf{LBA-адрес} & \textbf{Размер (секторов)} & \textbf{Назначение} \\ \hline
        LBA 0 & 1 & Защитный MBR-сектор \\ \hline
        LBA 1 & 1 & Первичный GPT-заголовок \\ \hline
        LBA 2 & 32 & Таблица разделов диска \\ \hline
        LBA 34 & NN & Содержимое разделов диска \\ \hline
        LBA -34 & 32 & Копия таблицы разделов диска \\ \hline
        LBA -2 & 1 & Копия GPT-заголовка \\ \hline
    \end{tabular}
    \end{tiny}
    \begin{itemize}
        \begin{footnotesize}
        \item Защитный MBR-сектор используется для совместимости
        \item Первичный GTP-заголовок содержит данные о всех
        LBA-адресах, использующихся для разметки диска на разделы
        \item Таблица разделов диска содержит записи (128Б) о разделах (128).
        Тип, GUID, начальный и конечный адрес, название, атрибуты.
        \end{footnotesize}
    \end{itemize}
\end{frame}

% Слайд Интерфейс ATA/IDE
\begin{frame}{Интерфейс ATA/IDE}
    \begin{itemize}
        \begin{footnotesize}
        \begin{spacing}{0.8}
            \item Интерфейс~--- техническое средство взаимодействия 2-х
            разнородных устройств: совокупность линий связи, сигналов,
            посылаемых по этим линиям, технических средств, поддерживающих
            эти линии (контроллеры интерфейсов) и правил (протокола) обмена
            \item IDE (Integrated Drive Electronics)~--- <<электроника,
            встроенная в привод>>: контроллер привода располагается в
            самом приводе, а не в виде отдельной платы расширения
            (первоначальное название)
            \begin{itemize}
                \begin{tiny}
                \begin{spacing}{0.9}
                \item улучшение характеристик накопителей (за счёт меньшего
                расстояния до контроллера);
                \item упрощение управления контроллером (так как контроллер
                канала IDE абстрагировался от деталей работы привода)
                \item удешевление производства (контроллер привода мог быть
                рассчитан только на «свой» привод, а не на все возможные;
                контроллер канала же вообще становился стандартным)
                \end{spacing}
                \end{tiny}
            \end{itemize}
            \item 8 регистров, занимающих 8 адресов в пространстве ввода-вывода.
            Ширина шины данных составляет 16 бит. Количество каналов может
            быть больше 2. К каждому каналу можно подключить два устройства
            (master и slave),одновременно работает только один канал
            \item EIDE (Enhanced IDE)~--- <<расширенный IDE>>), позволял
            использование приводов ёмкостью более 528Мб, до 8,4Гб,
            поддерживают LBA.
            \item ATA/ATAPI (Advanced Technology Attachment Packet
            Interface)~--- расширение стандарта для подключения CD-ROM
            и других устройств.
            \item SerialATA~--- последовательный ATA
        \end{spacing}
        \end{footnotesize}
    \end{itemize}
\end{frame}

% Слайд Семейство ATA
\begin{frame}{Семейство ATA}
    \begin{figure}
        \centering
        \includegraphics[height=6.4cm, keepaspectratio]{ATAFamily.png}
    \end{figure}
\end{frame}

% Слайд Интерфейс SCSI
\begin{frame}{Интерфейс SCSI}
    \begin{itemize}
        \begin{tiny}
        \begin{spacing}{1}
            \item SCSI (Small Computer System Interface) - набор стандартов для физического подключения и передачи данных между компьютерами и периферийными устройствами
            \item Разработан для объединения на одной шине различных устройств: жёсткие диски, накопители на магнитооптических дисках, приводы CD, DVD, стримеры, сканеры, принтеры.
            \item Взаимодействие идёт между инициатором и целевым устройством. Инициатор посылает команду целевому устройству, которое затем отправляет ответ инициатору.
            \item Команды SCSI посылаются в виде блоков описания команды (Command Descriptor Block, CDB). Длина каждого блока может составлять 6, 10, 12, 16 или 32 байта. В последних версиях SCSI блок может иметь переменную длину. Блок состоит из однобайтового кода команды и параметров команды.
            \item Команды SCSI делятся на четыре категории: N (non-data), W (запись данных от инициатора целевым устройством), R (чтение данных) и B (двусторонний обмен данными)
        \end{spacing}
            \begin{itemize}
                \begin{tiny}
                \begin{spacing}{1}
                \item Test unit ready: проверка готовности устройства, в том числе наличия диска в дисководе.
                \item Inquiry: запрос основных характеристик устройства.
                \item Send diagnostic: указание устройству провести самодиагностику и вернуть результат.
                \item Request sense — возвращает код ошибки предыдущей команды.
                \item Read capacity — возвращает ёмкость устройства.
                \item Format Unit
                \item Read/Write (4 варианта) — чтение/запись.
                \item Write and verify — запись и проверка.
                \item Mode select — установка параметров устройства.
                \item Mode sense — возвращает текущие параметры устройства.
                \end{spacing}
                \end{tiny}
            \end{itemize}
        \begin{spacing}{1}
            \item Каждое устройство на SCSI-шине имеет как минимум один номер логического устройства
        \end{spacing}
        \end{tiny}
    \end{itemize}
\end{frame}

% Слайд Семейство SCSI
\begin{frame}{Семейство SCSI}
    \begin{figure}
        \centering
        \includegraphics[height=5cm, keepaspectratio]{SCSIFamily.png}
    \end{figure}
\end{frame}

% Слайд Твердотельные накопители
\begin{frame}{Твердотельные накопители}
    \begin{itemize}
        \begin{footnotesize}
        \item Твердотельный накопитель (англ. Solid-State Drive, SSD) — немеханическое запоминающее устройство на основе микросхем памяти.
        \item SSD содержит управляющий контроллер.
        \item Преимущества
        \begin{itemize}
        \begin{tiny}
            \item Отсутствие движущихся частей
            \item Стабильность времени считывания файлов вне зависимости от их расположения или фрагментации
            \item Скорость чтения/записи выше, чем у распространенных жёстких дисков
            \item Количество произвольных операций ввода-вывода в секунду (IOPS) у SSD на порядок (на несколько порядков в случае записи) выше, чем у жёстких дисков
            \item Низкое энергопотребление
            \item Намного меньшая чувствительность к внешним электромагнитным полям
            \item Малые габариты и вес
        \end{tiny}
        \end{itemize}
        \item Недостатки
        \begin{itemize}
        \begin{footnotesize}
            \item ограниченное количество циклов перезаписи;
            \item цена
            \item снижение производительности при заполнении
            \item сложность в восстановлении информации в случае выхода из строя
        \end{footnotesize}
        \end{itemize}
        \item SerialATA – последовательный ATA
        \end{footnotesize}
    \end{itemize}
\end{frame}

\end{document}